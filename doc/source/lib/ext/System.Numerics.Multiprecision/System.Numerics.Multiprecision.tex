\documentclass[a4paper,oneside,11.000000pt]{book}

\usepackage{array}
\usepackage{amsmath}
\usepackage{amsfonts}
\usepackage{amssymb}
\usepackage{graphicx}
\usepackage{pifont}
\usepackage{a4wide}
\usepackage{url}
\usepackage{float}
\usepackage{layouts}
\usepackage{supertabular}
\usepackage{titlesec}
\usepackage{listings}
\usepackage{syntax}
\usepackage[colorlinks=true,linkcolor=blue]{hyperref}
\setcounter{secnumdepth}{10}
\setcounter{tocdepth}{4}

\titleformat{\chapter}[hang]{\bf\Huge}{\arabic{chapter}}{1em}{}{}
\titleclass{\subchapter}{straight}[\chapter]
\newcounter{subchapter}
\renewcommand{\thesubchapter}{\thechapter.\arabic{subchapter}}
\titleformat{\subchapter}[hang]{\bf\huge}{\thesubchapter}{0.5em}{}{}
\titlespacing{\subchapter}{0pt}{0pt}{0pt}
\renewcommand\thesection{\thesubchapter.\arabic{section}}
\makeatletter
\newcommand*\l@subchapter{\@dottedtocline{1}{1.5em}{2.3em}}
\renewcommand\l@section{\@dottedtocline{2}{3.8em}{3.2em}}
\renewcommand\l@subsection{\@dottedtocline{3}{7em}{4.1em}}
\renewcommand\l@subsubsection{\@dottedtocline{4}{11.1em}{5em}}
\makeatother

\makeatletter
\renewcommand\subparagraph{\@startsection{subparagraph}{5}{0mm}{-\baselineskip}{0.5\baselineskip}{\normalfont\normalsize\scshape}}
\makeatother

\lstdefinelanguage{Cmajor}{morekeywords={bool, true, false, sbyte, byte, short, ushort, int, uint, long, ulong, float, double, char, void, enum, cast, namespace, using, static, extern, is, explicit, delegate, inline, cdecl, nothrow, public, protected, private, internal, virtual, abstract, override, suppress, default, operator, class, return, if, else, switch, case, default, while, do, for, break, continue, goto, typedef, typename, const, null, this, base, construct, destroy, new, delete, sizeof, try, catch, throw, concept, where, axiom, and, or, not},sensitive=true,morecomment=[l]{//},morecomment=[s]{/*}{*/},morestring=[b]",morestring=[b]',}\lstloadlanguages{Cmajor}
\lstset{language=Cmajor,showstringspaces=false,breaklines=true,basicstyle=\small}
\makeatletter\def\toclevel@chapter{0}\def\toclevel@subchapter{1}\def\toclevel@section{2}\def\toclevel@subsection{3}\def\toclevel@subsubsection{4}\makeatother\begin{document}
\clearpage

\frontmatter
\title{\textsc{System.Numerics.Multiprecision Library Reference}
}
\date{\today}
\maketitle
\tableofcontents

\clearpage
\chapter{Description}
\begin{flushleft}
\hyperlink{System.Numerics.Multiprecision}{Multiprecision} is a multiple precision arithmetic library.
The library is implemented using GNU Multiple Precision Arithmetic Library 
(\url{http://gmplib.org/}).

\end{flushleft}
\chapter{Namespaces}
\begin{flushleft}
\begin{supertabular}[l]{!{\raggedright}p{7.44777cm}!{\raggedright}p{7.70223cm}}
\textbf{Namespace}
& \textbf{Description}
\\
\hline
\hyperlink{System.Numerics.Multiprecision}{System.Numerics.Multiprecision}
& Contains arbitrary precision integer type \hyperlink{System.Numerics.Multiprecision.BigInt}{BigInt}, arbitrary precision rational type \hyperlink{System.Numerics.Multiprecision.BigRational}{BigRational} and 
arbitrary precision floating point type \hyperlink{System.Numerics.Multiprecision.BigFloat}{BigFloat}.

\\
\end{supertabular}

\end{flushleft}
\clearpage
\mainmatter

\chapter{Usage}

\section{Referencing the System.Numerics.Multiprecision library}

Right-click a project node in IDE \verb.|. Project References... \verb.|.
Add System Extension Library Reference... \verb.|.
enable \emph{System.Numerics.Multiprecision} check box

\begin{flushleft}
or add following line to your project's .cmp file:
\begin{verbatim}
reference <ext/System.Numerics.Multiprecision/System.Numerics.Multiprecision.cml>;
\end{verbatim}
\end{flushleft}

\chapter{Installation}

The System.Numerics.Multiprecision library uses the GNU MP library
that is precompiled for the following platforms:

\begin{itemize}
\item 32-bit Windows (x86)
\item 64-bit Windows (x64)
\item 32-bit PC Linux (i686)
\item 64-bit PC Linux (x86\_64)
\end{itemize}

Current version of GNU MP library at the time of writing is 6.1.0.

If you want to install newer version of the GNU MP library or
if you have different platform, you can compile the GNU MP library from sources
by using the following instructions.

\section{Compiling the GNU MP library from sources in Windows}

\begin{itemize}

\item
Install MinGW-w64 GCC to a path that does not contain spaces.
I have mine installed in \verb|C:\mingw-w64|.

Installer can be obtained from\\
\url{http://sourceforge.net/projects/mingw-w64/files/Toolchains%20targetting%20Win32/Personal%20Builds/mingw-builds/installer/mingw-w64-install.exe/download}

\item
Install MSYS2.
Installation instructions can be found in\\
\url{http://sourceforge.net/p/msys2/wiki/MSYS2%20installation/}

I have mine installed in \verb|C:\msys64|.

\item
Start MSYS2 shell.

\item
In the MSYS2 shell obtain \textbf{tar} by executing
\begin{verbatim}
pacman -S tar
\end{verbatim}

\item
Obtain \textbf{make} by executing
\begin{verbatim}
pacman -S make
\end{verbatim}

\item
Mount your MinGW-directory (mine is in \verb|C:/mingw-w64|) by executing
\begin{verbatim}
mkdir /mingw
mount C:/mingw-w64 /mingw
\end{verbatim}

\item
Insert your MinGW bin-directory in front of the PATH by executing
\begin{verbatim}
export PATH=/mingw/mingw64/bin:$PATH
\end{verbatim}

\item
Now if you ask which GCC is used by executing
\begin{verbatim}
which gcc
\end{verbatim}
you should get \textbf{/mingw/mingw64/bin/gcc}.

\item
Download the GNU MP library from \url{http://gmplib.org/#DOWNLOAD} in bz2 format
and place it to your MSYS2 home directory (mine is in \verb|C:\msys64\home\Seppo|).

\item
Extract the GNU MP library by executing
\begin{verbatim}
tar xjf gmp-6.1.0.tar.bz2
\end{verbatim}

\item
Change to gmp-directory by executing
\begin{verbatim}
cd gmp-6.1.0
\end{verbatim}

\item
Configure the library for 64-bit Windows by executing
\begin{verbatim}
./configure --enable-static --disable-shared --host=x86_64-w64-mingw32
\end{verbatim}

For 32-bit Windows the command is
\begin{verbatim}
./configure --enable-static --disable-shared --host=i686-w64-mingw32
\end{verbatim}

Shared libraries are not used in Cmajor so we are disabling them.

\item
Make the library by executing
\begin{verbatim}
make
\end{verbatim}

\item
Install the library by executing
\begin{verbatim}
make install
\end{verbatim}

\item
Optionally test the library by executing
\begin{verbatim}
make check
\end{verbatim}

\item
Now the header file \textbf{gmp.h} should be in \verb|C:\msys64\usr\local\include| directory
and the library file \textbf{libgmp.a} should be in \verb|C:\msys64\usr\local\lib| directory.
Copy gmp.h and libgmp.a under the Cmajor System.Numerics.GmpIntf extension library to directory

\begin{verbatim}
%APPDATA%\Cmajor\system\ext\System.Numerics.GmpIntf\gmp\windows\x64
\end{verbatim}
for 64-bit Windows, or to
\begin{verbatim}
%APPDATA%\Cmajor\system\ext\System.Numerics.GmpIntf\gmp\windows\x86
\end{verbatim}
for 32-bit Windows.

\item
That's all.

\end{itemize}

\section{Compiling the GNU MP library from sources in Linux}

\begin{itemize}

\item
Download the GNU MP library from \url{http://gmplib.org/#DOWNLOAD} in bz2 format
and place it to your home directory.

\item
Extract the GNU MP library by executing
\begin{verbatim}
tar xjf gmp-6.1.0.tar.bz2
\end{verbatim}

\item
Change to gmp-directory by executing
\begin{verbatim}
cd gmp-6.1.0
\end{verbatim}

\item
Configure the library by executing
\begin{verbatim}
./configure --enable-static --disable-shared
\end{verbatim}

Shared libraries are not used in Cmajor so we are disabling them.

\item
Make the library by executing
\begin{verbatim}
make
\end{verbatim}

\item
Install the library by executing
\begin{verbatim}
make install
\end{verbatim}

\item
Optionally test the library by executing
\begin{verbatim}
make check
\end{verbatim}

\item
Now the header file \textbf{gmp.h} should be in \verb|/usr/local/include| directory
and the library file \textbf{libgmp.a} should be in \verb|/usr/local/lib| directory.
Copy gmp.h and libgmp.a under the Cmajor System.Numerics.GmpIntf extension library to 
\begin{verbatim}
<your cmajor directory>/system/ext/System.Numerics.GmpIntf/gmp/linux/x86_64
\end{verbatim}
directory for 64-bit Linux or to 
\begin{verbatim}
<your cmajor directory>/system/ext/System.Numerics.GmpIntf/gmp/linux/i686
\end{verbatim}
directory for 32-bit Linux.

\item
That's all.

\end{itemize}


\hypertarget{System.Numerics.Multiprecision}{\chapter{System.Numerics.Multiprecision Namespace}}
\begin{flushleft}
Contains arbitrary precision integer type \hyperlink{System.Numerics.Multiprecision.BigInt}{BigInt}, arbitrary precision rational type \hyperlink{System.Numerics.Multiprecision.BigRational}{BigRational} and 
arbitrary precision floating point type \hyperlink{System.Numerics.Multiprecision.BigFloat}{BigFloat}.

\end{flushleft}
\clearpage
\subchapter{Classes}
\begin{flushleft}
\begin{supertabular}[l]{ll}
\textbf{Class}
& \textbf{Description}
\\
\hline
\hyperlink{System.Numerics.Multiprecision.BigFloat}{BigFloat}
& An arbitrary precision floating point type.

\\
\hyperlink{System.Numerics.Multiprecision.BigInt}{BigInt}
& An arbitrary precision signed integer type.

\\
\hyperlink{System.Numerics.Multiprecision.BigRational}{BigRational}
& An arbitrary precision rational nunber type.

\\
\hyperlink{System.Numerics.Multiprecision.Precision}{Precision}
& Represents a precision of given number of digits.

\\
\end{supertabular}

\end{flushleft}
\clearpage

\hypertarget{System.Numerics.Multiprecision.BigFloat}{\section{BigFloat Class}}\begin{flushleft}
An arbitrary precision floating point type.

\end{flushleft}

\subsection*{Syntax}\texttt{public class BigFloat;}
\subsection{Member Functions}
\begin{flushleft}
\begin{supertabular}[l]{!{\raggedright}p{7.575cm}!{\raggedright}p{7.575cm}}
\textbf{Member Function}
& \textbf{Description}
\\
\hline
\hyperlink{System.Numerics.Multiprecision.BigFloat.constructor.P.System.Numerics.Multiprecision.BigFloat}{BigFloat()}
& Default constructor. Creates an instance of arbitrary precision floating point type and initializes it to zero.

\\
\hyperlink{System.Numerics.Multiprecision.BigFloat.constructor.P.System.Numerics.Multiprecision.BigFloat.C.R.System.Numerics.Multiprecision.BigFloat}{BigFloat(const System.\-Numerics.\-Multiprecision.\-BigFloat\&\-)}
& Copy constructor.

\\
\hyperlink{System.Numerics.Multiprecision.BigFloat.operator.assign.P.System.Numerics.Multiprecision.BigFloat.C.R.System.Numerics.Multiprecision.BigFloat}{operator=(const System.\-Numerics.\-Multiprecision.\-BigFloat\&\-)}
& Copy assignment.

\\
\hyperlink{System.Numerics.Multiprecision.BigFloat.constructor.P.System.Numerics.Multiprecision.BigFloat.RR.System.Numerics.Multiprecision.BigFloat}{BigFloat(System.\-Numerics.\-Multiprecision.\-BigFloat\&\-\&\-)}
& Move constructor.

\\
\hyperlink{System.Numerics.Multiprecision.BigFloat.operator.assign.P.System.Numerics.Multiprecision.BigFloat.RR.System.Numerics.Multiprecision.BigFloat}{operator=(System.\-Numerics.\-Multiprecision.\-BigFloat\&\-\&\-)}
& Move assignment.

\\
\hyperlink{System.Numerics.Multiprecision.BigFloat.constructor.P.System.Numerics.Multiprecision.BigFloat.C.R.System.Numerics.Multiprecision.BigInt}{BigFloat(const System.\-Numerics.\-Multiprecision.\-BigInt\&\-)}
& Constructor. Creates an instance of arbitrary precision floating point type and initializes it from the given arbitrary precision integer type.

\\
\hyperlink{System.Numerics.Multiprecision.BigFloat.constructor.P.System.Numerics.Multiprecision.BigFloat.C.R.System.Numerics.Multiprecision.BigRational}{BigFloat(const System.\-Numerics.\-Multiprecision.\-BigRational\&\-)}
& Constructor. Creates an instance of arbitrary precision floating point type and initializes it from the given arbitrary precision rational type.

\\
\hyperlink{System.Numerics.Multiprecision.BigFloat.constructor.P.System.Numerics.Multiprecision.BigFloat.C.R.System.Numerics.Multiprecision.Precision}{BigFloat(const System.\-Numerics.\-Multiprecision.\-Precision\&\-)}
& Constructor. Constructs an arbitrary precision floating point value with given precision.

\\
\hyperlink{System.Numerics.Multiprecision.BigFloat.constructor.P.System.Numerics.Multiprecision.BigFloat.C.R.System.String.char}{BigFloat(const System.\-String$<$\-char$>$\-\&\-)}
& Constructor. Constructs an arbitrary precision floating point value from given decimal digits.

\\
\hyperlink{System.Numerics.Multiprecision.BigFloat.constructor.P.System.Numerics.Multiprecision.BigFloat.C.R.System.String.char.int}{BigFloat(const System.\-String$<$\-char$>$\-\&\-, int)}
& Constructor. Constructs an arbitrary precision floating point value from given digits that are in given base.

\\
\hyperlink{System.Numerics.Multiprecision.BigFloat.constructor.P.System.Numerics.Multiprecision.BigFloat.double}{BigFloat(double)}
& Constructor. Constructs an arbitrary precision floating point value from given double precision value.

\\
\hyperlink{System.Numerics.Multiprecision.BigFloat.constructor.P.System.Numerics.Multiprecision.BigFloat.int}{BigFloat(int)}
& Constructor. Constructs an arbitrary precision floating point value from given integer value.

\\
\hyperlink{System.Numerics.Multiprecision.BigFloat.constructor.P.System.Numerics.Multiprecision.BigFloat.uint}{BigFloat(uint)}
& Constructor. Constructs an arbitrary precision floating point value from given unsigned integer value.

\\
\hyperlink{System.Numerics.Multiprecision.BigFloat.Handle.C.P.System.Numerics.Multiprecision.BigFloat}{Handle() const}
& Returns a handle to the GNU MP library arbitrary precision floating point number representation.

\\
\hyperlink{System.Numerics.Multiprecision.BigFloat.ToDouble.C.P.System.Numerics.Multiprecision.BigFloat}{ToDouble() const}
& Converts the arbitrary precision floating point number to a double precision value by truncating it.

\\
\hyperlink{System.Numerics.Multiprecision.BigFloat.ToString.C.P.System.Numerics.Multiprecision.BigFloat}{ToString() const}
& Returns the value of the \textbf{BigFloat}
 as a string.

\\
\hyperlink{System.Numerics.Multiprecision.BigFloat.ToString.C.P.System.Numerics.Multiprecision.BigFloat.int}{ToString(int) const}
& Returns the value of the \textbf{BigFloat}
 as a string using given base.

\\
\hyperlink{System.Numerics.Multiprecision.BigFloat.ToString.C.P.System.Numerics.Multiprecision.BigFloat.int.uint}{ToString(int, uint) const}
& Returns the value of the \textbf{BigFloat}
 as a string using given base and given number of digits.

\\
\hyperlink{System.Numerics.Multiprecision.BigFloat.operator.assign.P.System.Numerics.Multiprecision.BigFloat.C.R.System.Numerics.Multiprecision.BigInt}{operator=(const System.\-Numerics.\-Multiprecision.\-BigInt\&\-)}
& Assigns the value of the \textbf{BigFloat}
 from the given arbitrary precision integer.

\\
\hyperlink{System.Numerics.Multiprecision.BigFloat.operator.assign.P.System.Numerics.Multiprecision.BigFloat.C.R.System.Numerics.Multiprecision.BigRational}{operator=(const System.\-Numerics.\-Multiprecision.\-BigRational\&\-)}
& Assigns the value of the \textbf{BigFloat}
 from the given arbitrary precision rational.

\\
\hyperlink{System.Numerics.Multiprecision.BigFloat.operator.assign.P.System.Numerics.Multiprecision.BigFloat.double}{operator=(double)}
& Assigns the value of the \textbf{BigFloat}
 from the given double precision value.

\\
\hyperlink{System.Numerics.Multiprecision.BigFloat.operator.assign.P.System.Numerics.Multiprecision.BigFloat.int}{operator=(int)}
& Assigns the value of the \textbf{BigFloat}
 from the given integer value.

\\
\hyperlink{System.Numerics.Multiprecision.BigFloat.operator.assign.P.System.Numerics.Multiprecision.BigFloat.uint}{operator=(uint)}
& Assigns the value of the \textbf{BigFloat}
 from the given unsigned integer value.

\\
\hyperlink{System.Numerics.Multiprecision.BigFloat.destructor.P.System.Numerics.Multiprecision.BigFloat}{$\sim$BigFloat()}
& Frees memory occupied by the \textbf{BigFloat}
 instance.

\\
\end{supertabular}

\end{flushleft}
\clearpage

\hypertarget{System.Numerics.Multiprecision.BigFloat.constructor.P.System.Numerics.Multiprecision.BigFloat}{\subsubsection*{BigFloat() Member Function}}
\begin{flushleft}
Default constructor. Creates an instance of arbitrary precision floating point type and initializes it to zero.

\end{flushleft}
\subsubsection*{Syntax}\texttt{public BigFloat();}
\clearpage

\hypertarget{System.Numerics.Multiprecision.BigFloat.constructor.P.System.Numerics.Multiprecision.BigFloat.C.R.System.Numerics.Multiprecision.BigFloat}{\subsubsection*{BigFloat(const System.Numerics.Multiprecision.BigFloat\&) Member Function}}\begin{flushleft}
Copy constructor.

\end{flushleft}
\subsubsection*{Syntax}
\texttt{public BigFloat(const System.Numerics.Multiprecision.BigFloat\& that);}
\subsubsection*{Parameters}
\begin{flushleft}
\begin{supertabular}[l]{!{\raggedright}p{1.30721cm}!{\raggedright}p{8.55945cm}!{\raggedright}p{4.93333cm}}
\textbf{Name}
& \textbf{Type}
& \textbf{Description}
\\
\hline
that
& \hyperlink{System.Numerics.Multiprecision.BigFloat}{const System.\-Numerics.\-Multiprecision.\-BigFloat\&\-}
& A \hyperlink{System.Numerics.Multiprecision.BigFloat}{BigFloat} to copy from.

\\
\end{supertabular}

\end{flushleft}
\clearpage

\hypertarget{System.Numerics.Multiprecision.BigFloat.operator.assign.P.System.Numerics.Multiprecision.BigFloat.C.R.System.Numerics.Multiprecision.BigFloat}{\subsubsection*{operator=(const System.Numerics.Multiprecision.BigFloat\&) Member Function}}\begin{flushleft}
Copy assignment.

\end{flushleft}
\subsubsection*{Syntax}
\texttt{public void operator=(const System.Numerics.Multiprecision.BigFloat\& that);}
\subsubsection*{Parameters}
\begin{flushleft}
\begin{supertabular}[l]{!{\raggedright}p{1.30721cm}!{\raggedright}p{8.55945cm}!{\raggedright}p{4.93333cm}}
\textbf{Name}
& \textbf{Type}
& \textbf{Description}
\\
\hline
that
& \hyperlink{System.Numerics.Multiprecision.BigFloat}{const System.\-Numerics.\-Multiprecision.\-BigFloat\&\-}
& A \hyperlink{System.Numerics.Multiprecision.BigFloat}{BigFloat} to assign from.

\\
\end{supertabular}

\end{flushleft}
\clearpage

\hypertarget{System.Numerics.Multiprecision.BigFloat.constructor.P.System.Numerics.Multiprecision.BigFloat.RR.System.Numerics.Multiprecision.BigFloat}{\subsubsection*{BigFloat(System.Numerics.Multiprecision.BigFloat\&\&) Member Function}}\begin{flushleft}
Move constructor.

\end{flushleft}
\subsubsection*{Syntax}
\texttt{public BigFloat(System.Numerics.Multiprecision.BigFloat\&\& that);}
\subsubsection*{Parameters}
\begin{flushleft}
\begin{supertabular}[l]{!{\raggedright}p{1.30721cm}!{\raggedright}p{8.55945cm}!{\raggedright}p{4.93333cm}}
\textbf{Name}
& \textbf{Type}
& \textbf{Description}
\\
\hline
that
& \hyperlink{System.Numerics.Multiprecision.BigFloat}{System.\-Numerics.\-Multiprecision.\-BigFloat\&\-\&\-}
& A \hyperlink{System.Numerics.Multiprecision.BigFloat}{BigFloat} to move from.

\\
\end{supertabular}

\end{flushleft}
\clearpage

\hypertarget{System.Numerics.Multiprecision.BigFloat.operator.assign.P.System.Numerics.Multiprecision.BigFloat.RR.System.Numerics.Multiprecision.BigFloat}{\subsubsection*{operator=(System.Numerics.Multiprecision.BigFloat\&\&) Member Function}}\begin{flushleft}
Move assignment.

\end{flushleft}
\subsubsection*{Syntax}
\texttt{public void operator=(System.Numerics.Multiprecision.BigFloat\&\& \_\_parameter0);}
\subsubsection*{Parameters}
\begin{flushleft}
\begin{supertabular}[l]{!{\raggedright}p{2.61335cm}!{\raggedright}p{7.25332cm}!{\raggedright}p{4.93333cm}}
\textbf{Name}
& \textbf{Type}
& \textbf{Description}
\\
\hline
\_\_parameter0
& \hyperlink{System.Numerics.Multiprecision.BigFloat}{System.\-Numerics.\-Multiprecision.\-BigFloat\&\-\&\-}
& A \hyperlink{System.Numerics.Multiprecision.BigFloat}{BigFloat} to assign from.

\\
\end{supertabular}

\end{flushleft}
\clearpage

\hypertarget{System.Numerics.Multiprecision.BigFloat.constructor.P.System.Numerics.Multiprecision.BigFloat.C.R.System.Numerics.Multiprecision.BigInt}{\subsubsection*{BigFloat(const System.Numerics.Multiprecision.BigInt\&) Member Function}}
\begin{flushleft}
Constructor. Creates an instance of arbitrary precision floating point type and initializes it from the given arbitrary precision integer type.

\end{flushleft}
\subsubsection*{Syntax}
\texttt{public BigFloat(const System.Numerics.Multiprecision.BigInt\& that);}
\subsubsection*{Parameters}
\begin{flushleft}
\begin{supertabular}[l]{!{\raggedright}p{1.30721cm}!{\raggedright}p{8.55945cm}!{\raggedright}p{4.93333cm}}
\textbf{Name}
& \textbf{Type}
& \textbf{Description}
\\
\hline
that
& \hyperlink{System.Numerics.Multiprecision.BigInt}{const System.\-Numerics.\-Multiprecision.\-BigInt\&\-}
& An arbitrary precision integer type value to construct from.

\\
\end{supertabular}

\end{flushleft}
\clearpage

\hypertarget{System.Numerics.Multiprecision.BigFloat.constructor.P.System.Numerics.Multiprecision.BigFloat.C.R.System.Numerics.Multiprecision.BigRational}{\subsubsection*{BigFloat(const System.Numerics.Multiprecision.BigRational\&) Member Function}}
\begin{flushleft}
Constructor. Creates an instance of arbitrary precision floating point type and initializes it from the given arbitrary precision rational type.

\end{flushleft}
\subsubsection*{Syntax}
\texttt{public BigFloat(const System.Numerics.Multiprecision.BigRational\& that);}
\subsubsection*{Parameters}
\begin{flushleft}
\begin{supertabular}[l]{!{\raggedright}p{1.30721cm}!{\raggedright}p{8.55945cm}!{\raggedright}p{4.93333cm}}
\textbf{Name}
& \textbf{Type}
& \textbf{Description}
\\
\hline
that
& \hyperlink{System.Numerics.Multiprecision.BigRational}{const System.\-Numerics.\-Multiprecision.\-BigRational\&\-}
& An arbitrary precision rational type value to construct from.

\\
\end{supertabular}

\end{flushleft}
\clearpage

\hypertarget{System.Numerics.Multiprecision.BigFloat.constructor.P.System.Numerics.Multiprecision.BigFloat.C.R.System.Numerics.Multiprecision.Precision}{\subsubsection*{BigFloat(const System.Numerics.Multiprecision.Precision\&) Member Function}}
\begin{flushleft}
Constructor. Constructs an arbitrary precision floating point value with given precision.

\end{flushleft}
\subsubsection*{Syntax}
\texttt{public BigFloat(const System.Numerics.Multiprecision.Precision\& prec);}
\subsubsection*{Parameters}
\begin{flushleft}
\begin{supertabular}[l]{!{\raggedright}p{1.30721cm}!{\raggedright}p{8.55945cm}!{\raggedright}p{2.65538cm}}
\textbf{Name}
& \textbf{Type}
& \textbf{Description}
\\
\hline
prec
& \hyperlink{System.Numerics.Multiprecision.Precision}{const System.\-Numerics.\-Multiprecision.\-Precision\&\-}
& Precision.

\\
\end{supertabular}

\end{flushleft}
\clearpage

\hypertarget{System.Numerics.Multiprecision.BigFloat.constructor.P.System.Numerics.Multiprecision.BigFloat.C.R.System.String.char}{\subsubsection*{BigFloat(const System.String$<$char$>$\&) Member Function}}
\begin{flushleft}
Constructor. Constructs an arbitrary precision floating point value from given decimal digits.

\end{flushleft}
\subsubsection*{Syntax}
\texttt{public BigFloat(const System.String$<$char$>$\& str);}
\subsubsection*{Parameters}
\begin{flushleft}
\begin{supertabular}[l]{lll}
\textbf{Name}
& \textbf{Type}
& \textbf{Description}
\\
\hline
str
& const System.\-String$<$\-char$>$\-\&\-
& Digit string.

\\
\end{supertabular}

\end{flushleft}
\clearpage

\hypertarget{System.Numerics.Multiprecision.BigFloat.constructor.P.System.Numerics.Multiprecision.BigFloat.C.R.System.String.char.int}{\subsubsection*{BigFloat(const System.String$<$char$>$\&, int) Member Function}}
\begin{flushleft}
Constructor. Constructs an arbitrary precision floating point value from given digits that are in given base.

\end{flushleft}
\subsubsection*{Syntax}
\texttt{public BigFloat(const System.String$<$char$>$\& str, int base\_);}
\subsubsection*{Parameters}
\begin{flushleft}
\begin{supertabular}[l]{lll}
\textbf{Name}
& \textbf{Type}
& \textbf{Description}
\\
\hline
str
& const System.\-String$<$\-char$>$\-\&\-
& Digit string.

\\
base\_
& int
& Base of digits.

\\
\end{supertabular}

\end{flushleft}
\clearpage

\hypertarget{System.Numerics.Multiprecision.BigFloat.constructor.P.System.Numerics.Multiprecision.BigFloat.double}{\subsubsection*{BigFloat(double) Member Function}}
\begin{flushleft}
Constructor. Constructs an arbitrary precision floating point value from given double precision value.

\end{flushleft}
\subsubsection*{Syntax}\texttt{public BigFloat(double that);}

\subsubsection*{Parameters}
\begin{flushleft}
\begin{supertabular}[l]{lll}
\textbf{Name}
& \textbf{Type}
& \textbf{Description}
\\
\hline
that
& double
& A double precision value to initialize from.

\\
\end{supertabular}

\end{flushleft}
\clearpage

\hypertarget{System.Numerics.Multiprecision.BigFloat.constructor.P.System.Numerics.Multiprecision.BigFloat.int}{\subsubsection*{BigFloat(int) Member Function}}
\begin{flushleft}
Constructor. Constructs an arbitrary precision floating point value from given integer value.

\end{flushleft}
\subsubsection*{Syntax}\texttt{public BigFloat(int that);}

\subsubsection*{Parameters}
\begin{flushleft}
\begin{supertabular}[l]{lll}
\textbf{Name}
& \textbf{Type}
& \textbf{Description}
\\
\hline
that
& int
& An integer value to initialize from.

\\
\end{supertabular}

\end{flushleft}
\clearpage

\hypertarget{System.Numerics.Multiprecision.BigFloat.constructor.P.System.Numerics.Multiprecision.BigFloat.uint}{\subsubsection*{BigFloat(uint) Member Function}}
\begin{flushleft}
Constructor. Constructs an arbitrary precision floating point value from given unsigned integer value.

\end{flushleft}
\subsubsection*{Syntax}\texttt{public BigFloat(uint that);}

\subsubsection*{Parameters}
\begin{flushleft}
\begin{supertabular}[l]{lll}
\textbf{Name}
& \textbf{Type}
& \textbf{Description}
\\
\hline
that
& uint
& An unsigned integer value to initialize from.

\\
\end{supertabular}

\end{flushleft}
\clearpage

\hypertarget{System.Numerics.Multiprecision.BigFloat.Handle.C.P.System.Numerics.Multiprecision.BigFloat}{\subsubsection*{Handle() const Member Function}}
\begin{flushleft}
Returns a handle to the GNU MP library arbitrary precision floating point number representation.

\end{flushleft}
\subsubsection*{Syntax}\texttt{public void* Handle() const;}

\subsubsection*{Returns}void*
\begin{flushleft}
Returns a handle to the GNU MP library arbitrary precision floating point number representation.

\end{flushleft}
\clearpage

\hypertarget{System.Numerics.Multiprecision.BigFloat.ToDouble.C.P.System.Numerics.Multiprecision.BigFloat}{\subsubsection*{ToDouble() const Member Function}}
\begin{flushleft}
Converts the arbitrary precision floating point number to a double precision value by truncating it.

\end{flushleft}
\subsubsection*{Syntax}\texttt{public double ToDouble() const;}

\subsubsection*{Returns}double
\begin{flushleft}
Returns the arbitrary precision floating point number converted to a double precision value by truncating it.

\end{flushleft}
\clearpage

\hypertarget{System.Numerics.Multiprecision.BigFloat.ToString.C.P.System.Numerics.Multiprecision.BigFloat}{\subsubsection*{ToString() const Member Function}}
\begin{flushleft}
Returns the value of the \hyperlink{System.Numerics.Multiprecision.BigFloat}{BigFloat} as a string.

\end{flushleft}
\subsubsection*{Syntax}
\texttt{public System.String$<$char$>$ ToString() const;}
\subsubsection*{Returns}System.\-String$<$\-char$>$\-
\begin{flushleft}
Returns the value of the \hyperlink{System.Numerics.Multiprecision.BigFloat}{BigFloat} as a string.

\end{flushleft}
\clearpage

\hypertarget{System.Numerics.Multiprecision.BigFloat.ToString.C.P.System.Numerics.Multiprecision.BigFloat.int}{\subsubsection*{ToString(int) const Member Function}}
\begin{flushleft}
Returns the value of the \hyperlink{System.Numerics.Multiprecision.BigFloat}{BigFloat} as a string using given base.

\end{flushleft}
\subsubsection*{Syntax}
\texttt{public System.String$<$char$>$ ToString(int base\_) const;}
\subsubsection*{Parameters}
\begin{flushleft}
\begin{supertabular}[l]{lll}
\textbf{Name}
& \textbf{Type}
& \textbf{Description}
\\
\hline
base\_
& int
& Base.

\\
\end{supertabular}

\end{flushleft}
\subsubsection*{Returns}System.\-String$<$\-char$>$\-
\begin{flushleft}
Returns the value of the \hyperlink{System.Numerics.Multiprecision.BigFloat}{BigFloat} as a string using given base.

\end{flushleft}
\clearpage

\hypertarget{System.Numerics.Multiprecision.BigFloat.ToString.C.P.System.Numerics.Multiprecision.BigFloat.int.uint}{\subsubsection*{ToString(int, uint) const Member Function}}
\begin{flushleft}
Returns the value of the \hyperlink{System.Numerics.Multiprecision.BigFloat}{BigFloat} as a string using given base and given number of digits.

\end{flushleft}
\subsubsection*{Syntax}
\texttt{public System.String$<$char$>$ ToString(int base\_, uint numDigits) const;}
\subsubsection*{Parameters}
\begin{flushleft}
\begin{supertabular}[l]{lll}
\textbf{Name}
& \textbf{Type}
& \textbf{Description}
\\
\hline
base\_
& int
& Base.

\\
numDigits
& uint
& Number of digits.

\\
\end{supertabular}

\end{flushleft}
\subsubsection*{Returns}System.\-String$<$\-char$>$\-
\begin{flushleft}
Returns the value of the \hyperlink{System.Numerics.Multiprecision.BigFloat}{BigFloat} as a string using given base and given number of digits.

\end{flushleft}
\clearpage

\hypertarget{System.Numerics.Multiprecision.BigFloat.operator.assign.P.System.Numerics.Multiprecision.BigFloat.C.R.System.Numerics.Multiprecision.BigInt}{\subsubsection*{operator=(const System.Numerics.Multiprecision.BigInt\&) Member Function}}
\begin{flushleft}
Assigns the value of the \hyperlink{System.Numerics.Multiprecision.BigFloat}{BigFloat} from the given arbitrary precision integer.

\end{flushleft}
\subsubsection*{Syntax}
\texttt{public void operator=(const System.Numerics.Multiprecision.BigInt\& that);}
\subsubsection*{Parameters}
\begin{flushleft}
\begin{supertabular}[l]{!{\raggedright}p{1.30721cm}!{\raggedright}p{8.55945cm}!{\raggedright}p{4.93333cm}}
\textbf{Name}
& \textbf{Type}
& \textbf{Description}
\\
\hline
that
& \hyperlink{System.Numerics.Multiprecision.BigInt}{const System.\-Numerics.\-Multiprecision.\-BigInt\&\-}
& An arbitrary precision integer to assign from.

\\
\end{supertabular}

\end{flushleft}
\clearpage

\hypertarget{System.Numerics.Multiprecision.BigFloat.operator.assign.P.System.Numerics.Multiprecision.BigFloat.C.R.System.Numerics.Multiprecision.BigRational}{\subsubsection*{operator=(const System.Numerics.Multiprecision.BigRational\&) Member Function}}
\begin{flushleft}
Assigns the value of the \hyperlink{System.Numerics.Multiprecision.BigFloat}{BigFloat} from the given arbitrary precision rational.

\end{flushleft}
\subsubsection*{Syntax}
\texttt{public void operator=(const System.Numerics.Multiprecision.BigRational\& that);}
\subsubsection*{Parameters}
\begin{flushleft}
\begin{supertabular}[l]{!{\raggedright}p{1.30721cm}!{\raggedright}p{8.55945cm}!{\raggedright}p{4.93333cm}}
\textbf{Name}
& \textbf{Type}
& \textbf{Description}
\\
\hline
that
& \hyperlink{System.Numerics.Multiprecision.BigRational}{const System.\-Numerics.\-Multiprecision.\-BigRational\&\-}
& An arbitrary precision rational to assign from.

\\
\end{supertabular}

\end{flushleft}
\clearpage

\hypertarget{System.Numerics.Multiprecision.BigFloat.operator.assign.P.System.Numerics.Multiprecision.BigFloat.double}{\subsubsection*{operator=(double) Member Function}}
\begin{flushleft}
Assigns the value of the \hyperlink{System.Numerics.Multiprecision.BigFloat}{BigFloat} from the given double precision value.

\end{flushleft}
\subsubsection*{Syntax}\texttt{public void operator=(double that);}

\subsubsection*{Parameters}
\begin{flushleft}
\begin{supertabular}[l]{lll}
\textbf{Name}
& \textbf{Type}
& \textbf{Description}
\\
\hline
that
& double
& A double precision value to assign from.

\\
\end{supertabular}

\end{flushleft}
\clearpage

\hypertarget{System.Numerics.Multiprecision.BigFloat.operator.assign.P.System.Numerics.Multiprecision.BigFloat.int}{\subsubsection*{operator=(int) Member Function}}
\begin{flushleft}
Assigns the value of the \hyperlink{System.Numerics.Multiprecision.BigFloat}{BigFloat} from the given integer value.

\end{flushleft}
\subsubsection*{Syntax}\texttt{public void operator=(int that);}

\subsubsection*{Parameters}
\begin{flushleft}
\begin{supertabular}[l]{lll}
\textbf{Name}
& \textbf{Type}
& \textbf{Description}
\\
\hline
that
& int
& An integer value to assign from.

\\
\end{supertabular}

\end{flushleft}
\clearpage

\hypertarget{System.Numerics.Multiprecision.BigFloat.operator.assign.P.System.Numerics.Multiprecision.BigFloat.uint}{\subsubsection*{operator=(uint) Member Function}}
\begin{flushleft}
Assigns the value of the \hyperlink{System.Numerics.Multiprecision.BigFloat}{BigFloat} from the given unsigned integer value.

\end{flushleft}
\subsubsection*{Syntax}\texttt{public void operator=(uint that);}

\subsubsection*{Parameters}
\begin{flushleft}
\begin{supertabular}[l]{lll}
\textbf{Name}
& \textbf{Type}
& \textbf{Description}
\\
\hline
that
& uint
& An unsigned integer value to assign from.

\\
\end{supertabular}

\end{flushleft}
\clearpage

\hypertarget{System.Numerics.Multiprecision.BigFloat.destructor.P.System.Numerics.Multiprecision.BigFloat}{\subsubsection*{$\sim$BigFloat() Member Function}}
\begin{flushleft}
Frees memory occupied by the \hyperlink{System.Numerics.Multiprecision.BigFloat}{BigFloat} instance.

\end{flushleft}
\subsubsection*{Syntax}\texttt{public $\sim$BigFloat();}

\subsection{Nonmember Functions}
\begin{flushleft}
\begin{supertabular}[l]{!{\raggedright}p{7.575cm}!{\raggedright}p{7.575cm}}
\textbf{Function}
& \textbf{Description}
\\
\hline
\hyperlink{System.Numerics.Multiprecision.Abs.C.R.System.Numerics.Multiprecision.BigFloat}{Abs(const System.\-Numerics.\-Multiprecision.\-BigFloat\&\-)}
& Returns the absolute value of the given \textbf{BigFloat}
 value.

\\
\hyperlink{System.Numerics.Multiprecision.Ceil.C.R.System.Numerics.Multiprecision.BigFloat}{Ceil(const System.\-Numerics.\-Multiprecision.\-BigFloat\&\-)}
& Returns the \textbf{BigFloat}
 rounded up to the next arbitrary precision integer value.

\\
\hyperlink{System.Numerics.Multiprecision.Floor.C.R.System.Numerics.Multiprecision.BigFloat}{Floor(const System.\-Numerics.\-Multiprecision.\-BigFloat\&\-)}
& Returns the \textbf{BigFloat}
 rounded down to the previous arbitrary precision integer value.

\\
\hyperlink{System.Numerics.Multiprecision.Sqrt.C.R.System.Numerics.Multiprecision.BigFloat}{Sqrt(const System.\-Numerics.\-Multiprecision.\-BigFloat\&\-)}
& Returns the square root of the given \textbf{BigFloat}
 number.

\\
\hyperlink{System.Numerics.Multiprecision.Trunc.C.R.System.Numerics.Multiprecision.BigFloat}{Trunc(const System.\-Numerics.\-Multiprecision.\-BigFloat\&\-)}
& Returns the \textbf{BigFloat}
 truncated towards zero.

\\
\hyperlink{System.Numerics.Multiprecision.operator.times.C.R.System.Numerics.Multiprecision.BigFloat.C.R.System.Numerics.Multiprecision.BigFloat}{operator*(const System.\-Numerics.\-Multiprecision.\-BigFloat\&\-, const System.\-Numerics.\-Multiprecision.\-BigFloat\&\-)}
& Returns the product of given \textbf{BigFloat}
 value multiplied by another.

\\
\hyperlink{System.Numerics.Multiprecision.operator.plus.C.R.System.Numerics.Multiprecision.BigFloat.C.R.System.Numerics.Multiprecision.BigFloat}{operator+(const System.\-Numerics.\-Multiprecision.\-BigFloat\&\-, const System.\-Numerics.\-Multiprecision.\-BigFloat\&\-)}
& Returns the sum of given \textbf{BigFloat}
 value added to another.

\\
\hyperlink{System.Numerics.Multiprecision.operator.minus.C.R.System.Numerics.Multiprecision.BigFloat}{operator-(const System.\-Numerics.\-Multiprecision.\-BigFloat\&\-)}
& Returns the negation of \textbf{BigFloat}
.

\\
\hyperlink{System.Numerics.Multiprecision.operator.minus.C.R.System.Numerics.Multiprecision.BigFloat.C.R.System.Numerics.Multiprecision.BigFloat}{operator-(const System.\-Numerics.\-Multiprecision.\-BigFloat\&\-, const System.\-Numerics.\-Multiprecision.\-BigFloat\&\-)}
& Returns the difference of given \textbf{BigFloat}
 value subtracted from another.

\\
\hyperlink{System.Numerics.Multiprecision.operator.divides.C.R.System.Numerics.Multiprecision.BigFloat.C.R.System.Numerics.Multiprecision.BigFloat}{operator/(const System.\-Numerics.\-Multiprecision.\-BigFloat\&\-, const System.\-Numerics.\-Multiprecision.\-BigFloat\&\-)}
& Returns the quotient when given \textbf{BigFloat}
 is divided by another.

\\
\hyperlink{System.Numerics.Multiprecision.operator.less.C.R.System.Numerics.Multiprecision.BigFloat.C.R.System.Numerics.Multiprecision.BigFloat}{operator$<$\-(const System.\-Numerics.\-Multiprecision.\-BigFloat\&\-, const System.\-Numerics.\-Multiprecision.\-BigFloat\&\-)}
& Returns true if the first \textbf{BigFloat}
 is less than the second \textbf{BigFloat}
, false otherwise.

\\
\hyperlink{System.Numerics.Multiprecision.operator.equal.C.R.System.Numerics.Multiprecision.BigFloat.C.R.System.Numerics.Multiprecision.BigFloat}{operator==(const System.\-Numerics.\-Multiprecision.\-BigFloat\&\-, const System.\-Numerics.\-Multiprecision.\-BigFloat\&\-)}
& Returns true if the first \textbf{BigFloat}
 is equal to the second \textbf{BigFloat}
, false otherwise.

\\
\end{supertabular}

\end{flushleft}
\clearpage

\hypertarget{System.Numerics.Multiprecision.Abs.C.R.System.Numerics.Multiprecision.BigFloat}{\subsubsection*{Abs(const System.Numerics.Multiprecision.BigFloat\&) Function}}
\begin{flushleft}
Returns the absolute value of the given \hyperlink{System.Numerics.Multiprecision.BigFloat}{BigFloat} value.

\end{flushleft}
\subsubsection*{Syntax}
\texttt{public System.Numerics.Multiprecision.BigFloat Abs(const System.Numerics.Multiprecision.BigFloat\& x);}
\subsubsection*{Parameters}
\begin{flushleft}
\begin{supertabular}[l]{!{\raggedright}p{1.30721cm}!{\raggedright}p{8.55945cm}!{\raggedright}p{4.93333cm}}
\textbf{Name}
& \textbf{Type}
& \textbf{Description}
\\
\hline
x
& \hyperlink{System.Numerics.Multiprecision.BigFloat}{const System.\-Numerics.\-Multiprecision.\-BigFloat\&\-}
& Arbitrary precision floating point number.

\\
\end{supertabular}

\end{flushleft}
\subsubsection*{Returns}
\hyperlink{System.Numerics.Multiprecision.BigFloat}{System.\-Numerics.\-Multiprecision.\-BigFloat}
\begin{flushleft}
Returns the absolute value of the given \hyperlink{System.Numerics.Multiprecision.BigFloat}{BigFloat} value.

\end{flushleft}
\clearpage

\hypertarget{System.Numerics.Multiprecision.Ceil.C.R.System.Numerics.Multiprecision.BigFloat}{\subsubsection*{Ceil(const System.Numerics.Multiprecision.BigFloat\&) Function}}
\begin{flushleft}
Returns the \hyperlink{System.Numerics.Multiprecision.BigFloat}{BigFloat} rounded up to the next arbitrary precision integer value.

\end{flushleft}
\subsubsection*{Syntax}
\texttt{public System.Numerics.Multiprecision.BigFloat Ceil(const System.Numerics.Multiprecision.BigFloat\& x);}
\subsubsection*{Parameters}
\begin{flushleft}
\begin{supertabular}[l]{!{\raggedright}p{1.30721cm}!{\raggedright}p{8.55945cm}!{\raggedright}p{2.65538cm}}
\textbf{Name}
& \textbf{Type}
& \textbf{Description}
\\
\hline
x
& \hyperlink{System.Numerics.Multiprecision.BigFloat}{const System.\-Numerics.\-Multiprecision.\-BigFloat\&\-}
& \hyperlink{System.Numerics.Multiprecision.BigFloat}{BigFloat}.

\\
\end{supertabular}

\end{flushleft}
\subsubsection*{Returns}
\hyperlink{System.Numerics.Multiprecision.BigFloat}{System.\-Numerics.\-Multiprecision.\-BigFloat}
\begin{flushleft}
\hyperlink{System.Numerics.Multiprecision.BigFloat}{BigFloat}

\end{flushleft}
\clearpage

\hypertarget{System.Numerics.Multiprecision.Floor.C.R.System.Numerics.Multiprecision.BigFloat}{\subsubsection*{Floor(const System.Numerics.Multiprecision.BigFloat\&) Function}}
\begin{flushleft}
Returns the \hyperlink{System.Numerics.Multiprecision.BigFloat}{BigFloat} rounded down to the previous arbitrary precision integer value.

\end{flushleft}
\subsubsection*{Syntax}
\texttt{public System.Numerics.Multiprecision.BigFloat Floor(const System.Numerics.Multiprecision.BigFloat\& x);}
\subsubsection*{Parameters}
\begin{flushleft}
\begin{supertabular}[l]{!{\raggedright}p{1.30721cm}!{\raggedright}p{8.55945cm}!{\raggedright}p{2.65538cm}}
\textbf{Name}
& \textbf{Type}
& \textbf{Description}
\\
\hline
x
& \hyperlink{System.Numerics.Multiprecision.BigFloat}{const System.\-Numerics.\-Multiprecision.\-BigFloat\&\-}
& \hyperlink{System.Numerics.Multiprecision.BigFloat}{BigFloat}.

\\
\end{supertabular}

\end{flushleft}
\subsubsection*{Returns}
\hyperlink{System.Numerics.Multiprecision.BigFloat}{System.\-Numerics.\-Multiprecision.\-BigFloat}
\begin{flushleft}
Returns the \hyperlink{System.Numerics.Multiprecision.BigFloat}{BigFloat} rounded down to the previous arbitrary precision integer value.

\end{flushleft}
\clearpage

\hypertarget{System.Numerics.Multiprecision.Sqrt.C.R.System.Numerics.Multiprecision.BigFloat}{\subsubsection*{Sqrt(const System.Numerics.Multiprecision.BigFloat\&) Function}}
\begin{flushleft}
Returns the square root of the given \hyperlink{System.Numerics.Multiprecision.BigFloat}{BigFloat} number.

\end{flushleft}
\subsubsection*{Syntax}
\texttt{public System.Numerics.Multiprecision.BigFloat Sqrt(const System.Numerics.Multiprecision.BigFloat\& x);}
\subsubsection*{Parameters}
\begin{flushleft}
\begin{supertabular}[l]{!{\raggedright}p{1.30721cm}!{\raggedright}p{8.55945cm}!{\raggedright}p{2.65538cm}}
\textbf{Name}
& \textbf{Type}
& \textbf{Description}
\\
\hline
x
& \hyperlink{System.Numerics.Multiprecision.BigFloat}{const System.\-Numerics.\-Multiprecision.\-BigFloat\&\-}
& \hyperlink{System.Numerics.Multiprecision.BigFloat}{BigFloat}.

\\
\end{supertabular}

\end{flushleft}
\subsubsection*{Returns}
\hyperlink{System.Numerics.Multiprecision.BigFloat}{System.\-Numerics.\-Multiprecision.\-BigFloat}
\begin{flushleft}
Returns the square root of the given \hyperlink{System.Numerics.Multiprecision.BigFloat}{BigFloat} number.

\end{flushleft}
\clearpage

\hypertarget{System.Numerics.Multiprecision.Trunc.C.R.System.Numerics.Multiprecision.BigFloat}{\subsubsection*{Trunc(const System.Numerics.Multiprecision.BigFloat\&) Function}}
\begin{flushleft}
Returns the \hyperlink{System.Numerics.Multiprecision.BigFloat}{BigFloat} truncated towards zero.

\end{flushleft}
\subsubsection*{Syntax}
\texttt{public System.Numerics.Multiprecision.BigFloat Trunc(const System.Numerics.Multiprecision.BigFloat\& x);}
\subsubsection*{Parameters}
\begin{flushleft}
\begin{supertabular}[l]{!{\raggedright}p{1.30721cm}!{\raggedright}p{8.55945cm}!{\raggedright}p{2.65538cm}}
\textbf{Name}
& \textbf{Type}
& \textbf{Description}
\\
\hline
x
& \hyperlink{System.Numerics.Multiprecision.BigFloat}{const System.\-Numerics.\-Multiprecision.\-BigFloat\&\-}
& \hyperlink{System.Numerics.Multiprecision.BigFloat}{BigFloat}.

\\
\end{supertabular}

\end{flushleft}
\subsubsection*{Returns}
\hyperlink{System.Numerics.Multiprecision.BigFloat}{System.\-Numerics.\-Multiprecision.\-BigFloat}
\begin{flushleft}
Returns the \hyperlink{System.Numerics.Multiprecision.BigFloat}{BigFloat} truncated towards zero.

\end{flushleft}
\clearpage

\hypertarget{System.Numerics.Multiprecision.operator.times.C.R.System.Numerics.Multiprecision.BigFloat.C.R.System.Numerics.Multiprecision.BigFloat}{\subsubsection*{operator*(const System.Numerics.Multiprecision.BigFloat\&, const System.Numerics.Multiprecision.BigFloat\&) Function}}
\begin{flushleft}
Returns the product of given \hyperlink{System.Numerics.Multiprecision.BigFloat}{BigFloat} value multiplied by another.

\end{flushleft}
\subsubsection*{Syntax}
\texttt{public System.Numerics.Multiprecision.BigFloat operator*(const System.Numerics.Multiprecision.BigFloat\& left, const System.Numerics.Multiprecision.BigFloat\& right);}
\subsubsection*{Parameters}
\begin{flushleft}
\begin{supertabular}[l]{!{\raggedright}p{1.30721cm}!{\raggedright}p{8.55945cm}!{\raggedright}p{3.66839cm}}
\textbf{Name}
& \textbf{Type}
& \textbf{Description}
\\
\hline
left
& \hyperlink{System.Numerics.Multiprecision.BigFloat}{const System.\-Numerics.\-Multiprecision.\-BigFloat\&\-}
& Left operand.

\\
right
& \hyperlink{System.Numerics.Multiprecision.BigFloat}{const System.\-Numerics.\-Multiprecision.\-BigFloat\&\-}
& Right operand.

\\
\end{supertabular}

\end{flushleft}
\subsubsection*{Returns}
\hyperlink{System.Numerics.Multiprecision.BigFloat}{System.\-Numerics.\-Multiprecision.\-BigFloat}
\begin{flushleft}
Returns the product of given \hyperlink{System.Numerics.Multiprecision.BigFloat}{BigFloat} value multiplied by another.

\end{flushleft}
\clearpage

\hypertarget{System.Numerics.Multiprecision.operator.plus.C.R.System.Numerics.Multiprecision.BigFloat.C.R.System.Numerics.Multiprecision.BigFloat}{\subsubsection*{operator+(const System.Numerics.Multiprecision.BigFloat\&, const System.Numerics.Multiprecision.BigFloat\&) Function}}
\begin{flushleft}
Returns the sum of given \hyperlink{System.Numerics.Multiprecision.BigFloat}{BigFloat} value added to another.

\end{flushleft}
\subsubsection*{Syntax}
\texttt{public System.Numerics.Multiprecision.BigFloat operator+(const System.Numerics.Multiprecision.BigFloat\& left, const System.Numerics.Multiprecision.BigFloat\& right);}
\subsubsection*{Parameters}
\begin{flushleft}
\begin{supertabular}[l]{!{\raggedright}p{1.30721cm}!{\raggedright}p{8.55945cm}!{\raggedright}p{3.66839cm}}
\textbf{Name}
& \textbf{Type}
& \textbf{Description}
\\
\hline
left
& \hyperlink{System.Numerics.Multiprecision.BigFloat}{const System.\-Numerics.\-Multiprecision.\-BigFloat\&\-}
& Left operand.

\\
right
& \hyperlink{System.Numerics.Multiprecision.BigFloat}{const System.\-Numerics.\-Multiprecision.\-BigFloat\&\-}
& Right operand.

\\
\end{supertabular}

\end{flushleft}
\subsubsection*{Returns}
\hyperlink{System.Numerics.Multiprecision.BigFloat}{System.\-Numerics.\-Multiprecision.\-BigFloat}
\begin{flushleft}
Returns the sum of given \hyperlink{System.Numerics.Multiprecision.BigFloat}{BigFloat} value added to another.

\end{flushleft}
\clearpage

\hypertarget{System.Numerics.Multiprecision.operator.minus.C.R.System.Numerics.Multiprecision.BigFloat}{\subsubsection*{operator-(const System.Numerics.Multiprecision.BigFloat\&) Function}}
\begin{flushleft}
Returns the negation of \hyperlink{System.Numerics.Multiprecision.BigFloat}{BigFloat}.

\end{flushleft}
\subsubsection*{Syntax}
\texttt{public System.Numerics.Multiprecision.BigFloat operator-(const System.Numerics.Multiprecision.BigFloat\& x);}
\subsubsection*{Parameters}
\begin{flushleft}
\begin{supertabular}[l]{!{\raggedright}p{1.30721cm}!{\raggedright}p{8.55945cm}!{\raggedright}p{2.65538cm}}
\textbf{Name}
& \textbf{Type}
& \textbf{Description}
\\
\hline
x
& \hyperlink{System.Numerics.Multiprecision.BigFloat}{const System.\-Numerics.\-Multiprecision.\-BigFloat\&\-}
& A value.

\\
\end{supertabular}

\end{flushleft}
\subsubsection*{Returns}
\hyperlink{System.Numerics.Multiprecision.BigFloat}{System.\-Numerics.\-Multiprecision.\-BigFloat}
\begin{flushleft}
Returns the negation of \hyperlink{System.Numerics.Multiprecision.BigFloat}{BigFloat}.

\end{flushleft}
\clearpage

\hypertarget{System.Numerics.Multiprecision.operator.minus.C.R.System.Numerics.Multiprecision.BigFloat.C.R.System.Numerics.Multiprecision.BigFloat}{\subsubsection*{operator-(const System.Numerics.Multiprecision.BigFloat\&, const System.Numerics.Multiprecision.BigFloat\&) Function}}
\begin{flushleft}
Returns the difference of given \hyperlink{System.Numerics.Multiprecision.BigFloat}{BigFloat} value subtracted from another.

\end{flushleft}
\subsubsection*{Syntax}
\texttt{public System.Numerics.Multiprecision.BigFloat operator-(const System.Numerics.Multiprecision.BigFloat\& left, const System.Numerics.Multiprecision.BigFloat\& right);}
\subsubsection*{Parameters}
\begin{flushleft}
\begin{supertabular}[l]{!{\raggedright}p{1.30721cm}!{\raggedright}p{8.55945cm}!{\raggedright}p{3.66839cm}}
\textbf{Name}
& \textbf{Type}
& \textbf{Description}
\\
\hline
left
& \hyperlink{System.Numerics.Multiprecision.BigFloat}{const System.\-Numerics.\-Multiprecision.\-BigFloat\&\-}
& Left operand.

\\
right
& \hyperlink{System.Numerics.Multiprecision.BigFloat}{const System.\-Numerics.\-Multiprecision.\-BigFloat\&\-}
& Right operand.

\\
\end{supertabular}

\end{flushleft}
\subsubsection*{Returns}
\hyperlink{System.Numerics.Multiprecision.BigFloat}{System.\-Numerics.\-Multiprecision.\-BigFloat}
\begin{flushleft}
Returns the difference of given \hyperlink{System.Numerics.Multiprecision.BigFloat}{BigFloat} value subtracted from another.

\end{flushleft}
\clearpage

\hypertarget{System.Numerics.Multiprecision.operator.divides.C.R.System.Numerics.Multiprecision.BigFloat.C.R.System.Numerics.Multiprecision.BigFloat}{\subsubsection*{operator/(const System.Numerics.Multiprecision.BigFloat\&, const System.Numerics.Multiprecision.BigFloat\&) Function}}
\begin{flushleft}
Returns the quotient when given \hyperlink{System.Numerics.Multiprecision.BigFloat}{BigFloat} is divided by another.

\end{flushleft}
\subsubsection*{Syntax}
\texttt{public System.Numerics.Multiprecision.BigFloat operator/(const System.Numerics.Multiprecision.BigFloat\& left, const System.Numerics.Multiprecision.BigFloat\& right);}
\subsubsection*{Parameters}
\begin{flushleft}
\begin{supertabular}[l]{!{\raggedright}p{1.30721cm}!{\raggedright}p{8.55945cm}!{\raggedright}p{3.66839cm}}
\textbf{Name}
& \textbf{Type}
& \textbf{Description}
\\
\hline
left
& \hyperlink{System.Numerics.Multiprecision.BigFloat}{const System.\-Numerics.\-Multiprecision.\-BigFloat\&\-}
& Left operand.

\\
right
& \hyperlink{System.Numerics.Multiprecision.BigFloat}{const System.\-Numerics.\-Multiprecision.\-BigFloat\&\-}
& Right operand.

\\
\end{supertabular}

\end{flushleft}
\subsubsection*{Returns}
\hyperlink{System.Numerics.Multiprecision.BigFloat}{System.\-Numerics.\-Multiprecision.\-BigFloat}
\begin{flushleft}
Returns the quotient when given \hyperlink{System.Numerics.Multiprecision.BigFloat}{BigFloat} is divided by another.

\end{flushleft}
\clearpage

\hypertarget{System.Numerics.Multiprecision.operator.less.C.R.System.Numerics.Multiprecision.BigFloat.C.R.System.Numerics.Multiprecision.BigFloat}{\subsubsection*{operator$<$(const System.Numerics.Multiprecision.BigFloat\&, const System.Numerics.Multiprecision.BigFloat\&) Function}}
\begin{flushleft}
Returns true if the first \hyperlink{System.Numerics.Multiprecision.BigFloat}{BigFloat} is less than the second \hyperlink{System.Numerics.Multiprecision.BigFloat}{BigFloat}, false otherwise.

\end{flushleft}
\subsubsection*{Syntax}
\texttt{public bool operator$<$(const System.Numerics.Multiprecision.BigFloat\& left, const System.Numerics.Multiprecision.BigFloat\& right);}
\subsubsection*{Parameters}
\begin{flushleft}
\begin{supertabular}[l]{!{\raggedright}p{1.30721cm}!{\raggedright}p{8.55945cm}!{\raggedright}p{3.66839cm}}
\textbf{Name}
& \textbf{Type}
& \textbf{Description}
\\
\hline
left
& \hyperlink{System.Numerics.Multiprecision.BigFloat}{const System.\-Numerics.\-Multiprecision.\-BigFloat\&\-}
& Left operand.

\\
right
& \hyperlink{System.Numerics.Multiprecision.BigFloat}{const System.\-Numerics.\-Multiprecision.\-BigFloat\&\-}
& Right operand.

\\
\end{supertabular}

\end{flushleft}
\subsubsection*{Returns}bool
\begin{flushleft}
Returns true if the first \hyperlink{System.Numerics.Multiprecision.BigFloat}{BigFloat} is less than the second \hyperlink{System.Numerics.Multiprecision.BigFloat}{BigFloat}, false otherwise.

\end{flushleft}
\clearpage

\hypertarget{System.Numerics.Multiprecision.operator.equal.C.R.System.Numerics.Multiprecision.BigFloat.C.R.System.Numerics.Multiprecision.BigFloat}{\subsubsection*{operator==(const System.Numerics.Multiprecision.BigFloat\&, const System.Numerics.Multiprecision.BigFloat\&) Function}}
\begin{flushleft}
Returns true if the first \hyperlink{System.Numerics.Multiprecision.BigFloat}{BigFloat} is equal to the second \hyperlink{System.Numerics.Multiprecision.BigFloat}{BigFloat}, false otherwise.

\end{flushleft}
\subsubsection*{Syntax}
\texttt{public bool operator==(const System.Numerics.Multiprecision.BigFloat\& left, const System.Numerics.Multiprecision.BigFloat\& right);}
\subsubsection*{Parameters}
\begin{flushleft}
\begin{supertabular}[l]{!{\raggedright}p{1.30721cm}!{\raggedright}p{8.55945cm}!{\raggedright}p{3.66839cm}}
\textbf{Name}
& \textbf{Type}
& \textbf{Description}
\\
\hline
left
& \hyperlink{System.Numerics.Multiprecision.BigFloat}{const System.\-Numerics.\-Multiprecision.\-BigFloat\&\-}
& Left operand.

\\
right
& \hyperlink{System.Numerics.Multiprecision.BigFloat}{const System.\-Numerics.\-Multiprecision.\-BigFloat\&\-}
& Right operand.

\\
\end{supertabular}

\end{flushleft}
\subsubsection*{Returns}bool
\begin{flushleft}
Returns true if the first \hyperlink{System.Numerics.Multiprecision.BigFloat}{BigFloat} is equal to the second \hyperlink{System.Numerics.Multiprecision.BigFloat}{BigFloat}, false otherwise.

\end{flushleft}
\clearpage
\clearpage

\hypertarget{System.Numerics.Multiprecision.BigInt}{\section{BigInt Class}}\begin{flushleft}
An arbitrary precision signed integer type.

\end{flushleft}

\subsection*{Syntax}\texttt{public class BigInt;}
\subsection{Member Functions}
\begin{flushleft}
\begin{supertabular}[l]{!{\raggedright}p{7.575cm}!{\raggedright}p{7.575cm}}
\textbf{Member Function}
& \textbf{Description}
\\
\hline
\hyperlink{System.Numerics.Multiprecision.BigInt.constructor.P.System.Numerics.Multiprecision.BigInt}{BigInt()}
& Default constructor. Creates an instance of arbitrary precision signed integer and initializes it to zero.

\\
\hyperlink{System.Numerics.Multiprecision.BigInt.constructor.P.System.Numerics.Multiprecision.BigInt.C.R.System.Numerics.Multiprecision.BigInt}{BigInt(const System.\-Numerics.\-Multiprecision.\-BigInt\&\-)}
& Copy constructor.

\\
\hyperlink{System.Numerics.Multiprecision.BigInt.operator.assign.P.System.Numerics.Multiprecision.BigInt.C.R.System.Numerics.Multiprecision.BigInt}{operator=(const System.\-Numerics.\-Multiprecision.\-BigInt\&\-)}
& Copy assignment.

\\
\hyperlink{System.Numerics.Multiprecision.BigInt.constructor.P.System.Numerics.Multiprecision.BigInt.RR.System.Numerics.Multiprecision.BigInt}{BigInt(System.\-Numerics.\-Multiprecision.\-BigInt\&\-\&\-)}
& Move constructor.

\\
\hyperlink{System.Numerics.Multiprecision.BigInt.operator.assign.P.System.Numerics.Multiprecision.BigInt.RR.System.Numerics.Multiprecision.BigInt}{operator=(System.\-Numerics.\-Multiprecision.\-BigInt\&\-\&\-)}
& Move assignment.

\\
\hyperlink{System.Numerics.Multiprecision.BigInt.constructor.P.System.Numerics.Multiprecision.BigInt.C.R.System.String.char}{BigInt(const System.\-String$<$\-char$>$\-\&\-)}
& Constructor. Constructs an arbitrary precision floating point value from given decimal digits.

\\
\hyperlink{System.Numerics.Multiprecision.BigInt.constructor.P.System.Numerics.Multiprecision.BigInt.C.R.System.String.char.int}{BigInt(const System.\-String$<$\-char$>$\-\&\-, int)}
& Constructor. Constructs an arbitrary precision floating point value from given digits of given base.

\\
\hyperlink{System.Numerics.Multiprecision.BigInt.constructor.P.System.Numerics.Multiprecision.BigInt.int}{BigInt(int)}
& Creates an instance of arbitrary precision signed integer and initializes it to given signed integer value.

\\
\hyperlink{System.Numerics.Multiprecision.BigInt.constructor.P.System.Numerics.Multiprecision.BigInt.uint}{BigInt(uint)}
& Creates an instance of arbitrary precision signed integer and initializes it to given unsigned integer value.

\\
\hyperlink{System.Numerics.Multiprecision.BigInt.Handle.C.P.System.Numerics.Multiprecision.BigInt}{Handle() const}
& Returns a handle to the GNU MP library arbitrary precision integer representation.

\\
\hyperlink{System.Numerics.Multiprecision.BigInt.ToString.C.P.System.Numerics.Multiprecision.BigInt}{ToString() const}
& Returns the value of the \textbf{BigInt}
 as a string of decimal digits prefixed by minus sign if the value is negative.

\\
\hyperlink{System.Numerics.Multiprecision.BigInt.ToString.C.P.System.Numerics.Multiprecision.BigInt.int}{ToString(int) const}
& Returns the value of the \textbf{BigInt}
 as a string of digits in given base prefixed by minus sign if the value is negative.

\\
\hyperlink{System.Numerics.Multiprecision.BigInt.operator.assign.P.System.Numerics.Multiprecision.BigInt.int}{operator=(int)}
& Assigns the value of the \textbf{BigInt}
 to given signed integer value.

\\
\hyperlink{System.Numerics.Multiprecision.BigInt.operator.assign.P.System.Numerics.Multiprecision.BigInt.uint}{operator=(uint)}
& Assigns the value of the \textbf{BigInt}
 to given unsigned integer value.

\\
\hyperlink{System.Numerics.Multiprecision.BigInt.destructor.P.System.Numerics.Multiprecision.BigInt}{$\sim$BigInt()}
& Frees memory occupied by the \textbf{BigInt}
 instance.

\\
\end{supertabular}

\end{flushleft}
\clearpage

\hypertarget{System.Numerics.Multiprecision.BigInt.constructor.P.System.Numerics.Multiprecision.BigInt}{\subsubsection*{BigInt() Member Function}}
\begin{flushleft}
Default constructor. Creates an instance of arbitrary precision signed integer and initializes it to zero.

\end{flushleft}
\subsubsection*{Syntax}\texttt{public BigInt();}
\clearpage

\hypertarget{System.Numerics.Multiprecision.BigInt.constructor.P.System.Numerics.Multiprecision.BigInt.C.R.System.Numerics.Multiprecision.BigInt}{\subsubsection*{BigInt(const System.Numerics.Multiprecision.BigInt\&) Member Function}}\begin{flushleft}
Copy constructor.

\end{flushleft}
\subsubsection*{Syntax}
\texttt{public BigInt(const System.Numerics.Multiprecision.BigInt\& that);}
\subsubsection*{Parameters}
\begin{flushleft}
\begin{supertabular}[l]{!{\raggedright}p{1.30721cm}!{\raggedright}p{8.55945cm}!{\raggedright}p{4.93333cm}}
\textbf{Name}
& \textbf{Type}
& \textbf{Description}
\\
\hline
that
& \hyperlink{System.Numerics.Multiprecision.BigInt}{const System.\-Numerics.\-Multiprecision.\-BigInt\&\-}
& A \hyperlink{System.Numerics.Multiprecision.BigInt}{BigInt} to copy from.

\\
\end{supertabular}

\end{flushleft}
\clearpage

\hypertarget{System.Numerics.Multiprecision.BigInt.operator.assign.P.System.Numerics.Multiprecision.BigInt.C.R.System.Numerics.Multiprecision.BigInt}{\subsubsection*{operator=(const System.Numerics.Multiprecision.BigInt\&) Member Function}}\begin{flushleft}
Copy assignment.

\end{flushleft}
\subsubsection*{Syntax}
\texttt{public void operator=(const System.Numerics.Multiprecision.BigInt\& that);}
\subsubsection*{Parameters}
\begin{flushleft}
\begin{supertabular}[l]{!{\raggedright}p{1.30721cm}!{\raggedright}p{8.55945cm}!{\raggedright}p{4.6205cm}}
\textbf{Name}
& \textbf{Type}
& \textbf{Description}
\\
\hline
that
& \hyperlink{System.Numerics.Multiprecision.BigInt}{const System.\-Numerics.\-Multiprecision.\-BigInt\&\-}
& A \hyperlink{System.Numerics.Multiprecision.BigInt}{BigInt} to assign.

\\
\end{supertabular}

\end{flushleft}
\clearpage

\hypertarget{System.Numerics.Multiprecision.BigInt.constructor.P.System.Numerics.Multiprecision.BigInt.RR.System.Numerics.Multiprecision.BigInt}{\subsubsection*{BigInt(System.Numerics.Multiprecision.BigInt\&\&) Member Function}}\begin{flushleft}
Move constructor.

\end{flushleft}
\subsubsection*{Syntax}
\texttt{public BigInt(System.Numerics.Multiprecision.BigInt\&\& that);}
\subsubsection*{Parameters}
\begin{flushleft}
\begin{supertabular}[l]{!{\raggedright}p{1.30721cm}!{\raggedright}p{8.55945cm}!{\raggedright}p{4.93333cm}}
\textbf{Name}
& \textbf{Type}
& \textbf{Description}
\\
\hline
that
& \hyperlink{System.Numerics.Multiprecision.BigInt}{System.\-Numerics.\-Multiprecision.\-BigInt\&\-\&\-}
& A \hyperlink{System.Numerics.Multiprecision.BigInt}{BigInt} to move from.

\\
\end{supertabular}

\end{flushleft}
\clearpage

\hypertarget{System.Numerics.Multiprecision.BigInt.operator.assign.P.System.Numerics.Multiprecision.BigInt.RR.System.Numerics.Multiprecision.BigInt}{\subsubsection*{operator=(System.Numerics.Multiprecision.BigInt\&\&) Member Function}}\begin{flushleft}
Move assignment.

\end{flushleft}
\subsubsection*{Syntax}
\texttt{public void operator=(System.Numerics.Multiprecision.BigInt\&\& that);}
\subsubsection*{Parameters}
\begin{flushleft}
\begin{supertabular}[l]{!{\raggedright}p{1.30721cm}!{\raggedright}p{8.55945cm}!{\raggedright}p{4.6205cm}}
\textbf{Name}
& \textbf{Type}
& \textbf{Description}
\\
\hline
that
& \hyperlink{System.Numerics.Multiprecision.BigInt}{System.\-Numerics.\-Multiprecision.\-BigInt\&\-\&\-}
& A \hyperlink{System.Numerics.Multiprecision.BigInt}{BigInt} to assign.

\\
\end{supertabular}

\end{flushleft}
\clearpage

\hypertarget{System.Numerics.Multiprecision.BigInt.constructor.P.System.Numerics.Multiprecision.BigInt.C.R.System.String.char}{\subsubsection*{BigInt(const System.String$<$char$>$\&) Member Function}}
\begin{flushleft}
Constructor. Constructs an arbitrary precision floating point value from given decimal digits.

\end{flushleft}
\subsubsection*{Syntax}
\texttt{public BigInt(const System.String$<$char$>$\& str);}
\subsubsection*{Parameters}
\begin{flushleft}
\begin{supertabular}[l]{lll}
\textbf{Name}
& \textbf{Type}
& \textbf{Description}
\\
\hline
str
& const System.\-String$<$\-char$>$\-\&\-
& Decimal digit string.

\\
\end{supertabular}

\end{flushleft}
\clearpage

\hypertarget{System.Numerics.Multiprecision.BigInt.constructor.P.System.Numerics.Multiprecision.BigInt.C.R.System.String.char.int}{\subsubsection*{BigInt(const System.String$<$char$>$\&, int) Member Function}}
\begin{flushleft}
Constructor. Constructs an arbitrary precision floating point value from given digits of given base.

\end{flushleft}
\subsubsection*{Syntax}
\texttt{public BigInt(const System.String$<$char$>$\& str, int base\_);}
\subsubsection*{Parameters}
\begin{flushleft}
\begin{supertabular}[l]{lll}
\textbf{Name}
& \textbf{Type}
& \textbf{Description}
\\
\hline
str
& const System.\-String$<$\-char$>$\-\&\-
& Digit string.

\\
base\_
& int
& Base of digits.

\\
\end{supertabular}

\end{flushleft}
\clearpage

\hypertarget{System.Numerics.Multiprecision.BigInt.constructor.P.System.Numerics.Multiprecision.BigInt.int}{\subsubsection*{BigInt(int) Member Function}}
\begin{flushleft}
Creates an instance of arbitrary precision signed integer and initializes it to given signed integer value.

\end{flushleft}
\subsubsection*{Syntax}\texttt{public BigInt(int that);}

\subsubsection*{Parameters}
\begin{flushleft}
\begin{supertabular}[l]{lll}
\textbf{Name}
& \textbf{Type}
& \textbf{Description}
\\
\hline
that
& int
& A signed integer value.

\\
\end{supertabular}

\end{flushleft}
\clearpage

\hypertarget{System.Numerics.Multiprecision.BigInt.constructor.P.System.Numerics.Multiprecision.BigInt.uint}{\subsubsection*{BigInt(uint) Member Function}}
\begin{flushleft}
Creates an instance of arbitrary precision signed integer and initializes it to given unsigned integer value.

\end{flushleft}
\subsubsection*{Syntax}\texttt{public BigInt(uint that);}

\subsubsection*{Parameters}
\begin{flushleft}
\begin{supertabular}[l]{lll}
\textbf{Name}
& \textbf{Type}
& \textbf{Description}
\\
\hline
that
& uint
& An unsigned integer value.

\\
\end{supertabular}

\end{flushleft}
\clearpage

\hypertarget{System.Numerics.Multiprecision.BigInt.Handle.C.P.System.Numerics.Multiprecision.BigInt}{\subsubsection*{Handle() const Member Function}}
\begin{flushleft}
Returns a handle to the GNU MP library arbitrary precision integer representation.

\end{flushleft}
\subsubsection*{Syntax}\texttt{public void* Handle() const;}

\subsubsection*{Returns}void*
\begin{flushleft}
Returns a handle to the GNU MP library arbitrary precision integer representation.

\end{flushleft}
\clearpage

\hypertarget{System.Numerics.Multiprecision.BigInt.ToString.C.P.System.Numerics.Multiprecision.BigInt}{\subsubsection*{ToString() const Member Function}}
\begin{flushleft}
Returns the value of the \hyperlink{System.Numerics.Multiprecision.BigInt}{BigInt} as a string of decimal digits prefixed by minus sign if the value is negative.

\end{flushleft}
\subsubsection*{Syntax}
\texttt{public System.String$<$char$>$ ToString() const;}
\subsubsection*{Returns}System.\-String$<$\-char$>$\-
\begin{flushleft}
Returns the value of the \hyperlink{System.Numerics.Multiprecision.BigInt}{BigInt} as a string of decimal digits prefixed by minus sign if the value is negative.

\end{flushleft}
\clearpage

\hypertarget{System.Numerics.Multiprecision.BigInt.ToString.C.P.System.Numerics.Multiprecision.BigInt.int}{\subsubsection*{ToString(int) const Member Function}}
\begin{flushleft}
Returns the value of the \hyperlink{System.Numerics.Multiprecision.BigInt}{BigInt} as a string of digits in given base prefixed by minus sign if the value is negative.

\end{flushleft}
\subsubsection*{Syntax}
\texttt{public System.String$<$char$>$ ToString(int base\_) const;}
\subsubsection*{Parameters}
\begin{flushleft}
\begin{supertabular}[l]{lll}
\textbf{Name}
& \textbf{Type}
& \textbf{Description}
\\
\hline
base\_
& int
& Base of digits. Base may vary from 2 to 62.

\\
\end{supertabular}

\end{flushleft}
\subsubsection*{Returns}System.\-String$<$\-char$>$\-
\begin{flushleft}
Returns the value of the \hyperlink{System.Numerics.Multiprecision.BigInt}{BigInt} as a string of digits in given base prefixed by minus sign if the value is negative.

\end{flushleft}
\clearpage

\hypertarget{System.Numerics.Multiprecision.BigInt.operator.assign.P.System.Numerics.Multiprecision.BigInt.int}{\subsubsection*{operator=(int) Member Function}}
\begin{flushleft}
Assigns the value of the \hyperlink{System.Numerics.Multiprecision.BigInt}{BigInt} to given signed integer value.

\end{flushleft}
\subsubsection*{Syntax}\texttt{public void operator=(int that);}

\subsubsection*{Parameters}
\begin{flushleft}
\begin{supertabular}[l]{lll}
\textbf{Name}
& \textbf{Type}
& \textbf{Description}
\\
\hline
that
& int
& A signed integer value.

\\
\end{supertabular}

\end{flushleft}
\clearpage

\hypertarget{System.Numerics.Multiprecision.BigInt.operator.assign.P.System.Numerics.Multiprecision.BigInt.uint}{\subsubsection*{operator=(uint) Member Function}}
\begin{flushleft}
Assigns the value of the \hyperlink{System.Numerics.Multiprecision.BigInt}{BigInt} to given unsigned integer value.

\end{flushleft}
\subsubsection*{Syntax}\texttt{public void operator=(uint that);}

\subsubsection*{Parameters}
\begin{flushleft}
\begin{supertabular}[l]{lll}
\textbf{Name}
& \textbf{Type}
& \textbf{Description}
\\
\hline
that
& uint
& An unsigned integer value.

\\
\end{supertabular}

\end{flushleft}
\clearpage

\hypertarget{System.Numerics.Multiprecision.BigInt.destructor.P.System.Numerics.Multiprecision.BigInt}{\subsubsection*{$\sim$BigInt() Member Function}}
\begin{flushleft}
Frees memory occupied by the \hyperlink{System.Numerics.Multiprecision.BigInt}{BigInt} instance.

\end{flushleft}
\subsubsection*{Syntax}\texttt{public $\sim$BigInt();}

\subsection{Nonmember Functions}
\begin{flushleft}
\begin{supertabular}[l]{!{\raggedright}p{7.575cm}!{\raggedright}p{7.575cm}}
\textbf{Function}
& \textbf{Description}
\\
\hline
\hyperlink{System.Numerics.Multiprecision.Abs.C.R.System.Numerics.Multiprecision.BigInt}{Abs(const System.\-Numerics.\-Multiprecision.\-BigInt\&\-)}
& Returns absolute value of given \textbf{BigInt}
.

\\
\hyperlink{System.Numerics.Multiprecision.ClearBit.R.System.Numerics.Multiprecision.BigInt.uint}{ClearBit(System.\-Numerics.\-Multiprecision.\-BigInt\&\-, uint)}
& Clear given bit of given \textbf{BigInt}
.

\\
\hyperlink{System.Numerics.Multiprecision.SetBit.R.System.Numerics.Multiprecision.BigInt.uint}{SetBit(System.\-Numerics.\-Multiprecision.\-BigInt\&\-, uint)}
& Set given bit of given \textbf{BigInt}
.

\\
\hyperlink{System.Numerics.Multiprecision.TestBit.R.System.Numerics.Multiprecision.BigInt.uint}{TestBit(System.\-Numerics.\-Multiprecision.\-BigInt\&\-, uint)}
& Returns true if given bit of given \textbf{BigInt}
 is set, false otherwise.

\\
\hyperlink{System.Numerics.Multiprecision.ToggleBit.R.System.Numerics.Multiprecision.BigInt.uint}{ToggleBit(System.\-Numerics.\-Multiprecision.\-BigInt\&\-, uint)}
& Toggle given bit of given \textbf{BigInt}
.

\\
\hyperlink{System.Numerics.Multiprecision.operator.remainder.C.R.System.Numerics.Multiprecision.BigInt.C.R.System.Numerics.Multiprecision.BigInt}{operator\%(const System.\-Numerics.\-Multiprecision.\-BigInt\&\-, const System.\-Numerics.\-Multiprecision.\-BigInt\&\-)}
& Returns the remainder when given \textbf{BigInt}
 is divided by another.

\\
\hyperlink{System.Numerics.Multiprecision.operator.and.C.R.System.Numerics.Multiprecision.BigInt.C.R.System.Numerics.Multiprecision.BigInt}{operator\&\-(const System.\-Numerics.\-Multiprecision.\-BigInt\&\-, const System.\-Numerics.\-Multiprecision.\-BigInt\&\-)}
& Returns bitwise AND of two \textbf{BigInt}
 values.

\\
\hyperlink{System.Numerics.Multiprecision.operator.times.C.R.System.Numerics.Multiprecision.BigInt.C.R.System.Numerics.Multiprecision.BigInt}{operator*(const System.\-Numerics.\-Multiprecision.\-BigInt\&\-, const System.\-Numerics.\-Multiprecision.\-BigInt\&\-)}
& Returns the product of given \textbf{BigInt}
 value multiplied by another.

\\
\hyperlink{System.Numerics.Multiprecision.operator.plus.C.R.System.Numerics.Multiprecision.BigInt.C.R.System.Numerics.Multiprecision.BigInt}{operator+(const System.\-Numerics.\-Multiprecision.\-BigInt\&\-, const System.\-Numerics.\-Multiprecision.\-BigInt\&\-)}
& Returns the sum of given \textbf{BigInt}
 value added to another.

\\
\hyperlink{System.Numerics.Multiprecision.operator.minus.C.R.System.Numerics.Multiprecision.BigInt}{operator-(const System.\-Numerics.\-Multiprecision.\-BigInt\&\-)}
& Returns the negation of \textbf{BigInt}
.

\\
\hyperlink{System.Numerics.Multiprecision.operator.minus.C.R.System.Numerics.Multiprecision.BigInt.C.R.System.Numerics.Multiprecision.BigInt}{operator-(const System.\-Numerics.\-Multiprecision.\-BigInt\&\-, const System.\-Numerics.\-Multiprecision.\-BigInt\&\-)}
& Returns the difference of given \textbf{BigInt}
 value subtracted from another.

\\
\hyperlink{System.Numerics.Multiprecision.operator.divides.C.R.System.Numerics.Multiprecision.BigInt.C.R.System.Numerics.Multiprecision.BigInt}{operator/(const System.\-Numerics.\-Multiprecision.\-BigInt\&\-, const System.\-Numerics.\-Multiprecision.\-BigInt\&\-)}
& Returns the quotient when given \textbf{BigInt}
 is divided by another.

\\
\hyperlink{System.Numerics.Multiprecision.operator.less.C.R.System.Numerics.Multiprecision.BigInt.C.R.System.Numerics.Multiprecision.BigInt}{operator$<$\-(const System.\-Numerics.\-Multiprecision.\-BigInt\&\-, const System.\-Numerics.\-Multiprecision.\-BigInt\&\-)}
& Returns true if the first \textbf{BigInt}
 is less than the second \textbf{BigInt}
, false otherwise.

\\
\hyperlink{System.Numerics.Multiprecision.operator.shiftLeft.R.System.IO.OutputStream.C.R.System.Numerics.Multiprecision.BigInt}{operator$<$\-$<$\-(System.\-IO.\-OutputStream\&\-, const System.\-Numerics.\-Multiprecision.\-BigInt\&\-)}
& Puts the value of the given \textbf{BigInt}
 to the given output stream as string of decimal digits prefixed by minus sign if the value is negative.

\\
\hyperlink{System.Numerics.Multiprecision.operator.equal.C.R.System.Numerics.Multiprecision.BigInt.C.R.System.Numerics.Multiprecision.BigInt}{operator==(const System.\-Numerics.\-Multiprecision.\-BigInt\&\-, const System.\-Numerics.\-Multiprecision.\-BigInt\&\-)}
& Returns true if the first \textbf{BigInt}
 is equal to the second \textbf{BigInt}
, false otherwise.

\\
\hyperlink{System.Numerics.Multiprecision.operator.xor.C.R.System.Numerics.Multiprecision.BigInt.C.R.System.Numerics.Multiprecision.BigInt}{operator\^{}(const System.\-Numerics.\-Multiprecision.\-BigInt\&\-, const System.\-Numerics.\-Multiprecision.\-BigInt\&\-)}
& Returns bitwise XOR of two \textbf{BigInt}
 values.

\\
\hyperlink{System.Numerics.Multiprecision.operator.or.C.R.System.Numerics.Multiprecision.BigInt.C.R.System.Numerics.Multiprecision.BigInt}{operator|(const System.\-Numerics.\-Multiprecision.\-BigInt\&\-, const System.\-Numerics.\-Multiprecision.\-BigInt\&\-)}
& Returns bitwise inclusive OR of two \textbf{BigInt}
 values.

\\
\hyperlink{System.Numerics.Multiprecision.operator.complement.C.R.System.Numerics.Multiprecision.BigInt}{operator$\sim$(const System.\-Numerics.\-Multiprecision.\-BigInt\&\-)}
& Returns bitwise complement of \textbf{BigInt}
 value.

\\
\end{supertabular}

\end{flushleft}
\clearpage

\hypertarget{System.Numerics.Multiprecision.Abs.C.R.System.Numerics.Multiprecision.BigInt}{\subsubsection*{Abs(const System.Numerics.Multiprecision.BigInt\&) Function}}
\begin{flushleft}
Returns absolute value of given \hyperlink{System.Numerics.Multiprecision.BigInt}{BigInt}.

\end{flushleft}
\subsubsection*{Syntax}
\texttt{public System.Numerics.Multiprecision.BigInt Abs(const System.Numerics.Multiprecision.BigInt\& x);}
\subsubsection*{Parameters}
\begin{flushleft}
\begin{supertabular}[l]{!{\raggedright}p{1.30721cm}!{\raggedright}p{8.55945cm}!{\raggedright}p{2.65538cm}}
\textbf{Name}
& \textbf{Type}
& \textbf{Description}
\\
\hline
x
& \hyperlink{System.Numerics.Multiprecision.BigInt}{const System.\-Numerics.\-Multiprecision.\-BigInt\&\-}
& A \hyperlink{System.Numerics.Multiprecision.BigInt}{BigInt}.

\\
\end{supertabular}

\end{flushleft}
\subsubsection*{Returns}
\hyperlink{System.Numerics.Multiprecision.BigInt}{System.\-Numerics.\-Multiprecision.\-BigInt}
\begin{flushleft}
Returns absolute value of given \hyperlink{System.Numerics.Multiprecision.BigInt}{BigInt}.

\end{flushleft}
\clearpage

\hypertarget{System.Numerics.Multiprecision.ClearBit.R.System.Numerics.Multiprecision.BigInt.uint}{\subsubsection*{ClearBit(System.Numerics.Multiprecision.BigInt\&, uint) Function}}
\begin{flushleft}
Clear given bit of given \hyperlink{System.Numerics.Multiprecision.BigInt}{BigInt}.

\end{flushleft}
\subsubsection*{Syntax}
\texttt{public void ClearBit(System.Numerics.Multiprecision.BigInt\& x, uint bitIndex);}
\subsubsection*{Parameters}
\begin{flushleft}
\begin{supertabular}[l]{!{\raggedright}p{1.94573cm}!{\raggedright}p{7.92093cm}!{\raggedright}p{4.93333cm}}
\textbf{Name}
& \textbf{Type}
& \textbf{Description}
\\
\hline
x
& \hyperlink{System.Numerics.Multiprecision.BigInt}{System.\-Numerics.\-Multiprecision.\-BigInt\&\-}
& A reference to a \hyperlink{System.Numerics.Multiprecision.BigInt}{BigInt}.

\\
bitIndex
& uint
& Index of bit to clear.

\\
\end{supertabular}

\end{flushleft}
\clearpage

\hypertarget{System.Numerics.Multiprecision.SetBit.R.System.Numerics.Multiprecision.BigInt.uint}{\subsubsection*{SetBit(System.Numerics.Multiprecision.BigInt\&, uint) Function}}
\begin{flushleft}
Set given bit of given \hyperlink{System.Numerics.Multiprecision.BigInt}{BigInt}.

\end{flushleft}
\subsubsection*{Syntax}
\texttt{public void SetBit(System.Numerics.Multiprecision.BigInt\& x, uint bitIndex);}
\subsubsection*{Parameters}
\begin{flushleft}
\begin{supertabular}[l]{!{\raggedright}p{1.94573cm}!{\raggedright}p{7.92093cm}!{\raggedright}p{4.93333cm}}
\textbf{Name}
& \textbf{Type}
& \textbf{Description}
\\
\hline
x
& \hyperlink{System.Numerics.Multiprecision.BigInt}{System.\-Numerics.\-Multiprecision.\-BigInt\&\-}
& A reference to a \hyperlink{System.Numerics.Multiprecision.BigInt}{BigInt}.

\\
bitIndex
& uint
& Index of bit to set.

\\
\end{supertabular}

\end{flushleft}
\clearpage

\hypertarget{System.Numerics.Multiprecision.TestBit.R.System.Numerics.Multiprecision.BigInt.uint}{\subsubsection*{TestBit(System.Numerics.Multiprecision.BigInt\&, uint) Function}}
\begin{flushleft}
Returns true if given bit of given \hyperlink{System.Numerics.Multiprecision.BigInt}{BigInt} is set, false otherwise.

\end{flushleft}
\subsubsection*{Syntax}
\texttt{public bool TestBit(System.Numerics.Multiprecision.BigInt\& x, uint bitIndex);}
\subsubsection*{Parameters}
\begin{flushleft}
\begin{supertabular}[l]{!{\raggedright}p{1.94573cm}!{\raggedright}p{7.92093cm}!{\raggedright}p{4.93333cm}}
\textbf{Name}
& \textbf{Type}
& \textbf{Description}
\\
\hline
x
& \hyperlink{System.Numerics.Multiprecision.BigInt}{System.\-Numerics.\-Multiprecision.\-BigInt\&\-}
& A reference to a \hyperlink{System.Numerics.Multiprecision.BigInt}{BigInt}.

\\
bitIndex
& uint
& Index of bit to test.

\\
\end{supertabular}

\end{flushleft}
\subsubsection*{Returns}bool
\begin{flushleft}
Returns true if given bit of given \hyperlink{System.Numerics.Multiprecision.BigInt}{BigInt} is set, false otherwise.

\end{flushleft}
\clearpage

\hypertarget{System.Numerics.Multiprecision.ToggleBit.R.System.Numerics.Multiprecision.BigInt.uint}{\subsubsection*{ToggleBit(System.Numerics.Multiprecision.BigInt\&, uint) Function}}
\begin{flushleft}
Toggle given bit of given \hyperlink{System.Numerics.Multiprecision.BigInt}{BigInt}.

\end{flushleft}
\subsubsection*{Syntax}
\texttt{public void ToggleBit(System.Numerics.Multiprecision.BigInt\& x, uint bitIndex);}
\subsubsection*{Parameters}
\begin{flushleft}
\begin{supertabular}[l]{!{\raggedright}p{1.94573cm}!{\raggedright}p{7.92093cm}!{\raggedright}p{4.93333cm}}
\textbf{Name}
& \textbf{Type}
& \textbf{Description}
\\
\hline
x
& \hyperlink{System.Numerics.Multiprecision.BigInt}{System.\-Numerics.\-Multiprecision.\-BigInt\&\-}
& A reference to a \hyperlink{System.Numerics.Multiprecision.BigInt}{BigInt}.

\\
bitIndex
& uint
& Index of bit to toggle.

\\
\end{supertabular}

\end{flushleft}
\clearpage

\hypertarget{System.Numerics.Multiprecision.operator.remainder.C.R.System.Numerics.Multiprecision.BigInt.C.R.System.Numerics.Multiprecision.BigInt}{\subsubsection*{operator\%(const System.Numerics.Multiprecision.BigInt\&, const System.Numerics.Multiprecision.BigInt\&) Function}}
\begin{flushleft}
Returns the remainder when given \hyperlink{System.Numerics.Multiprecision.BigInt}{BigInt} is divided by another.

\end{flushleft}
\subsubsection*{Syntax}
\texttt{public System.Numerics.Multiprecision.BigInt operator\%(const System.Numerics.Multiprecision.BigInt\& left, const System.Numerics.Multiprecision.BigInt\& right);}
\subsubsection*{Parameters}
\begin{flushleft}
\begin{supertabular}[l]{!{\raggedright}p{1.30721cm}!{\raggedright}p{8.55945cm}!{\raggedright}p{2.65538cm}}
\textbf{Name}
& \textbf{Type}
& \textbf{Description}
\\
\hline
left
& \hyperlink{System.Numerics.Multiprecision.BigInt}{const System.\-Numerics.\-Multiprecision.\-BigInt\&\-}
& Divisor.

\\
right
& \hyperlink{System.Numerics.Multiprecision.BigInt}{const System.\-Numerics.\-Multiprecision.\-BigInt\&\-}
& Dividend.

\\
\end{supertabular}

\end{flushleft}
\subsubsection*{Returns}
\hyperlink{System.Numerics.Multiprecision.BigInt}{System.\-Numerics.\-Multiprecision.\-BigInt}
\begin{flushleft}
Returns the remainder when \hyperlink{System.Numerics.Multiprecision.operator.remainder.C.R.System.Numerics.Multiprecision.BigInt.C.R.System.Numerics.Multiprecision.BigInt.left}{left} is divided by 
\hyperlink{System.Numerics.Multiprecision.operator.remainder.C.R.System.Numerics.Multiprecision.BigInt.C.R.System.Numerics.Multiprecision.BigInt.right}{right}.

\end{flushleft}
\clearpage

\hypertarget{System.Numerics.Multiprecision.operator.and.C.R.System.Numerics.Multiprecision.BigInt.C.R.System.Numerics.Multiprecision.BigInt}{\subsubsection*{operator\&(const System.Numerics.Multiprecision.BigInt\&, const System.Numerics.Multiprecision.BigInt\&) Function}}
\begin{flushleft}
Returns bitwise AND of two \hyperlink{System.Numerics.Multiprecision.BigInt}{BigInt} values.

\end{flushleft}
\subsubsection*{Syntax}
\texttt{public System.Numerics.Multiprecision.BigInt operator\&(const System.Numerics.Multiprecision.BigInt\& left, const System.Numerics.Multiprecision.BigInt\& right);}
\subsubsection*{Parameters}
\begin{flushleft}
\begin{supertabular}[l]{!{\raggedright}p{1.30721cm}!{\raggedright}p{8.55945cm}!{\raggedright}p{3.66839cm}}
\textbf{Name}
& \textbf{Type}
& \textbf{Description}
\\
\hline
left
& \hyperlink{System.Numerics.Multiprecision.BigInt}{const System.\-Numerics.\-Multiprecision.\-BigInt\&\-}
& Left operand.

\\
right
& \hyperlink{System.Numerics.Multiprecision.BigInt}{const System.\-Numerics.\-Multiprecision.\-BigInt\&\-}
& Right operand.

\\
\end{supertabular}

\end{flushleft}
\subsubsection*{Returns}
\hyperlink{System.Numerics.Multiprecision.BigInt}{System.\-Numerics.\-Multiprecision.\-BigInt}
\begin{flushleft}
Returns bitwise AND of given \hyperlink{System.Numerics.Multiprecision.BigInt}{BigInt} values.

\end{flushleft}
\clearpage

\hypertarget{System.Numerics.Multiprecision.operator.times.C.R.System.Numerics.Multiprecision.BigInt.C.R.System.Numerics.Multiprecision.BigInt}{\subsubsection*{operator*(const System.Numerics.Multiprecision.BigInt\&, const System.Numerics.Multiprecision.BigInt\&) Function}}
\begin{flushleft}
Returns the product of given \hyperlink{System.Numerics.Multiprecision.BigInt}{BigInt} value multiplied by another.

\end{flushleft}
\subsubsection*{Syntax}
\texttt{public System.Numerics.Multiprecision.BigInt operator*(const System.Numerics.Multiprecision.BigInt\& left, const System.Numerics.Multiprecision.BigInt\& right);}
\subsubsection*{Parameters}
\begin{flushleft}
\begin{supertabular}[l]{!{\raggedright}p{1.30721cm}!{\raggedright}p{8.55945cm}!{\raggedright}p{3.66839cm}}
\textbf{Name}
& \textbf{Type}
& \textbf{Description}
\\
\hline
left
& \hyperlink{System.Numerics.Multiprecision.BigInt}{const System.\-Numerics.\-Multiprecision.\-BigInt\&\-}
& Left operand.

\\
right
& \hyperlink{System.Numerics.Multiprecision.BigInt}{const System.\-Numerics.\-Multiprecision.\-BigInt\&\-}
& Right operand.

\\
\end{supertabular}

\end{flushleft}
\subsubsection*{Returns}
\hyperlink{System.Numerics.Multiprecision.BigInt}{System.\-Numerics.\-Multiprecision.\-BigInt}
\begin{flushleft}
Returns the product of given \hyperlink{System.Numerics.Multiprecision.BigInt}{BigInt} value multiplied by another.

\end{flushleft}
\clearpage

\hypertarget{System.Numerics.Multiprecision.operator.plus.C.R.System.Numerics.Multiprecision.BigInt.C.R.System.Numerics.Multiprecision.BigInt}{\subsubsection*{operator+(const System.Numerics.Multiprecision.BigInt\&, const System.Numerics.Multiprecision.BigInt\&) Function}}
\begin{flushleft}
Returns the sum of given \hyperlink{System.Numerics.Multiprecision.BigInt}{BigInt} value added to another.

\end{flushleft}
\subsubsection*{Syntax}
\texttt{public System.Numerics.Multiprecision.BigInt operator+(const System.Numerics.Multiprecision.BigInt\& left, const System.Numerics.Multiprecision.BigInt\& right);}
\subsubsection*{Parameters}
\begin{flushleft}
\begin{supertabular}[l]{!{\raggedright}p{1.30721cm}!{\raggedright}p{8.55945cm}!{\raggedright}p{3.66839cm}}
\textbf{Name}
& \textbf{Type}
& \textbf{Description}
\\
\hline
left
& \hyperlink{System.Numerics.Multiprecision.BigInt}{const System.\-Numerics.\-Multiprecision.\-BigInt\&\-}
& Left operand.

\\
right
& \hyperlink{System.Numerics.Multiprecision.BigInt}{const System.\-Numerics.\-Multiprecision.\-BigInt\&\-}
& Right operand.

\\
\end{supertabular}

\end{flushleft}
\subsubsection*{Returns}
\hyperlink{System.Numerics.Multiprecision.BigInt}{System.\-Numerics.\-Multiprecision.\-BigInt}
\begin{flushleft}
Returns the sum of given \hyperlink{System.Numerics.Multiprecision.BigInt}{BigInt} value added to another.

\end{flushleft}
\clearpage

\hypertarget{System.Numerics.Multiprecision.operator.minus.C.R.System.Numerics.Multiprecision.BigInt}{\subsubsection*{operator-(const System.Numerics.Multiprecision.BigInt\&) Function}}
\begin{flushleft}
Returns the negation of \hyperlink{System.Numerics.Multiprecision.BigInt}{BigInt}.

\end{flushleft}
\subsubsection*{Syntax}
\texttt{public System.Numerics.Multiprecision.BigInt operator-(const System.Numerics.Multiprecision.BigInt\& x);}
\subsubsection*{Parameters}
\begin{flushleft}
\begin{supertabular}[l]{!{\raggedright}p{1.30721cm}!{\raggedright}p{8.55945cm}!{\raggedright}p{2.65538cm}}
\textbf{Name}
& \textbf{Type}
& \textbf{Description}
\\
\hline
x
& \hyperlink{System.Numerics.Multiprecision.BigInt}{const System.\-Numerics.\-Multiprecision.\-BigInt\&\-}
& A value.

\\
\end{supertabular}

\end{flushleft}
\subsubsection*{Returns}
\hyperlink{System.Numerics.Multiprecision.BigInt}{System.\-Numerics.\-Multiprecision.\-BigInt}
\begin{flushleft}
Returns the negation of \hyperlink{System.Numerics.Multiprecision.BigInt}{BigInt}.

\end{flushleft}
\clearpage

\hypertarget{System.Numerics.Multiprecision.operator.minus.C.R.System.Numerics.Multiprecision.BigInt.C.R.System.Numerics.Multiprecision.BigInt}{\subsubsection*{operator-(const System.Numerics.Multiprecision.BigInt\&, const System.Numerics.Multiprecision.BigInt\&) Function}}
\begin{flushleft}
Returns the difference of given \hyperlink{System.Numerics.Multiprecision.BigInt}{BigInt} value subtracted from another.

\end{flushleft}
\subsubsection*{Syntax}
\texttt{public System.Numerics.Multiprecision.BigInt operator-(const System.Numerics.Multiprecision.BigInt\& left, const System.Numerics.Multiprecision.BigInt\& right);}
\subsubsection*{Parameters}
\begin{flushleft}
\begin{supertabular}[l]{!{\raggedright}p{1.30721cm}!{\raggedright}p{8.55945cm}!{\raggedright}p{3.66839cm}}
\textbf{Name}
& \textbf{Type}
& \textbf{Description}
\\
\hline
left
& \hyperlink{System.Numerics.Multiprecision.BigInt}{const System.\-Numerics.\-Multiprecision.\-BigInt\&\-}
& Left operand.

\\
right
& \hyperlink{System.Numerics.Multiprecision.BigInt}{const System.\-Numerics.\-Multiprecision.\-BigInt\&\-}
& Right operand.

\\
\end{supertabular}

\end{flushleft}
\subsubsection*{Returns}
\hyperlink{System.Numerics.Multiprecision.BigInt}{System.\-Numerics.\-Multiprecision.\-BigInt}
\begin{flushleft}
Returns the difference of given \hyperlink{System.Numerics.Multiprecision.BigInt}{BigInt} value subtracted from another.

\end{flushleft}
\clearpage

\hypertarget{System.Numerics.Multiprecision.operator.divides.C.R.System.Numerics.Multiprecision.BigInt.C.R.System.Numerics.Multiprecision.BigInt}{\subsubsection*{operator/(const System.Numerics.Multiprecision.BigInt\&, const System.Numerics.Multiprecision.BigInt\&) Function}}
\begin{flushleft}
Returns the quotient when given \hyperlink{System.Numerics.Multiprecision.BigInt}{BigInt} is divided by another.

\end{flushleft}
\subsubsection*{Syntax}
\texttt{public System.Numerics.Multiprecision.BigInt operator/(const System.Numerics.Multiprecision.BigInt\& left, const System.Numerics.Multiprecision.BigInt\& right);}
\subsubsection*{Parameters}
\begin{flushleft}
\begin{supertabular}[l]{!{\raggedright}p{1.30721cm}!{\raggedright}p{8.55945cm}!{\raggedright}p{2.65538cm}}
\textbf{Name}
& \textbf{Type}
& \textbf{Description}
\\
\hline
left
& \hyperlink{System.Numerics.Multiprecision.BigInt}{const System.\-Numerics.\-Multiprecision.\-BigInt\&\-}
& Divisor.

\\
right
& \hyperlink{System.Numerics.Multiprecision.BigInt}{const System.\-Numerics.\-Multiprecision.\-BigInt\&\-}
& Dividend.

\\
\end{supertabular}

\end{flushleft}
\subsubsection*{Returns}
\hyperlink{System.Numerics.Multiprecision.BigInt}{System.\-Numerics.\-Multiprecision.\-BigInt}
\begin{flushleft}
Returns the quotient when given \hyperlink{System.Numerics.Multiprecision.BigInt}{BigInt} is divided by another.

\end{flushleft}
\subsubsection*{Remarks}
\begin{flushleft}
Quotient is rounded towards zero.

\end{flushleft}
\clearpage

\hypertarget{System.Numerics.Multiprecision.operator.less.C.R.System.Numerics.Multiprecision.BigInt.C.R.System.Numerics.Multiprecision.BigInt}{\subsubsection*{operator$<$(const System.Numerics.Multiprecision.BigInt\&, const System.Numerics.Multiprecision.BigInt\&) Function}}
\begin{flushleft}
Returns true if the first \hyperlink{System.Numerics.Multiprecision.BigInt}{BigInt} is less than the second \hyperlink{System.Numerics.Multiprecision.BigInt}{BigInt}, false otherwise.

\end{flushleft}
\subsubsection*{Syntax}
\texttt{public bool operator$<$(const System.Numerics.Multiprecision.BigInt\& left, const System.Numerics.Multiprecision.BigInt\& right);}
\subsubsection*{Parameters}
\begin{flushleft}
\begin{supertabular}[l]{!{\raggedright}p{1.30721cm}!{\raggedright}p{8.55945cm}!{\raggedright}p{3.66839cm}}
\textbf{Name}
& \textbf{Type}
& \textbf{Description}
\\
\hline
left
& \hyperlink{System.Numerics.Multiprecision.BigInt}{const System.\-Numerics.\-Multiprecision.\-BigInt\&\-}
& Left operand.

\\
right
& \hyperlink{System.Numerics.Multiprecision.BigInt}{const System.\-Numerics.\-Multiprecision.\-BigInt\&\-}
& Right operand.

\\
\end{supertabular}

\end{flushleft}
\subsubsection*{Returns}bool
\begin{flushleft}
Returns true if the first \hyperlink{System.Numerics.Multiprecision.BigInt}{BigInt} is less than the second \hyperlink{System.Numerics.Multiprecision.BigInt}{BigInt}, false otherwise.

\end{flushleft}
\clearpage

\hypertarget{System.Numerics.Multiprecision.operator.shiftLeft.R.System.IO.OutputStream.C.R.System.Numerics.Multiprecision.BigInt}{\subsubsection*{operator$<$$<$(System.IO.OutputStream\&, const System.Numerics.Multiprecision.BigInt\&) Function}}
\begin{flushleft}
Puts the value of the given \hyperlink{System.Numerics.Multiprecision.BigInt}{BigInt} to the given output stream as string of decimal digits prefixed by minus sign if the value is negative.

\end{flushleft}
\subsubsection*{Syntax}
\texttt{public System.IO.OutputStream\& operator$<$$<$(System.IO.OutputStream\& s, const System.Numerics.Multiprecision.BigInt\& x);}
\subsubsection*{Parameters}
\begin{flushleft}
\begin{supertabular}[l]{!{\raggedright}p{1.30721cm}!{\raggedright}p{8.55945cm}!{\raggedright}p{4.49442cm}}
\textbf{Name}
& \textbf{Type}
& \textbf{Description}
\\
\hline
s
& System.\-IO.\-OutputStream\&\-
& An output stream.

\\
x
& \hyperlink{System.Numerics.Multiprecision.BigInt}{const System.\-Numerics.\-Multiprecision.\-BigInt\&\-}
& A \hyperlink{System.Numerics.Multiprecision.BigInt}{BigInt} value.

\\
\end{supertabular}

\end{flushleft}
\subsubsection*{Returns}System.\-IO.\-OutputStream\&\-
\begin{flushleft}
Returns a reference to the output stream.

\end{flushleft}
\clearpage

\hypertarget{System.Numerics.Multiprecision.operator.equal.C.R.System.Numerics.Multiprecision.BigInt.C.R.System.Numerics.Multiprecision.BigInt}{\subsubsection*{operator==(const System.Numerics.Multiprecision.BigInt\&, const System.Numerics.Multiprecision.BigInt\&) Function}}
\begin{flushleft}
Returns true if the first \hyperlink{System.Numerics.Multiprecision.BigInt}{BigInt} is equal to the second \hyperlink{System.Numerics.Multiprecision.BigInt}{BigInt}, false otherwise.

\end{flushleft}
\subsubsection*{Syntax}
\texttt{public bool operator==(const System.Numerics.Multiprecision.BigInt\& left, const System.Numerics.Multiprecision.BigInt\& right);}
\subsubsection*{Parameters}
\begin{flushleft}
\begin{supertabular}[l]{!{\raggedright}p{1.30721cm}!{\raggedright}p{8.55945cm}!{\raggedright}p{3.66839cm}}
\textbf{Name}
& \textbf{Type}
& \textbf{Description}
\\
\hline
left
& \hyperlink{System.Numerics.Multiprecision.BigInt}{const System.\-Numerics.\-Multiprecision.\-BigInt\&\-}
& Left operand.

\\
right
& \hyperlink{System.Numerics.Multiprecision.BigInt}{const System.\-Numerics.\-Multiprecision.\-BigInt\&\-}
& Right operand.

\\
\end{supertabular}

\end{flushleft}
\subsubsection*{Returns}bool
\begin{flushleft}
Returns true if the first \hyperlink{System.Numerics.Multiprecision.BigInt}{BigInt} is equal to the second \hyperlink{System.Numerics.Multiprecision.BigInt}{BigInt}, false otherwise.

\end{flushleft}
\clearpage

\hypertarget{System.Numerics.Multiprecision.operator.xor.C.R.System.Numerics.Multiprecision.BigInt.C.R.System.Numerics.Multiprecision.BigInt}{\subsubsection*{operator\^{}(const System.Numerics.Multiprecision.BigInt\&, const System.Numerics.Multiprecision.BigInt\&) Function}}
\begin{flushleft}
Returns bitwise XOR of two \hyperlink{System.Numerics.Multiprecision.BigInt}{BigInt} values.

\end{flushleft}
\subsubsection*{Syntax}
\texttt{public System.Numerics.Multiprecision.BigInt operator\^{}(const System.Numerics.Multiprecision.BigInt\& left, const System.Numerics.Multiprecision.BigInt\& right);}
\subsubsection*{Parameters}
\begin{flushleft}
\begin{supertabular}[l]{!{\raggedright}p{1.30721cm}!{\raggedright}p{8.55945cm}!{\raggedright}p{3.66839cm}}
\textbf{Name}
& \textbf{Type}
& \textbf{Description}
\\
\hline
left
& \hyperlink{System.Numerics.Multiprecision.BigInt}{const System.\-Numerics.\-Multiprecision.\-BigInt\&\-}
& Left operand.

\\
right
& \hyperlink{System.Numerics.Multiprecision.BigInt}{const System.\-Numerics.\-Multiprecision.\-BigInt\&\-}
& Right operand.

\\
\end{supertabular}

\end{flushleft}
\subsubsection*{Returns}
\hyperlink{System.Numerics.Multiprecision.BigInt}{System.\-Numerics.\-Multiprecision.\-BigInt}
\begin{flushleft}
Returns bitwise XOR of two \hyperlink{System.Numerics.Multiprecision.BigInt}{BigInt} values.

\end{flushleft}
\clearpage

\hypertarget{System.Numerics.Multiprecision.operator.or.C.R.System.Numerics.Multiprecision.BigInt.C.R.System.Numerics.Multiprecision.BigInt}{\subsubsection*{operator$|$(const System.Numerics.Multiprecision.BigInt\&, const System.Numerics.Multiprecision.BigInt\&) Function}}
\begin{flushleft}
Returns bitwise inclusive OR of two \hyperlink{System.Numerics.Multiprecision.BigInt}{BigInt} values.

\end{flushleft}
\subsubsection*{Syntax}
\texttt{public System.Numerics.Multiprecision.BigInt operator$|$(const System.Numerics.Multiprecision.BigInt\& left, const System.Numerics.Multiprecision.BigInt\& right);}
\subsubsection*{Parameters}
\begin{flushleft}
\begin{supertabular}[l]{!{\raggedright}p{1.30721cm}!{\raggedright}p{8.55945cm}!{\raggedright}p{3.66839cm}}
\textbf{Name}
& \textbf{Type}
& \textbf{Description}
\\
\hline
left
& \hyperlink{System.Numerics.Multiprecision.BigInt}{const System.\-Numerics.\-Multiprecision.\-BigInt\&\-}
& Left operand.

\\
right
& \hyperlink{System.Numerics.Multiprecision.BigInt}{const System.\-Numerics.\-Multiprecision.\-BigInt\&\-}
& Right operand.

\\
\end{supertabular}

\end{flushleft}
\subsubsection*{Returns}
\hyperlink{System.Numerics.Multiprecision.BigInt}{System.\-Numerics.\-Multiprecision.\-BigInt}
\begin{flushleft}
Returns bitwise inclusive OR of two \hyperlink{System.Numerics.Multiprecision.BigInt}{BigInt} values.

\end{flushleft}
\clearpage

\hypertarget{System.Numerics.Multiprecision.operator.complement.C.R.System.Numerics.Multiprecision.BigInt}{\subsubsection*{operator$\sim$(const System.Numerics.Multiprecision.BigInt\&) Function}}
\begin{flushleft}
Returns bitwise complement of \hyperlink{System.Numerics.Multiprecision.BigInt}{BigInt} value.

\end{flushleft}
\subsubsection*{Syntax}
\texttt{public System.Numerics.Multiprecision.BigInt operator$\sim$(const System.Numerics.Multiprecision.BigInt\& x);}
\subsubsection*{Parameters}
\begin{flushleft}
\begin{supertabular}[l]{!{\raggedright}p{1.30721cm}!{\raggedright}p{8.55945cm}!{\raggedright}p{2.65538cm}}
\textbf{Name}
& \textbf{Type}
& \textbf{Description}
\\
\hline
x
& \hyperlink{System.Numerics.Multiprecision.BigInt}{const System.\-Numerics.\-Multiprecision.\-BigInt\&\-}
& Operand.

\\
\end{supertabular}

\end{flushleft}
\subsubsection*{Returns}
\hyperlink{System.Numerics.Multiprecision.BigInt}{System.\-Numerics.\-Multiprecision.\-BigInt}
\begin{flushleft}
Returns bitwise complement of \hyperlink{System.Numerics.Multiprecision.BigInt}{BigInt} value.

\end{flushleft}
\clearpage

\hypertarget{System.Numerics.Multiprecision.BigRational}{\section{BigRational Class}}\begin{flushleft}
An arbitrary precision rational nunber type.

\end{flushleft}

\subsection*{Syntax}\texttt{public class BigRational;}
\subsection{Member Functions}
\begin{flushleft}
\begin{supertabular}[l]{!{\raggedright}p{7.575cm}!{\raggedright}p{7.575cm}}
\textbf{Member Function}
& \textbf{Description}
\\
\hline
\hyperlink{System.Numerics.Multiprecision.BigRational.constructor.P.System.Numerics.Multiprecision.BigRational}{BigRational()}
& Default constructor. Creates an instance of arbitrary precision rational number and initializes it to zero.

\\
\hyperlink{System.Numerics.Multiprecision.BigRational.constructor.P.System.Numerics.Multiprecision.BigRational.C.R.System.Numerics.Multiprecision.BigRational}{BigRational(const System.\-Numerics.\-Multiprecision.\-BigRational\&\-)}
& Copy constructor. 

\\
\hyperlink{System.Numerics.Multiprecision.BigRational.operator.assign.P.System.Numerics.Multiprecision.BigRational.C.R.System.Numerics.Multiprecision.BigRational}{operator=(const System.\-Numerics.\-Multiprecision.\-BigRational\&\-)}
& Copy assignment.

\\
\hyperlink{System.Numerics.Multiprecision.BigRational.constructor.P.System.Numerics.Multiprecision.BigRational.RR.System.Numerics.Multiprecision.BigRational}{BigRational(System.\-Numerics.\-Multiprecision.\-BigRational\&\-\&\-)}
& Move constructor.

\\
\hyperlink{System.Numerics.Multiprecision.BigRational.operator.assign.P.System.Numerics.Multiprecision.BigRational.RR.System.Numerics.Multiprecision.BigRational}{operator=(System.\-Numerics.\-Multiprecision.\-BigRational\&\-\&\-)}
& Move assignment.

\\
\hyperlink{System.Numerics.Multiprecision.BigRational.constructor.P.System.Numerics.Multiprecision.BigRational.C.R.System.Numerics.Multiprecision.BigInt}{BigRational(const System.\-Numerics.\-Multiprecision.\-BigInt\&\-)}
& Constructor. Creates an instance of arbitrary precision rational type and initializes it from the given arbitrary precision integer type.

\\
\hyperlink{System.Numerics.Multiprecision.BigRational.constructor.P.System.Numerics.Multiprecision.BigRational.C.R.System.String.char}{BigRational(const System.\-String$<$\-char$>$\-\&\-)}
& Constructor. Creates an instance of arbitrary precision rational type and initializes it from the given string.

\\
\hyperlink{System.Numerics.Multiprecision.BigRational.constructor.P.System.Numerics.Multiprecision.BigRational.C.R.System.String.char.int}{BigRational(const System.\-String$<$\-char$>$\-\&\-, int)}
& Constructor. Creates an instance of arbitrary precision rational type and initializes it from the given string of given base.

\\
\hyperlink{System.Numerics.Multiprecision.BigRational.constructor.P.System.Numerics.Multiprecision.BigRational.int}{BigRational(int)}
& Constructor. Creates an instance of arbitrary precision rational type and initializes it from the given integer value.

\\
\hyperlink{System.Numerics.Multiprecision.BigRational.constructor.P.System.Numerics.Multiprecision.BigRational.uint}{BigRational(uint)}
& Constructor. Creates an instance of arbitrary precision rational type and initializes it from the given unsigned integer value.

\\
\hyperlink{System.Numerics.Multiprecision.BigRational.Denominator.C.P.System.Numerics.Multiprecision.BigRational}{Denominator() const}
& Returns the denominator of the arbitrary precision ration number.

\\
\hyperlink{System.Numerics.Multiprecision.BigRational.Handle.C.P.System.Numerics.Multiprecision.BigRational}{Handle() const}
& Returns a handle to the GNU MP library arbitrary precision rational representation.

\\
\hyperlink{System.Numerics.Multiprecision.BigRational.Numerator.C.P.System.Numerics.Multiprecision.BigRational}{Numerator() const}
& Returns the numerator of the arbitrary precision ration number.

\\
\hyperlink{System.Numerics.Multiprecision.BigRational.ToString.C.P.System.Numerics.Multiprecision.BigRational}{ToString() const}
& Returns the value of the \textbf{BigRational}
 as a string.

\\
\hyperlink{System.Numerics.Multiprecision.BigRational.ToString.C.P.System.Numerics.Multiprecision.BigRational.int}{ToString(int) const}
& Returns the value of the \textbf{BigRational}
 as a string of given base.

\\
\hyperlink{System.Numerics.Multiprecision.BigRational.destructor.P.System.Numerics.Multiprecision.BigRational}{$\sim$BigRational()}
& Frees memory occupied by the \textbf{BigRational}
 instance.

\\
\end{supertabular}

\end{flushleft}
\clearpage

\hypertarget{System.Numerics.Multiprecision.BigRational.constructor.P.System.Numerics.Multiprecision.BigRational}{\subsubsection*{BigRational() Member Function}}
\begin{flushleft}
Default constructor. Creates an instance of arbitrary precision rational number and initializes it to zero.

\end{flushleft}
\subsubsection*{Syntax}\texttt{public BigRational();}
\clearpage

\hypertarget{System.Numerics.Multiprecision.BigRational.constructor.P.System.Numerics.Multiprecision.BigRational.C.R.System.Numerics.Multiprecision.BigRational}{\subsubsection*{BigRational(const System.Numerics.Multiprecision.BigRational\&) Member Function}}\begin{flushleft}
Copy constructor. 

\end{flushleft}
\subsubsection*{Syntax}
\texttt{public BigRational(const System.Numerics.Multiprecision.BigRational\& that);}
\subsubsection*{Parameters}
\begin{flushleft}
\begin{supertabular}[l]{!{\raggedright}p{1.30721cm}!{\raggedright}p{8.55945cm}!{\raggedright}p{4.93333cm}}
\textbf{Name}
& \textbf{Type}
& \textbf{Description}
\\
\hline
that
& \hyperlink{System.Numerics.Multiprecision.BigRational}{const System.\-Numerics.\-Multiprecision.\-BigRational\&\-}
& A \hyperlink{System.Numerics.Multiprecision.BigRational}{BigRational} to copy from.

\\
\end{supertabular}

\end{flushleft}
\clearpage

\hypertarget{System.Numerics.Multiprecision.BigRational.operator.assign.P.System.Numerics.Multiprecision.BigRational.C.R.System.Numerics.Multiprecision.BigRational}{\subsubsection*{operator=(const System.Numerics.Multiprecision.BigRational\&) Member Function}}\begin{flushleft}
Copy assignment.

\end{flushleft}
\subsubsection*{Syntax}
\texttt{public void operator=(const System.Numerics.Multiprecision.BigRational\& that);}
\subsubsection*{Parameters}
\begin{flushleft}
\begin{supertabular}[l]{!{\raggedright}p{1.30721cm}!{\raggedright}p{8.55945cm}!{\raggedright}p{4.93333cm}}
\textbf{Name}
& \textbf{Type}
& \textbf{Description}
\\
\hline
that
& \hyperlink{System.Numerics.Multiprecision.BigRational}{const System.\-Numerics.\-Multiprecision.\-BigRational\&\-}
& A \hyperlink{System.Numerics.Multiprecision.BigRational}{BigRational} to assign from.

\\
\end{supertabular}

\end{flushleft}
\clearpage

\hypertarget{System.Numerics.Multiprecision.BigRational.constructor.P.System.Numerics.Multiprecision.BigRational.RR.System.Numerics.Multiprecision.BigRational}{\subsubsection*{BigRational(System.Numerics.Multiprecision.BigRational\&\&) Member Function}}\begin{flushleft}
Move constructor.

\end{flushleft}
\subsubsection*{Syntax}
\texttt{public BigRational(System.Numerics.Multiprecision.BigRational\&\& that);}
\subsubsection*{Parameters}
\begin{flushleft}
\begin{supertabular}[l]{!{\raggedright}p{1.30721cm}!{\raggedright}p{8.55945cm}!{\raggedright}p{4.93333cm}}
\textbf{Name}
& \textbf{Type}
& \textbf{Description}
\\
\hline
that
& \hyperlink{System.Numerics.Multiprecision.BigRational}{System.\-Numerics.\-Multiprecision.\-BigRational\&\-\&\-}
& A \hyperlink{System.Numerics.Multiprecision.BigRational}{BigRational} to move from.

\\
\end{supertabular}

\end{flushleft}
\clearpage

\hypertarget{System.Numerics.Multiprecision.BigRational.operator.assign.P.System.Numerics.Multiprecision.BigRational.RR.System.Numerics.Multiprecision.BigRational}{\subsubsection*{operator=(System.Numerics.Multiprecision.BigRational\&\&) Member Function}}\begin{flushleft}
Move assignment.

\end{flushleft}
\subsubsection*{Syntax}
\texttt{public void operator=(System.Numerics.Multiprecision.BigRational\&\& that);}
\subsubsection*{Parameters}
\begin{flushleft}
\begin{supertabular}[l]{!{\raggedright}p{1.30721cm}!{\raggedright}p{8.55945cm}!{\raggedright}p{4.93333cm}}
\textbf{Name}
& \textbf{Type}
& \textbf{Description}
\\
\hline
that
& \hyperlink{System.Numerics.Multiprecision.BigRational}{System.\-Numerics.\-Multiprecision.\-BigRational\&\-\&\-}
& A \hyperlink{System.Numerics.Multiprecision.BigRational}{BigRational} to assign from.

\\
\end{supertabular}

\end{flushleft}
\clearpage

\hypertarget{System.Numerics.Multiprecision.BigRational.constructor.P.System.Numerics.Multiprecision.BigRational.C.R.System.Numerics.Multiprecision.BigInt}{\subsubsection*{BigRational(const System.Numerics.Multiprecision.BigInt\&) Member Function}}
\begin{flushleft}
Constructor. Creates an instance of arbitrary precision rational type and initializes it from the given arbitrary precision integer type.

\end{flushleft}
\subsubsection*{Syntax}
\texttt{public BigRational(const System.Numerics.Multiprecision.BigInt\& that);}
\subsubsection*{Parameters}
\begin{flushleft}
\begin{supertabular}[l]{!{\raggedright}p{1.30721cm}!{\raggedright}p{8.55945cm}!{\raggedright}p{4.93333cm}}
\textbf{Name}
& \textbf{Type}
& \textbf{Description}
\\
\hline
that
& \hyperlink{System.Numerics.Multiprecision.BigInt}{const System.\-Numerics.\-Multiprecision.\-BigInt\&\-}
& An arbitrary precision integer type value to initialize from.

\\
\end{supertabular}

\end{flushleft}
\clearpage

\hypertarget{System.Numerics.Multiprecision.BigRational.constructor.P.System.Numerics.Multiprecision.BigRational.C.R.System.String.char}{\subsubsection*{BigRational(const System.String$<$char$>$\&) Member Function}}
\begin{flushleft}
Constructor. Creates an instance of arbitrary precision rational type and initializes it from the given string.

\end{flushleft}
\subsubsection*{Syntax}
\texttt{public BigRational(const System.String$<$char$>$\& str);}
\subsubsection*{Parameters}
\begin{flushleft}
\begin{supertabular}[l]{lll}
\textbf{Name}
& \textbf{Type}
& \textbf{Description}
\\
\hline
str
& const System.\-String$<$\-char$>$\-\&\-
& A string to initialize from.

\\
\end{supertabular}

\end{flushleft}
\clearpage

\hypertarget{System.Numerics.Multiprecision.BigRational.constructor.P.System.Numerics.Multiprecision.BigRational.C.R.System.String.char.int}{\subsubsection*{BigRational(const System.String$<$char$>$\&, int) Member Function}}
\begin{flushleft}
Constructor. Creates an instance of arbitrary precision rational type and initializes it from the given string of given base.

\end{flushleft}
\subsubsection*{Syntax}
\texttt{public BigRational(const System.String$<$char$>$\& str, int base\_);}
\subsubsection*{Parameters}
\begin{flushleft}
\begin{supertabular}[l]{lll}
\textbf{Name}
& \textbf{Type}
& \textbf{Description}
\\
\hline
str
& const System.\-String$<$\-char$>$\-\&\-
& A string to initialize from.

\\
base\_
& int
& Base of string.

\\
\end{supertabular}

\end{flushleft}
\clearpage

\hypertarget{System.Numerics.Multiprecision.BigRational.constructor.P.System.Numerics.Multiprecision.BigRational.int}{\subsubsection*{BigRational(int) Member Function}}
\begin{flushleft}
Constructor. Creates an instance of arbitrary precision rational type and initializes it from the given integer value.

\end{flushleft}
\subsubsection*{Syntax}\texttt{public BigRational(int that);}

\subsubsection*{Parameters}
\begin{flushleft}
\begin{supertabular}[l]{lll}
\textbf{Name}
& \textbf{Type}
& \textbf{Description}
\\
\hline
that
& int
& An integer value to initalize from.

\\
\end{supertabular}

\end{flushleft}
\clearpage

\hypertarget{System.Numerics.Multiprecision.BigRational.constructor.P.System.Numerics.Multiprecision.BigRational.uint}{\subsubsection*{BigRational(uint) Member Function}}
\begin{flushleft}
Constructor. Creates an instance of arbitrary precision rational type and initializes it from the given unsigned integer value.

\end{flushleft}
\subsubsection*{Syntax}\texttt{public BigRational(uint that);}

\subsubsection*{Parameters}
\begin{flushleft}
\begin{supertabular}[l]{lll}
\textbf{Name}
& \textbf{Type}
& \textbf{Description}
\\
\hline
that
& uint
& An unsigned integer value to initalize from.

\\
\end{supertabular}

\end{flushleft}
\clearpage

\hypertarget{System.Numerics.Multiprecision.BigRational.Denominator.C.P.System.Numerics.Multiprecision.BigRational}{\subsubsection*{Denominator() const Member Function}}
\begin{flushleft}
Returns the denominator of the arbitrary precision ration number.

\end{flushleft}
\subsubsection*{Syntax}
\texttt{public System.Numerics.Multiprecision.BigInt Denominator() const;}
\subsubsection*{Returns}
\hyperlink{System.Numerics.Multiprecision.BigInt}{System.\-Numerics.\-Multiprecision.\-BigInt}
\begin{flushleft}
Returns the denominator of the arbitrary precision ration number.

\end{flushleft}
\clearpage

\hypertarget{System.Numerics.Multiprecision.BigRational.Handle.C.P.System.Numerics.Multiprecision.BigRational}{\subsubsection*{Handle() const Member Function}}
\begin{flushleft}
Returns a handle to the GNU MP library arbitrary precision rational representation.

\end{flushleft}
\subsubsection*{Syntax}\texttt{public void* Handle() const;}

\subsubsection*{Returns}void*
\begin{flushleft}
Returns a handle to the GNU MP library arbitrary precision rational representation.

\end{flushleft}
\clearpage

\hypertarget{System.Numerics.Multiprecision.BigRational.Numerator.C.P.System.Numerics.Multiprecision.BigRational}{\subsubsection*{Numerator() const Member Function}}
\begin{flushleft}
Returns the numerator of the arbitrary precision ration number.

\end{flushleft}
\subsubsection*{Syntax}
\texttt{public System.Numerics.Multiprecision.BigInt Numerator() const;}
\subsubsection*{Returns}
\hyperlink{System.Numerics.Multiprecision.BigInt}{System.\-Numerics.\-Multiprecision.\-BigInt}
\begin{flushleft}
Returns the numerator of the arbitrary precision ration number.

\end{flushleft}
\clearpage

\hypertarget{System.Numerics.Multiprecision.BigRational.ToString.C.P.System.Numerics.Multiprecision.BigRational}{\subsubsection*{ToString() const Member Function}}
\begin{flushleft}
Returns the value of the \hyperlink{System.Numerics.Multiprecision.BigRational}{BigRational} as a string.

\end{flushleft}
\subsubsection*{Syntax}
\texttt{public System.String$<$char$>$ ToString() const;}
\subsubsection*{Returns}System.\-String$<$\-char$>$\-
\begin{flushleft}
Returns the value of the \hyperlink{System.Numerics.Multiprecision.BigRational}{BigRational} as a string.

\end{flushleft}
\clearpage

\hypertarget{System.Numerics.Multiprecision.BigRational.ToString.C.P.System.Numerics.Multiprecision.BigRational.int}{\subsubsection*{ToString(int) const Member Function}}
\begin{flushleft}
Returns the value of the \hyperlink{System.Numerics.Multiprecision.BigRational}{BigRational} as a string of given base.

\end{flushleft}
\subsubsection*{Syntax}
\texttt{public System.String$<$char$>$ ToString(int base\_) const;}
\subsubsection*{Parameters}
\begin{flushleft}
\begin{supertabular}[l]{lll}
\textbf{Name}
& \textbf{Type}
& \textbf{Description}
\\
\hline
base\_
& int
& Base.

\\
\end{supertabular}

\end{flushleft}
\subsubsection*{Returns}System.\-String$<$\-char$>$\-
\begin{flushleft}
Returns the value of the \hyperlink{System.Numerics.Multiprecision.BigRational}{BigRational} as a string of given base.

\end{flushleft}
\clearpage

\hypertarget{System.Numerics.Multiprecision.BigRational.destructor.P.System.Numerics.Multiprecision.BigRational}{\subsubsection*{$\sim$BigRational() Member Function}}
\begin{flushleft}
Frees memory occupied by the \hyperlink{System.Numerics.Multiprecision.BigRational}{BigRational} instance.

\end{flushleft}
\subsubsection*{Syntax}\texttt{public $\sim$BigRational();}

\subsection{Nonmember Functions}
\begin{flushleft}
\begin{supertabular}[l]{!{\raggedright}p{7.575cm}!{\raggedright}p{7.575cm}}
\textbf{Function}
& \textbf{Description}
\\
\hline
\hyperlink{System.Numerics.Multiprecision.Abs.C.R.System.Numerics.Multiprecision.BigRational}{Abs(const System.\-Numerics.\-Multiprecision.\-BigRational\&\-)}
& Returns the absolute value of the given \textbf{BigRational}
 value.

\\
\hyperlink{System.Numerics.Multiprecision.operator.times.C.R.System.Numerics.Multiprecision.BigRational.C.R.System.Numerics.Multiprecision.BigRational}{operator*(const System.\-Numerics.\-Multiprecision.\-BigRational\&\-, const System.\-Numerics.\-Multiprecision.\-BigRational\&\-)}
& Returns the product of given \textbf{BigRational}
 value multiplied by another.

\\
\hyperlink{System.Numerics.Multiprecision.operator.plus.C.R.System.Numerics.Multiprecision.BigRational.C.R.System.Numerics.Multiprecision.BigRational}{operator+(const System.\-Numerics.\-Multiprecision.\-BigRational\&\-, const System.\-Numerics.\-Multiprecision.\-BigRational\&\-)}
& Returns the sum of given \textbf{BigRational}
 value added to another.

\\
\hyperlink{System.Numerics.Multiprecision.operator.minus.C.R.System.Numerics.Multiprecision.BigRational}{operator-(const System.\-Numerics.\-Multiprecision.\-BigRational\&\-)}
& Returns the negation of \textbf{BigRational}
.

\\
\hyperlink{System.Numerics.Multiprecision.operator.minus.C.R.System.Numerics.Multiprecision.BigRational.C.R.System.Numerics.Multiprecision.BigRational}{operator-(const System.\-Numerics.\-Multiprecision.\-BigRational\&\-, const System.\-Numerics.\-Multiprecision.\-BigRational\&\-)}
& Returns the difference of given \textbf{BigRational}
 value subtracted from another.

\\
\hyperlink{System.Numerics.Multiprecision.operator.divides.C.R.System.Numerics.Multiprecision.BigRational.C.R.System.Numerics.Multiprecision.BigRational}{operator/(const System.\-Numerics.\-Multiprecision.\-BigRational\&\-, const System.\-Numerics.\-Multiprecision.\-BigRational\&\-)}
& Returns the quotient when given \textbf{BigRational}
 is divided by another.

\\
\hyperlink{System.Numerics.Multiprecision.operator.less.C.R.System.Numerics.Multiprecision.BigRational.C.R.System.Numerics.Multiprecision.BigRational}{operator$<$\-(const System.\-Numerics.\-Multiprecision.\-BigRational\&\-, const System.\-Numerics.\-Multiprecision.\-BigRational\&\-)}
& Returns true if the first \textbf{BigRational}
 is less than the second \textbf{BigRational}
, false otherwise.

\\
\hyperlink{System.Numerics.Multiprecision.operator.shiftLeft.R.System.IO.OutputStream.C.R.System.Numerics.Multiprecision.BigRational}{operator$<$\-$<$\-(System.\-IO.\-OutputStream\&\-, const System.\-Numerics.\-Multiprecision.\-BigRational\&\-)}
& Puts the value of the given \textbf{BigRational}
 to the given output stream as string.

\\
\hyperlink{System.Numerics.Multiprecision.operator.equal.C.R.System.Numerics.Multiprecision.BigRational.C.R.System.Numerics.Multiprecision.BigRational}{operator==(const System.\-Numerics.\-Multiprecision.\-BigRational\&\-, const System.\-Numerics.\-Multiprecision.\-BigRational\&\-)}
& Returns true if the first \textbf{BigRational}
 is equal to the second \textbf{BigRational}
, false otherwise.

\\
\end{supertabular}

\end{flushleft}
\clearpage

\hypertarget{System.Numerics.Multiprecision.Abs.C.R.System.Numerics.Multiprecision.BigRational}{\subsubsection*{Abs(const System.Numerics.Multiprecision.BigRational\&) Function}}
\begin{flushleft}
Returns the absolute value of the given \hyperlink{System.Numerics.Multiprecision.BigRational}{BigRational} value.

\end{flushleft}
\subsubsection*{Syntax}
\texttt{public System.Numerics.Multiprecision.BigRational Abs(const System.Numerics.Multiprecision.BigRational\& x);}
\subsubsection*{Parameters}
\begin{flushleft}
\begin{supertabular}[l]{!{\raggedright}p{1.30721cm}!{\raggedright}p{8.55945cm}!{\raggedright}p{3.10422cm}}
\textbf{Name}
& \textbf{Type}
& \textbf{Description}
\\
\hline
x
& \hyperlink{System.Numerics.Multiprecision.BigRational}{const System.\-Numerics.\-Multiprecision.\-BigRational\&\-}
& \hyperlink{System.Numerics.Multiprecision.BigRational}{BigRational}.

\\
\end{supertabular}

\end{flushleft}
\subsubsection*{Returns}
\hyperlink{System.Numerics.Multiprecision.BigRational}{System.\-Numerics.\-Multiprecision.\-BigRational}
\begin{flushleft}
Returns the absolute value of the given \hyperlink{System.Numerics.Multiprecision.BigRational}{BigRational} value.

\end{flushleft}
\clearpage

\hypertarget{System.Numerics.Multiprecision.operator.times.C.R.System.Numerics.Multiprecision.BigRational.C.R.System.Numerics.Multiprecision.BigRational}{\subsubsection*{operator*(const System.Numerics.Multiprecision.BigRational\&, const System.Numerics.Multiprecision.BigRational\&) Function}}
\begin{flushleft}
Returns the product of given \hyperlink{System.Numerics.Multiprecision.BigRational}{BigRational} value multiplied by another.

\end{flushleft}
\subsubsection*{Syntax}
\texttt{public System.Numerics.Multiprecision.BigRational operator*(const System.Numerics.Multiprecision.BigRational\& left, const System.Numerics.Multiprecision.BigRational\& right);}
\subsubsection*{Parameters}
\begin{flushleft}
\begin{supertabular}[l]{!{\raggedright}p{1.30721cm}!{\raggedright}p{8.55945cm}!{\raggedright}p{3.66839cm}}
\textbf{Name}
& \textbf{Type}
& \textbf{Description}
\\
\hline
left
& \hyperlink{System.Numerics.Multiprecision.BigRational}{const System.\-Numerics.\-Multiprecision.\-BigRational\&\-}
& Left operand.

\\
right
& \hyperlink{System.Numerics.Multiprecision.BigRational}{const System.\-Numerics.\-Multiprecision.\-BigRational\&\-}
& Right operand.

\\
\end{supertabular}

\end{flushleft}
\subsubsection*{Returns}
\hyperlink{System.Numerics.Multiprecision.BigRational}{System.\-Numerics.\-Multiprecision.\-BigRational}
\begin{flushleft}
Returns the product of given \hyperlink{System.Numerics.Multiprecision.BigRational}{BigRational} value multiplied by another.

\end{flushleft}
\clearpage

\hypertarget{System.Numerics.Multiprecision.operator.plus.C.R.System.Numerics.Multiprecision.BigRational.C.R.System.Numerics.Multiprecision.BigRational}{\subsubsection*{operator+(const System.Numerics.Multiprecision.BigRational\&, const System.Numerics.Multiprecision.BigRational\&) Function}}
\begin{flushleft}
Returns the sum of given \hyperlink{System.Numerics.Multiprecision.BigRational}{BigRational} value added to another.

\end{flushleft}
\subsubsection*{Syntax}
\texttt{public System.Numerics.Multiprecision.BigRational operator+(const System.Numerics.Multiprecision.BigRational\& left, const System.Numerics.Multiprecision.BigRational\& right);}
\subsubsection*{Parameters}
\begin{flushleft}
\begin{supertabular}[l]{!{\raggedright}p{1.30721cm}!{\raggedright}p{8.55945cm}!{\raggedright}p{3.66839cm}}
\textbf{Name}
& \textbf{Type}
& \textbf{Description}
\\
\hline
left
& \hyperlink{System.Numerics.Multiprecision.BigRational}{const System.\-Numerics.\-Multiprecision.\-BigRational\&\-}
& Left operand.

\\
right
& \hyperlink{System.Numerics.Multiprecision.BigRational}{const System.\-Numerics.\-Multiprecision.\-BigRational\&\-}
& Right operand.

\\
\end{supertabular}

\end{flushleft}
\subsubsection*{Returns}
\hyperlink{System.Numerics.Multiprecision.BigRational}{System.\-Numerics.\-Multiprecision.\-BigRational}
\begin{flushleft}
Returns the sum of given \hyperlink{System.Numerics.Multiprecision.BigRational}{BigRational} value added to another.

\end{flushleft}
\clearpage

\hypertarget{System.Numerics.Multiprecision.operator.minus.C.R.System.Numerics.Multiprecision.BigRational}{\subsubsection*{operator-(const System.Numerics.Multiprecision.BigRational\&) Function}}
\begin{flushleft}
Returns the negation of \hyperlink{System.Numerics.Multiprecision.BigRational}{BigRational}.

\end{flushleft}
\subsubsection*{Syntax}
\texttt{public System.Numerics.Multiprecision.BigRational operator-(const System.Numerics.Multiprecision.BigRational\& x);}
\subsubsection*{Parameters}
\begin{flushleft}
\begin{supertabular}[l]{!{\raggedright}p{1.30721cm}!{\raggedright}p{8.55945cm}!{\raggedright}p{3.10422cm}}
\textbf{Name}
& \textbf{Type}
& \textbf{Description}
\\
\hline
x
& \hyperlink{System.Numerics.Multiprecision.BigRational}{const System.\-Numerics.\-Multiprecision.\-BigRational\&\-}
& \hyperlink{System.Numerics.Multiprecision.BigRational}{BigRational}.

\\
\end{supertabular}

\end{flushleft}
\subsubsection*{Returns}
\hyperlink{System.Numerics.Multiprecision.BigRational}{System.\-Numerics.\-Multiprecision.\-BigRational}
\begin{flushleft}
Returns the negation of \hyperlink{System.Numerics.Multiprecision.BigRational}{BigRational}.

\end{flushleft}
\clearpage

\hypertarget{System.Numerics.Multiprecision.operator.minus.C.R.System.Numerics.Multiprecision.BigRational.C.R.System.Numerics.Multiprecision.BigRational}{\subsubsection*{operator-(const System.Numerics.Multiprecision.BigRational\&, const System.Numerics.Multiprecision.BigRational\&) Function}}
\begin{flushleft}
Returns the difference of given \hyperlink{System.Numerics.Multiprecision.BigRational}{BigRational} value subtracted from another.

\end{flushleft}
\subsubsection*{Syntax}
\texttt{public System.Numerics.Multiprecision.BigRational operator-(const System.Numerics.Multiprecision.BigRational\& left, const System.Numerics.Multiprecision.BigRational\& right);}
\subsubsection*{Parameters}
\begin{flushleft}
\begin{supertabular}[l]{!{\raggedright}p{1.30721cm}!{\raggedright}p{8.55945cm}!{\raggedright}p{3.66839cm}}
\textbf{Name}
& \textbf{Type}
& \textbf{Description}
\\
\hline
left
& \hyperlink{System.Numerics.Multiprecision.BigRational}{const System.\-Numerics.\-Multiprecision.\-BigRational\&\-}
& Left operand.

\\
right
& \hyperlink{System.Numerics.Multiprecision.BigRational}{const System.\-Numerics.\-Multiprecision.\-BigRational\&\-}
& Right operand.

\\
\end{supertabular}

\end{flushleft}
\subsubsection*{Returns}
\hyperlink{System.Numerics.Multiprecision.BigRational}{System.\-Numerics.\-Multiprecision.\-BigRational}
\begin{flushleft}
Returns the difference of given \hyperlink{System.Numerics.Multiprecision.BigRational}{BigRational} value subtracted from another.

\end{flushleft}
\clearpage

\hypertarget{System.Numerics.Multiprecision.operator.divides.C.R.System.Numerics.Multiprecision.BigRational.C.R.System.Numerics.Multiprecision.BigRational}{\subsubsection*{operator/(const System.Numerics.Multiprecision.BigRational\&, const System.Numerics.Multiprecision.BigRational\&) Function}}
\begin{flushleft}
Returns the quotient when given \hyperlink{System.Numerics.Multiprecision.BigRational}{BigRational} is divided by another.

\end{flushleft}
\subsubsection*{Syntax}
\texttt{public System.Numerics.Multiprecision.BigRational operator/(const System.Numerics.Multiprecision.BigRational\& left, const System.Numerics.Multiprecision.BigRational\& right);}
\subsubsection*{Parameters}
\begin{flushleft}
\begin{supertabular}[l]{!{\raggedright}p{1.30721cm}!{\raggedright}p{8.55945cm}!{\raggedright}p{2.65538cm}}
\textbf{Name}
& \textbf{Type}
& \textbf{Description}
\\
\hline
left
& \hyperlink{System.Numerics.Multiprecision.BigRational}{const System.\-Numerics.\-Multiprecision.\-BigRational\&\-}
& Divisor.

\\
right
& \hyperlink{System.Numerics.Multiprecision.BigRational}{const System.\-Numerics.\-Multiprecision.\-BigRational\&\-}
& Dividend.

\\
\end{supertabular}

\end{flushleft}
\subsubsection*{Returns}
\hyperlink{System.Numerics.Multiprecision.BigRational}{System.\-Numerics.\-Multiprecision.\-BigRational}
\begin{flushleft}
Returns the quotient when given \hyperlink{System.Numerics.Multiprecision.BigRational}{BigRational} is divided by another.

\end{flushleft}
\clearpage

\hypertarget{System.Numerics.Multiprecision.operator.less.C.R.System.Numerics.Multiprecision.BigRational.C.R.System.Numerics.Multiprecision.BigRational}{\subsubsection*{operator$<$(const System.Numerics.Multiprecision.BigRational\&, const System.Numerics.Multiprecision.BigRational\&) Function}}
\begin{flushleft}
Returns true if the first \hyperlink{System.Numerics.Multiprecision.BigRational}{BigRational} is less than the second \hyperlink{System.Numerics.Multiprecision.BigRational}{BigRational}, false otherwise.

\end{flushleft}
\subsubsection*{Syntax}
\texttt{public bool operator$<$(const System.Numerics.Multiprecision.BigRational\& left, const System.Numerics.Multiprecision.BigRational\& right);}
\subsubsection*{Parameters}
\begin{flushleft}
\begin{supertabular}[l]{!{\raggedright}p{1.30721cm}!{\raggedright}p{8.55945cm}!{\raggedright}p{3.66839cm}}
\textbf{Name}
& \textbf{Type}
& \textbf{Description}
\\
\hline
left
& \hyperlink{System.Numerics.Multiprecision.BigRational}{const System.\-Numerics.\-Multiprecision.\-BigRational\&\-}
& Left operand.

\\
right
& \hyperlink{System.Numerics.Multiprecision.BigRational}{const System.\-Numerics.\-Multiprecision.\-BigRational\&\-}
& Right operand.

\\
\end{supertabular}

\end{flushleft}
\subsubsection*{Returns}bool
\begin{flushleft}
Returns true if the first \hyperlink{System.Numerics.Multiprecision.BigRational}{BigRational} is less than the second \hyperlink{System.Numerics.Multiprecision.BigRational}{BigRational}, false otherwise.

\end{flushleft}
\clearpage

\hypertarget{System.Numerics.Multiprecision.operator.shiftLeft.R.System.IO.OutputStream.C.R.System.Numerics.Multiprecision.BigRational}{\subsubsection*{operator$<$$<$(System.IO.OutputStream\&, const System.Numerics.Multiprecision.BigRational\&) Function}}
\begin{flushleft}
Puts the value of the given \hyperlink{System.Numerics.Multiprecision.BigRational}{BigRational} to the given output stream as string.

\end{flushleft}
\subsubsection*{Syntax}
\texttt{public System.IO.OutputStream\& operator$<$$<$(System.IO.OutputStream\& s, const System.Numerics.Multiprecision.BigRational\& x);}
\subsubsection*{Parameters}
\begin{flushleft}
\begin{supertabular}[l]{!{\raggedright}p{1.30721cm}!{\raggedright}p{8.55945cm}!{\raggedright}p{4.93333cm}}
\textbf{Name}
& \textbf{Type}
& \textbf{Description}
\\
\hline
s
& System.\-IO.\-OutputStream\&\-
& An output stream.

\\
x
& \hyperlink{System.Numerics.Multiprecision.BigRational}{const System.\-Numerics.\-Multiprecision.\-BigRational\&\-}
& A \hyperlink{System.Numerics.Multiprecision.BigRational}{BigRational} value.

\\
\end{supertabular}

\end{flushleft}
\subsubsection*{Returns}System.\-IO.\-OutputStream\&\-
\begin{flushleft}
Returns a reference to the output stream.

\end{flushleft}
\clearpage

\hypertarget{System.Numerics.Multiprecision.operator.equal.C.R.System.Numerics.Multiprecision.BigRational.C.R.System.Numerics.Multiprecision.BigRational}{\subsubsection*{operator==(const System.Numerics.Multiprecision.BigRational\&, const System.Numerics.Multiprecision.BigRational\&) Function}}
\begin{flushleft}
Returns true if the first \hyperlink{System.Numerics.Multiprecision.BigRational}{BigRational} is equal to the second \hyperlink{System.Numerics.Multiprecision.BigRational}{BigRational}, false otherwise.

\end{flushleft}
\subsubsection*{Syntax}
\texttt{public bool operator==(const System.Numerics.Multiprecision.BigRational\& left, const System.Numerics.Multiprecision.BigRational\& right);}
\subsubsection*{Parameters}
\begin{flushleft}
\begin{supertabular}[l]{!{\raggedright}p{1.30721cm}!{\raggedright}p{8.55945cm}!{\raggedright}p{3.66839cm}}
\textbf{Name}
& \textbf{Type}
& \textbf{Description}
\\
\hline
left
& \hyperlink{System.Numerics.Multiprecision.BigRational}{const System.\-Numerics.\-Multiprecision.\-BigRational\&\-}
& Left operand.

\\
right
& \hyperlink{System.Numerics.Multiprecision.BigRational}{const System.\-Numerics.\-Multiprecision.\-BigRational\&\-}
& Right operand.

\\
\end{supertabular}

\end{flushleft}
\subsubsection*{Returns}bool
\begin{flushleft}
Returns true if the first \hyperlink{System.Numerics.Multiprecision.BigRational}{BigRational} is equal to the second \hyperlink{System.Numerics.Multiprecision.BigRational}{BigRational}, false otherwise.

\end{flushleft}
\clearpage

\hypertarget{System.Numerics.Multiprecision.Precision}{\section{Precision Class}}
\begin{flushleft}
Represents a precision of given number of digits.

\end{flushleft}
\subsection*{Syntax}\texttt{public class Precision;}

\subsection{Member Functions}
\begin{flushleft}
\begin{supertabular}[l]{!{\raggedright}p{7.575cm}!{\raggedright}p{7.575cm}}
\textbf{Member Function}
& \textbf{Description}
\\
\hline
\hyperlink{System.Numerics.Multiprecision.Precision.constructor.P.System.Numerics.Multiprecision.Precision}{Precision()}
& Default constructor. Initializes the precision with zero digits.

\\
\hyperlink{System.Numerics.Multiprecision.Precision.constructor.P.System.Numerics.Multiprecision.Precision.C.R.System.Numerics.Multiprecision.Precision}{Precision(const System.\-Numerics.\-Multiprecision.\-Precision\&\-)}
& Copy constructor.

\\
\hyperlink{System.Numerics.Multiprecision.Precision.operator.assign.P.System.Numerics.Multiprecision.Precision.C.R.System.Numerics.Multiprecision.Precision}{operator=(const System.\-Numerics.\-Multiprecision.\-Precision\&\-)}
& Copy assignment.

\\
\hyperlink{System.Numerics.Multiprecision.Precision.constructor.P.System.Numerics.Multiprecision.Precision.RR.System.Numerics.Multiprecision.Precision}{Precision(System.\-Numerics.\-Multiprecision.\-Precision\&\-\&\-)}
& Move constructor.

\\
\hyperlink{System.Numerics.Multiprecision.Precision.operator.assign.P.System.Numerics.Multiprecision.Precision.RR.System.Numerics.Multiprecision.Precision}{operator=(System.\-Numerics.\-Multiprecision.\-Precision\&\-\&\-)}
& Move assignment.

\\
\hyperlink{System.Numerics.Multiprecision.Precision.constructor.P.System.Numerics.Multiprecision.Precision.uint}{Precision(uint)}
& Constructor. Initializes the precision with given number of digits.

\\
\hyperlink{System.Numerics.Multiprecision.Precision.operator\_uint.C.P.System.Numerics.Multiprecision.Precision}{operator\_\-uint() const}
& Returns the number of digits of this precision.

\\
\end{supertabular}

\end{flushleft}
\clearpage

\hypertarget{System.Numerics.Multiprecision.Precision.constructor.P.System.Numerics.Multiprecision.Precision}{\subsubsection*{Precision() Member Function}}
\begin{flushleft}
Default constructor. Initializes the precision with zero digits.

\end{flushleft}
\subsubsection*{Syntax}\texttt{public Precision();}
\clearpage

\hypertarget{System.Numerics.Multiprecision.Precision.constructor.P.System.Numerics.Multiprecision.Precision.C.R.System.Numerics.Multiprecision.Precision}{\subsubsection*{Precision(const System.Numerics.Multiprecision.Precision\&) Member Function}}\begin{flushleft}
Copy constructor.

\end{flushleft}
\subsubsection*{Syntax}
\texttt{public Precision(const System.Numerics.Multiprecision.Precision\& that);}
\subsubsection*{Parameters}
\begin{flushleft}
\begin{supertabular}[l]{!{\raggedright}p{1.30721cm}!{\raggedright}p{8.55945cm}!{\raggedright}p{4.55908cm}}
\textbf{Name}
& \textbf{Type}
& \textbf{Description}
\\
\hline
that
& \hyperlink{System.Numerics.Multiprecision.Precision}{const System.\-Numerics.\-Multiprecision.\-Precision\&\-}
& Argument to copy.

\\
\end{supertabular}

\end{flushleft}
\clearpage

\hypertarget{System.Numerics.Multiprecision.Precision.operator.assign.P.System.Numerics.Multiprecision.Precision.C.R.System.Numerics.Multiprecision.Precision}{\subsubsection*{operator=(const System.Numerics.Multiprecision.Precision\&) Member Function}}\begin{flushleft}
Copy assignment.

\end{flushleft}
\subsubsection*{Syntax}
\texttt{public void operator=(const System.Numerics.Multiprecision.Precision\& that);}
\subsubsection*{Parameters}
\begin{flushleft}
\begin{supertabular}[l]{!{\raggedright}p{1.30721cm}!{\raggedright}p{8.55945cm}!{\raggedright}p{4.89046cm}}
\textbf{Name}
& \textbf{Type}
& \textbf{Description}
\\
\hline
that
& \hyperlink{System.Numerics.Multiprecision.Precision}{const System.\-Numerics.\-Multiprecision.\-Precision\&\-}
& Argument to assign.

\\
\end{supertabular}

\end{flushleft}
\clearpage

\hypertarget{System.Numerics.Multiprecision.Precision.constructor.P.System.Numerics.Multiprecision.Precision.RR.System.Numerics.Multiprecision.Precision}{\subsubsection*{Precision(System.Numerics.Multiprecision.Precision\&\&) Member Function}}\begin{flushleft}
Move constructor.

\end{flushleft}
\subsubsection*{Syntax}
\texttt{public Precision(System.Numerics.Multiprecision.Precision\&\& that);}
\subsubsection*{Parameters}
\begin{flushleft}
\begin{supertabular}[l]{!{\raggedright}p{1.30721cm}!{\raggedright}p{8.55945cm}!{\raggedright}p{4.93333cm}}
\textbf{Name}
& \textbf{Type}
& \textbf{Description}
\\
\hline
that
& \hyperlink{System.Numerics.Multiprecision.Precision}{System.\-Numerics.\-Multiprecision.\-Precision\&\-\&\-}
& Argument to move from.

\\
\end{supertabular}

\end{flushleft}
\clearpage

\hypertarget{System.Numerics.Multiprecision.Precision.operator.assign.P.System.Numerics.Multiprecision.Precision.RR.System.Numerics.Multiprecision.Precision}{\subsubsection*{operator=(System.Numerics.Multiprecision.Precision\&\&) Member Function}}\begin{flushleft}
Move assignment.

\end{flushleft}
\subsubsection*{Syntax}
\texttt{public void operator=(System.Numerics.Multiprecision.Precision\&\& that);}
\subsubsection*{Parameters}
\begin{flushleft}
\begin{supertabular}[l]{!{\raggedright}p{1.30721cm}!{\raggedright}p{8.55945cm}!{\raggedright}p{4.93333cm}}
\textbf{Name}
& \textbf{Type}
& \textbf{Description}
\\
\hline
that
& \hyperlink{System.Numerics.Multiprecision.Precision}{System.\-Numerics.\-Multiprecision.\-Precision\&\-\&\-}
& Argument to assign from.

\\
\end{supertabular}

\end{flushleft}
\clearpage

\hypertarget{System.Numerics.Multiprecision.Precision.constructor.P.System.Numerics.Multiprecision.Precision.uint}{\subsubsection*{Precision(uint) Member Function}}
\begin{flushleft}
Constructor. Initializes the precision with given number of digits.

\end{flushleft}
\subsubsection*{Syntax}\texttt{public Precision(uint prec\_);}

\subsubsection*{Parameters}
\begin{flushleft}
\begin{supertabular}[l]{lll}
\textbf{Name}
& \textbf{Type}
& \textbf{Description}
\\
\hline
prec\_
& uint
& Number of digits.

\\
\end{supertabular}

\end{flushleft}
\clearpage

\hypertarget{System.Numerics.Multiprecision.Precision.operator\_uint.C.P.System.Numerics.Multiprecision.Precision}{\subsubsection*{operator\_uint() const Member Function}}
\begin{flushleft}
Returns the number of digits of this precision.

\end{flushleft}
\subsubsection*{Syntax}\texttt{public uint operator\_uint() const;}

\subsubsection*{Returns}uint
\begin{flushleft}
Returns the number of digits of this precision.

\end{flushleft}
\clearpage
\subchapter{Functions}
\begin{flushleft}
\begin{supertabular}[l]{ll}
\textbf{Function}
& \textbf{Description}
\\
\hline
\end{supertabular}

\end{flushleft}
\clearpage
\clearpage
\end{document}


