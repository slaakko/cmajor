\documentclass[oneside, a4paper, 11pt]{article}
\usepackage[utf8]{inputenc}
\usepackage{a4wide}
\usepackage{graphics}
\usepackage{url}
\usepackage[colorlinks=true,linkcolor=blue]{hyperref}
\usepackage{listings}
\lstdefinelanguage{Cmajor}{morekeywords={assert, bool, true, false, sbyte, byte, short, ushort, int, uint, long, ulong,
float, double, char, void, enum, cast, namespace, using, static, extern, is, explicit, delegate, inline, cdecl, nothrow,
public, protected, private, internal, virtual, abstract, override, suppress, default, operator, class, return, if, else,
switch, case, default, while, do, for, break, continue, goto, typedef, typename, const, null, this, base,
construct, destroy, new, delete, sizeof, try, catch, throw, concept, where, axiom, and, or, not},sensitive=true,morecomment=[l]{//},morecomment=[s]{/*}{*/},morestring=[b]\",morestring=[b]',}
\lstloadlanguages{Cmajor}
\lstset{language=Cmajor,showstringspaces=false,breaklines=true,basicstyle=\small,frameround=fttt}

\begin{document}

\title{Cmajor Concepts}
\author{Seppo Laakko}
\date{\today}
\maketitle

\section{Introduction}

In Cmajor programming language functions and classes can be parameterized with types.
A set of syntactic and semantic requirements for the actual types that can correctly be
substituted for a formal type parameter in such a parameterized function or class is
called a \emph{concept}.

Concepts as a feature of a programming language enable the compiler to give better error
messages when a substituted type does not satisfy syntactic requirements.
They also enable the compiler to check that a formal type parameter
constrained by a concept has all the necessary operations when it compiles a body of a
parameterized function.

In this tutorial we will look over the tools that Cmajor programming language has to offer
with respect to concepts.

\section{Signature Constraints}

In Cmajor programming language functions and classes can be parameterized with types.
A parameterized function is called a \emph{function template} and
a parameterized class is called a \emph{class template}.
Cmajor follows the same naming convention as C++ in this respect.
Figure \ref{fig:funtemplate} shows an example of a function template in Cmajor.

\begin{figure}[htb]\caption{Function Template}\label{fig:funtemplate}
\begin{lstlisting}[frame=trBL]
T Add<T>(T a, T b)
{
    return a + b;
}
\end{lstlisting}
\end{figure}

In case of Add function template it is natural to ask, what the type parameter T can be.
Surely one must be able to pass object of type T as a parameter and return it by value.
This requires that one can copy objects of type T, i.e. T must be \emph{copy constructible}.
One must also be able to add together two objects of type T.
For now we can say that T must be additive.
In Cmajor those requirements for template parameter types can be collected together and
form a \emph{concept}. Figure \ref{fig:additive} shows the definition of this additive concept.

\begin{figure}[htb]\caption{Additive Concept}\label{fig:additive}
\begin{lstlisting}[frame=trBL]
concept Additive<T>
{
    T(const T&);
    T operator+(T, T);
}
\end{lstlisting}
\end{figure}

The requirement that T must be copy constructible is expressed by \verb|T(const T&)| and
the requirement that one must be able to add two objects of T is expressed by \verb|T operator+(T, T)|.
A synonym for a syntactic requirement like the preceding is a \emph{constraint},
because it constrains what type T can be.
Requirements that one must be able to do certain operations with objects of type T are called
\emph{signature constraints} in Cmajor.

Now we can rewrite the Add function template with Additive concept requirements to form a
\emph{constrained function template}. Figure \ref{fig:constrainedfuntemplate} shows the
rewritten Add function template.

\begin{figure}[htb]\caption{Constrained Function Template}\label{fig:constrainedfuntemplate}
\begin{lstlisting}[frame=trBL]
T Add<T>(T a, T b) where T is Additive
{
    return a + b;
}
\end{lstlisting}
\end{figure}

The \textbf{where} clause states that the template parameter T must satisfy the requirements of Additive concept.

\begin{figure}[htb]\caption{Additive Test}\label{fig:additivetest}
\begin{lstlisting}[frame=trBL]
using System;

concept Additive<T>
{
    T(const T&);
    T operator+(T, T);
}

T Add<T>(T a, T b) where T is Additive
{
    return a + b;
}

void main()
{
    int a = 1;
    int b = 2;
    int si = Add(a, b);             //  ok
    bool sb = Add(true, false);     //  error
}
\end{lstlisting}
\end{figure}

Let us now see what the Cmajor compiler thinks about the preceding definitions.
When compiling the test program in figure \ref{fig:additivetest} the
Add function call with two integer arguments compiles fine,
because surely one can copy integers and add them together.
However calling the Add function with two Boolean values fails with an error message like this:

\begin{verbatim}
Error: overload resolution failed: 'Add(bool, bool)' not found, or there are no
acceptable conversions for all argument types. 1 viable functions examined:
type 'bool' does not fulfill the requirements of concept Additive<T> because:
function signature 'operator+(bool, bool)' not found (file 'C:/Programming/cmajo
rbin/doc/source/concepts/examples/additive/additive.cm', line 19):
    bool sb = Add(true, false); //  error
              ^^^^^^^^^^^^^^^^
\end{verbatim}

It means that the Add function template has not been instantiated for \textbf{bool} type,
because \textbf{bool} does not conform to Additive concept. It does not conform,
because there is no \verb|operator+| member function in Boolean type and no
free \verb|operator+| function that takes Boolean parameters.
Because of these facts, the overload resolution fails with the \verb|Add(bool, bool)| function
not found.

\section{Associated Types and Typename Constraints}

Techniques used in generic programming are somewhat different to traditional object oriented programming.
In generic programming the types often do not form hierarchies, but they are otherwise related:
one type is \emph{associated} with another. An associated type is defined inside another type or
there is a \textbf{typedef} inside a type that names the associated type.

For example in List class template the type contained in list is an associated type of the List type,
and it is named ValueType (Fig. \ref{fig:list}.)

\begin{figure}[htb]\caption{List Class Template}\label{fig:list}
\begin{lstlisting}[frame=trBL]
class List<T> where T is Semiregular
{
    public typedef T ValueType;
//  ...
}
\end{lstlisting}
\end{figure}

To express that a type has an associated type in relation to Cmajor concepts, one
can define a \emph{typename constraint} in a concept definition.
For example in the definition of Container concept there is a requirement that
a type conforming to Container concept has an associated type called ValueType
(Fig. \ref{fig:container}.) \footnote{The syntax of typename constraint has
changed since version 0.9.1.}

\begin{figure}[htb]\caption{Container Concept}\label{fig:container}
\begin{lstlisting}[frame=trBL]
concept Container<T>
{
    typename T.ValueType;
//  ...
}
\end{lstlisting}
\end{figure}

The List class template has the associated type ValueType in form of \textbf{typedef} definition,
so it conforms to Container concept in this respect.

\section{Embedded Constraints}

A concept definition can state additional requirements for the template type parameter or
for an associated type in form of an \emph{embedded constraint}.

For example Semiregular concept (Fig. \ref{fig:semiregular}) has an embedded constraint that
a semiregular type is DefaultConstructible, CopyConstructible, Destructible and Assignable,
where DefaultConstructible, CopyConstructible, Destructible and Assignable are concepts defined
elsewhere.

\begin{figure}[htb]\caption{Semiregular Concept}\label{fig:semiregular}
\begin{lstlisting}[frame=trBL]
concept Semiregular<T>
{
    where T is DefaultConstructible and T is CopyConstructible and T is Destructible and T is Assignable;
}
\end{lstlisting}
\end{figure}

\section{Multiparameter Constraints}

Sometimes a concept involves two or more types.
For example Assignable$<$T, U$>$ (Fig. \ref{fig:multiconcept})
concept states the requirement that type T implements an assignment operator that
takes type U parameter.

\begin{figure}[htb]\caption{Multiparameter Concept}\label{fig:multiconcept}
\begin{lstlisting}[frame=trBL]
concept Assignable<T, U>
{
    void operator=(const U&);
}
\end{lstlisting}
\end{figure}

\section{Concept Hierarchies}

In generic programming classes do not often form hierarchies, while concepts often do.
When a class derives from another class, it inherits the members of the base class and it can
\emph{override} virtual functions defined in the base class.
With respect to concepts, a concept can \emph{refine} another concept.
When concept refines another, it can define additional requirements and \emph{redefine}
requirements defined in the refined concept for parameterized types and for types
associated with the parameterized types.

\subsection{Iterator Hierarchy}

The Cmajor System library borrows and adapts the implementation of many algorithms, containers and concepts from
the C++ Standard Template Library \cite{STL}, among other things iterator concepts.
Iterators in Cmajor form a refinement hierarchy shown in figure \ref{fig:iteratorhierarchy}.

\begin{figure}[htb]
\caption{Iterator Hierarchy}
\label{fig:iteratorhierarchy}
\vspace{0.5cm}
\includegraphics{concepts2}
\end{figure}

\begin{figure}[htb]\caption{Trivial, Input and Output Iterators}
\label{fig:trivioiterators}
\begin{lstlisting}[frame=trBL]
concept TrivialIterator<T> where T is Semiregular
{
    typename T.ValueType;
    where T.ValueType is Semiregular;
    typename T.ReferenceType;
    where T.ReferenceType is T.ValueType&;
    T.ReferenceType operator*();
    typename T.PointerType;
    where T.PointerType is T.ValueType*;
    T.PointerType operator->();
}

concept OutputIterator<T>: TrivialIterator<T>
{
    T& operator++();
}

concept InputIterator<T>: TrivialIterator<T>
{
    T& operator++();
    where T is Regular;
}
\end{lstlisting}
\end{figure}

The TrivialIterator (Fig. \ref{fig:trivioiterators}) is a concept that collects
the common typename and signature requirements for all iterator types.
There is no class in Cmajor that is a model of bare trivial iterator concept.
Every iterator must define associated types ValueType, ReferenceType and
PointerType: these are typename constraints.
ValueType must be semiregular: this is an embedded constraint.
Every iterator must also define \verb|operator->| function that
returns a PointerType: this is a signature constraint.

The OutputIterator concept refines the TrivialIterator concept by
adding \verb|operator++| signature constraint.
We can say that an output iterator is a trivial iterator that can
be incremented.
Likewise an input iterator is a trivial iterator that can be incremented and
compared for equality and inequality.
That is the requirement added by the Regular concept.

\begin{figure}[htb]\caption{Forward, Bidirectional and Random Access Iterators}\label{fig:fwdbidraiterators}
\begin{lstlisting}[frame=trBL]
concept ForwardIterator<T>: InputIterator<T>
{
    where T is OutputIterator;
}

concept BidirectionalIterator<T>: ForwardIterator<T>
{
    T& operator--();
}

concept RandomAccessIterator<T>: BidirectionalIterator<T>
{
    T.ReferenceType operator[](int index);
    T operator+(T, int);
    T operator+(int, T);
    T operator-(T, int);
    int operator-(T, T);
    where T is LessThanComparable;
}
\end{lstlisting}
\end{figure}

A forward iterator (Fig. \ref{fig:fwdbidraiterators}) refines in a sense both input and output iterators, but Cmajor lacks
multiple refinement, so the requirement that forward iterator is an output iterator is
expressed as an embedded constraint.
Additionally, a forward iterator is a multipass input iterator, but this requirement cannot
be expressed in Cmajor syntax.

Some algorithms require an iterator that can walk both forwards and backwards, but cannot ``jump'':
this is what a bidirectional iterator can do.

Finally some algorithms require pointer-like operations: indexing, moving arbitrary offsets forwards and backwards,
computing the distance between two iterators and comparing two iterators with less than, greater than,
less or equal to and greater or equal to relations.
This is what a random access iterator can do.

In analogy to class inheritance where a derived class can be used in place of the base class
(the \emph{derived} IS-A \emph{base}), the same applies to concept refinement:
a type that conforms to a more constrained concept, can be used in context where
a type that conforms only to a less constrained concept is required.
Thus a random access iterator IS-A bidirectional iterator IS-A forward iterator.

\subsection{Container Hierarchy}

The container hierarchy goes parallel with the iterator hierarchy.

\begin{figure}[htb]\caption{Container Concept}\label{fig:containerconcepts}
\begin{lstlisting}[frame=trBL]
concept Container<T> where T is Semiregular
{
    typename T.ValueType;
    typename T.Iterator;
    typename T.ConstIterator;
    where T.Iterator is TrivialIterator and T.ConstIterator is TrivialIterator and T.ValueType is T.Iterator.ValueType;
    T.Iterator T.Begin();
    T.ConstIterator T.CBegin();
    T.Iterator T.End();
    T.ConstIterator T.CEnd();
    int T.Count();
    bool T.IsEmpty();
}
\end{lstlisting}
\end{figure}

The container concept (Fig. \ref{fig:containerconcepts}) defines some common requirements
for all container types.
It requires that the container has associated iterator and constant iterator types
that conform to trivial iterator concept.

It also links the value type of the container to the value type of the iterator by
requiring them to be the same type.
It finally requires that the container has member functions that return
iterator types and some other member functions common to all containers.

\begin{figure}[htb]\caption{Forward Container Concept}\label{fig:fwdcontainer}
\begin{lstlisting}[frame=trBL]
concept ForwardContainer<T>: Container<T>
{
    where T.Iterator is ForwardIterator and T.ConstIterator is ForwardIterator;
}
\end{lstlisting}
\end{figure}

A forward container concept (Fig. \ref{fig:fwdcontainer})
redefines the requirement in container concept for the iterator type to be not just
a trivial iterator but a forward iterator.
Likewise bidirectional container refines forward container and redefines the requirement
for the iterator type to be not just a forward iterator but a bidirectional iterator,
and random access container refines bidirectional container and redefines the iterator
type to be not just a bidirectional iterator but a random access iterator.

\section{Built-in Concepts}

There are some concepts built into the Cmajor language.

The \textbf{Same$<$T, U$>$} concept sets a requirement that T and U are exactly
the same type. For example if A is \textbf{const int\&}, B is \textbf{const int\&} and
C is \textbf{int\&}, Same$<$A, B$>$ is true, but Same$<$A, C$>$ is false.

The \textbf{Derived$<$T, U$>$} concept sets a requirement that T and U are class types and
T is derived from U.

The \textbf{Convertible$<$T, U$>$} concept sets a requirement that type T is implicitly convertible
to U. For example \textbf{int} is implicitly convertible to \textbf{double}, but not vice versa.

The \textbf{ExplicitlyConvertible$<$T, U$>$} concept sets a requirement that T can be implicitly or
explicitly converted to U. For example \textbf{double} is explicitly convertible to \textbf{int}
(by using a cast), but \textbf{void*} is not explicitly convertible to \textbf{int}.

The \textbf{Common$<$T, U$>$} concept sets a requirement that T and U have a common type that both are
convertible to. The concept exposes the common type as a typedef CommonType.
For example \textbf{Common$<$int, double$>$} is true and their common type is \textbf{double},
but \textbf{Common$<$void*, int$>$} is false.

\section{Axioms}

So far the requirements for parameterized types have been purely syntactic ---
the type must have certain operation or it must have an associated type of the given name.

In Cmajor one can also express semantic requirements for parameterized types
in form of \emph{axioms} (Fig. \ref{fig:equalitycomparable}.)

\begin{figure}[htb]\caption{Equality Comparable Concept}\label{fig:equalitycomparable}
\begin{lstlisting}[frame=trBL]
concept EqualityComparable<T>
{
    bool operator==(T, T);
    axiom equal(T a, T b) { a == b <=> eq(a, b); }
    axiom reflexive(T a) { a == a; }
    axiom symmetric(T a, T b) { a == b => b == a; }
    axiom transitive(T a, T b, T c) { a == b && b == c => a == c; }
    axiom notEqualTo(T a, T b) { a != b <=> !(a == b); }
}
\end{lstlisting}
\end{figure}

Axioms are not processed by the Cmajor compiler in any other way than
parsing their syntax.
Axioms are documentation for the programmer so the programmer can reason
how a type that models a concept behaves.

\section{Using Concepts}

Now we have roughly walked through the tools Cmajor has to offer for defining concepts.
So how do we use them?

\subsection{Generic Algorithm}

\begin{figure}[htb]\caption{Copy Algorithm}\label{fig:copy}
\begin{lstlisting}[frame=trBL]
public O Copy<I, O>(I begin, I end, O to)
    where I is InputIterator and O is OutputIterator and
    Assignable<O.ValueType, I.ValueType>
{
    while (begin != end)
    {
        *to = *begin;
        ++begin;
        ++to;
    }
}
\end{lstlisting}
\end{figure}

Let us take a look at the copy algorithm that copies an input sequence to an output sequence.
The Copy function has a \textbf{where} clause that lists the requirements for the template parameters I and O
combining them with \textbf{and} connective (Fig. \ref{fig:copy}.)

First Copy requires that I template parameter is a model of an input iterator,
because we need it to iterate through the input sequence and
to test whether we have reached the end of the input sequence. That is what input iterators can do.
Then it requires that O template parameter is a model of an output iterator, because we need it to
generate the output sequence. That is what output iterators can do.
Finally we need to assign an object that is of type ValueType associated with input iterator type (I.ValueType
\footnote{Associated types are referred to using the \textbf{.} notation in Cmajor.}) to an
object that is of type ValueType associated with output iterator type (O.ValueType).
That is how we copy values from input sequence to output sequence.
It takes a form of a multiparameter Assignable$<$T, U$>$ constraint.

You can pass random access iterators to the Copy function (because they are input and output iterators),
but the algorithm requires only input and output iterators. This makes it generic.

\subsection{Container Concepts and Overloading}

\begin{figure}[htb]\caption{Sorting a Container}\label{fig:sort}
\begin{lstlisting}[frame=trBL]
public void Sort<C>(C& c)
    where C is RandomAccessContainer and C.Iterator.ValueType is TotallyOrdered
{
    // ...
}

public void Sort<C>(C& c)
    where C is ForwardContainer and C.Iterator.ValueType is TotallyOrdered
{
    List<C.ValueType> list;
    Copy(c.CBegin(), c.CEnd(), BackInserter(list));
    Sort(list);
    Copy(list.CBegin(), list.CEnd(), c.Begin());
}
\end{lstlisting}
\end{figure}

Let us take a look at the sort algorithm for containers (Fig. \ref{fig:sort}.)
It has two overloads. Both take a container parameter, but one constrains the container
to be a random access container, and the other constrains the container
to be just a forward container.

Suppose you have a \verb|List<int> list| that need to be sorted
\footnote{In spite of its name, System.List is functionally equal to
STL's std::vector class template.}.
List is a model of a random access container, because the iterators it provides are
random access iterators.

Now you call \verb|Sort(list)|. The compiler checks the sort overloads.
First it finds that sort overload for a random access container is a match,
because List is a random access container. Then it finds that the sort overload for
a forward container matches also, because the constraint check for a forward
container succeeds, when the container is actually a random access container.
However the sort for a random access container is a better match,
because random access container concept refines forward container concept,
so the sort for a random access container gets instantiated and called.

Suppose now that you have a \verb|ForwardList<int> fwdlist|.
System.Collections.ForwardList is a model of a forward container
\footnote{System.Collections.ForwardList is a singly linked list.}.
Now you call \verb|Sort(fwdlist)|. Again the compiler checks the sort overloads,
but this time it rejects the sort for a random access container, because
ForwardList fails the constraint check for this overload.
So the one that gets instantiated and called this time is the sort for a forward
container --- a slower version of the sort algorithm.

\subsection{Iterator Concepts and Overloading}

First take a look at overloads of Distance function that
returns a distance between two iterators.
It has two overloads.

\begin{figure}[htb]\caption{Distance for Forward Iterators}\label{fig:distanceslow}
\begin{lstlisting}[frame=trBL]
public nothrow int Distance<I>(I first, I last)
    where I is ForwardItetor
{
    int distance = 0;
    while (first != last)
    {
        ++first;
        ++distance;
    }
    return distance;
}
\end{lstlisting}
\end{figure}

\begin{figure}[htb]\caption{Distance for Random Access Iterators}\label{fig:distancefast}
\begin{lstlisting}[frame=trBL]
public nothrow inline int Distance<I>(I first, I last)
    where I is RandomAccessIterator
{
    return last - first;
}
\end{lstlisting}
\end{figure}

The first one is a slow version for a forward iterator (Fig. \ref{fig:distanceslow}.)
It counts the number of steps it takes to reach iterator \emph{last} from iterator \emph{first}.

The second one is a fast version  for a random access iterator (Fig. \ref{fig:distancefast}.)
It simply returns the difference of \emph{last} and \emph{first} iterators.
It can do this because random access iterators support this operation.

\begin{figure}[htb]\caption{Next for Forward Iterators}\label{fig:nextslow}
\begin{lstlisting}[frame=trBL]
public nothrow I Next<I>(I i, int n)
    where I is ForwardIterator
{
    #assert(n >= 0);
    while (n > 0)
    {
        ++i;
        --n;
    }
    return i;
}
\end{lstlisting}
\end{figure}

\begin{figure}[htb]\caption{Next for Random Access Iterators}\label{fig:nextfast}
\begin{lstlisting}[frame=trBL]
public nothrow inline I Next<I>(I i, int n)
    where I is RandomAccessIterator
{
    return i + n;
}
\end{lstlisting}
\end{figure}

Then look at the overloads of a Next function, that
returns an iterator advanced the specified number of steps.
Again two overloads, one for forward iterators and one for random access
iterators.
The first version (Fig. \ref{fig:nextslow}) increments a forward iterator given
number of steps.
The second version (Fig. \ref{fig:nextfast}) simply returns a sum of a random access iterator
and an offset.

\clearpage
\begin{figure}[htb]\caption{Lower Bound}\label{fig:lowerbound}
\begin{lstlisting}[frame=trBL]
public nothrow I LowerBound<I, T>(I first, I last, const T& value) where I is ForwardIterator and TotallyOrdered<T, I.ValueType>
{
    int len = Distance(first, last);
    while (len > 0)
    {
        int half = len >> 1;
        I middle = Next(first, half);
        if (value > *middle)
        {
            first = middle;
            ++first;
            len = len - half - 1;
        }
        else // value <= *middle
        {
            len = half;
        }
    }
    return first;
}
\end{lstlisting}
\end{figure}

Finally take a look at the lower bound function (Fig. \ref{fig:lowerbound}.)
It searches a value in a sorted sequence using binary search and
returns an iterator pointing to the first position that has a value that is equal to or greater than the given value.
It uses the Distance and Next helper functions.

Now comes the clue of this story. There is only one version and that is for the forward iterator.
Where's the random access version?
We don't need it, because when the lower bound function is instantiated with a random access iterator, the
compiler instantiates the fast random access versions of the Distance and Next functions and they will be called.

\begin{thebibliography}{5}

\bibitem{ACDSTL}
A Concept Design for the STL\\
\url{http://www.open-std.org/jtc1/sc22/wg21/docs/papers/2012/n3351.pdf}

\bibitem{LSGPCPP}
Concepts: Linguistic Suppport for Generic Progamming in C++\\
\url{http://www.osl.iu.edu/publications/prints/2006/Gregor06:Concepts.pdf}

\bibitem{CLCTP}
Concepts Lite: Constraining Templates with Predicates\\
\url{http://www.open-std.org/jtc1/sc22/wg21/docs/papers/2013/n3580.pdf}

\bibitem{DCLCPP}
Design of Concept Libraries for C++\\
\url{http://www.stroustrup.com/sle2011-concepts.pdf}

\bibitem{STL}
Standard Template Library Programmer's Guide\\
\url{http://www.sgi.com/tech/stl/}

\end{thebibliography}

\end{document}
