\documentclass[oneside, a4paper, 11pt]{article}
\usepackage[utf8]{inputenc}
\usepackage{a4wide}
\usepackage{graphics}
\usepackage{url}
\usepackage[colorlinks=true,linkcolor=blue]{hyperref}
\usepackage{listings}
\lstdefinelanguage{Cmajor}{morekeywords={assert, bool, true, false, sbyte, byte, short, ushort, int, uint, long, ulong,
float, double, char, void, enum, cast, namespace, using, static, extern, is, explicit, delegate, inline, cdecl, nothrow,
public, protected, private, internal, virtual, abstract, override, suppress, default, operator, class, return, if, else,
switch, case, default, while, do, for, break, continue, goto, typedef, typename, const, null, this, base, unit_test,
construct, destroy, new, delete, sizeof, try, catch, throw, concept, where, axiom, and, or, not},sensitive=true,morecomment=[l]{//},morecomment=[s]{/*}{*/},morestring=[b]\",morestring=[b]',}
\lstloadlanguages{Cmajor}
\lstset{language=Cmajor,showstringspaces=false,breaklines=true,basicstyle=\small,frameround=fttt}

\begin{document}

\title{Unit Testing in Cmajor}
\author{Seppo Laakko}
\maketitle

\section{Tutorial}

Cmajor has a command line tool named \textbf{cmunit}, that
eases testing Cmajor libraries.
Assume that you have a library \emph{Foo} that contains some
functions and classes to be tested (Fig. \ref{fig:foo}).

\begin{figure}[htb]\caption{Foo}\label{fig:foo}
\begin{lstlisting}[frame=trBL]
namespace Foo
{
    public int Add(int x, int y)
    {
        return x + y;
    }

    public int Sub(int x, int y)
    {
        return x - y;
    }

    public class Cls
    {
        public int Twice(int x)
        {
            return 2 * x;
        }
        public int Square(int x)
        {
            return x * x;
        }

    }
}
\end{lstlisting}
\end{figure}

\clearpage
First we create a project file for the test project:

\begin{verbatim}
project TestFoo;
target=program;
assembly <testfoo.cma>;
reference <../foo.cml>;
source <testfoo.cm>;
\end{verbatim}

It references the library \textbf{foo.cml} to be tested and
contains one source file \textbf{testfoo.cm} that will contain the
tests.

Next we create that source file (Fig. \ref{fig:testfoo}).

\begin{figure}[htb]\caption{testfoo.cm}\label{fig:testfoo}
\begin{lstlisting}[frame=trBL]
using Foo;

namespace Tests
{
    public unit_test void TestAdd()
    {
        #assert(Add(1, 2) == 3);
        #assert(Add(0, 100) == 100);
        // ...
    }

    public unit_test void TestSub()
    {
        #assert(Sub(1, 2) == -1);
        // ...
    }

    public unit_test void TestCls()
    {
        Cls c;
        #assert(c.Twice(3) == 6);
        #assert(c.Square(2) == 4);
        // ...
    }

}
\end{lstlisting}
\end{figure}

The source file contains functions
that have a \textbf{unit\_test} specifier
and assert statements.
The \textbf{unit\_test} specifier
tells \textbf{cmunit} program that
the function is a unit test and not
just an ordinary function.
The asserts contain test expressions
that should be true.
The \textbf{cmunit} executes
the asserts and counts how many of them
succeeded and how many failed.

Next we run the tests using \textbf{cmunit}:

\begin{verbatim}
C:\Programming\cmajorbin\test\foo\testfoo>cmunit testfoo.cmp
testing project 'TestFoo' using llvm backend and debug configuration...
C:/Programming/cmajorbin/test/foo/testfoo/testfoo.cm
> Tests.TestAdd
< Tests.TestAdd: 2 assertions, 2 passed, 0 failed
> Tests.TestSub
< Tests.TestSub: 1 assertions, 1 passed, 0 failed
> Tests.TestCls
< Tests.TestCls: 2 assertions, 2 passed, 0 failed
C:/Programming/cmajorbin/test/foo/testfoo/testfoo.cm (ran 3 tests): 3 passed, 0
failed (0 did not compile, 0 exceptions, 0 crashed, 0 unknown results)
test results for project 'TestFoo' (ran 3 tests):
passed             : 3
failed:            : 0
  did not compile  : 0
  exceptions       : 0
  crashed          : 0
  unknown result   : 0

9 seconds
\end{verbatim}

\textbf{cmunit} takes each unit test in turn and creates a
program for each. Program's main() function calls the unit test.
It then compiles the program. If the compilation fails, the
test fails right away. Otherwise \textbf{cmunit}
executes the program. Each assert is executed and results
for them are printed. If all asserts succeed the test program returns 0,
if some assertion failed the test program returns 1.
If the unit test threw an exception, the test program returns 2.
If the program receives a segmentation violation signal, 255 is returned.
\textbf{cmunit} records the return values and prints a report.

Now if Sub function is modified to simulate a bug (Fig. \ref{fig:modifiedfoo}),

\begin{figure}[htb]\caption{Modified Foo}\label{fig:modifiedfoo}
\begin{lstlisting}[frame=trBL]
// ...
public int Sub(int x, int y)
{
    return x - y - 1;
}
// ...
\end{lstlisting}
\end{figure}

and \textbf{cmunit} is executed again. We will see the result of a failed
assertion:

\begin{verbatim}
C:\Programming\cmajorbin\test\foo\testfoo>cmunit testfoo.cmp
testing project 'TestFoo' using llvm backend and debug configuration...
C:/Programming/cmajorbin/test/foo/testfoo/testfoo.cm
> Tests.TestAdd
< Tests.TestAdd: 2 assertions, 2 passed, 0 failed
> Tests.TestSub
  assertion 'Sub(1, 2) == -1' FAILED in file C:/Programming/cmajorbin/test/foo/t
estfoo/testfoo.cm at line 14
< Tests.TestSub: 1 assertions, 0 passed, 1 failed
> Tests.TestCls
< Tests.TestCls: 2 assertions, 2 passed, 0 failed
C:/Programming/cmajorbin/test/foo/testfoo/testfoo.cm (ran 3 tests): 2 passed, 1
failed (0 did not compile, 0 exceptions, 0 crashed, 0 unknown results)
test results for project 'TestFoo' (ran 3 tests):
passed             : 2
failed:            : 1
  did not compile  : 0
  exceptions       : 0
  crashed          : 0
  unknown result   : 0

7 seconds
\end{verbatim}

\section{Reference}

\begin{verbatim}
Cmajor release mode unit test engine version 1.0.0
Usage: cmunit [options] {file.cms | file.cmp}
Compile and run unit tests in solution file.cms or project file.cmp
options:
-config=debug   : use debug configuration (default)
-config=release : use release configuration
-backend=llvm   : use LLVM backend (default)
-backend=c      : use C backend
-file=FILE      : run only unit tests in file FILE
-test=TEST      : run only unit test TEST
-D SYMBOL       : define conditional compilation symbol SYMBOL\end{verbatim}

By default \textbf{cmunit} compiles and runs all tests in the given project or solution
using debug configuration and LLVM backend settings.
The configuration used can be given using the -config option and the backend used
using the -backend option.

By using the -file=FILE option together with a project or solution \textbf{cmunit} compiles and
runs only tests in the source file FILE.

By using the -test=TEST option together with a project or solution \textbf{cmunit} compiles and
runs only given test named TEST.

\end{document}
