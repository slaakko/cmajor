\documentclass[oneside, a4paper, 11pt]{article}
\usepackage[utf8]{inputenc}
\usepackage{a4wide}
\usepackage{graphics}
\usepackage{url}
\usepackage[colorlinks=true,linkcolor=blue]{hyperref}

\begin{document}

\title{Profiling Cmajor Applications}
\author{Seppo Laakko}
\maketitle

\section{Profiling in Windows}

\subsection{Creating a profile report}

The steps for creating a profile report
for the \emph{Downloads} example application are as follows.

(This example profiles the LLVM backend
but the steps are the same for profiling the C backend.)

\begin{itemize}

\item
If you haven't build the system library using profile configuration
in installation time, now is the time to do it:

\begin{itemize}

\item
Open Cmajor Development Environment.

\item
Open Built-in Projects $|$ System Library project.

\item
Select profile configuration from Configuration combo box.

\item
Select Build $|$ Build Solution.
This builds the projects in the System solution and instruments
the functions in its projects with calls to collect timing information.

\end{itemize}

\item
Open Built-in Projects $|$ Examples project.

\item
Select profile configuration from Configuration combo box.

\item
Right click the Downloads project and choose Build.
This builds the Downloads project and instruments the functions in it
with calls to collect timing information.
Also a \textbf{downloads.cmprof} file is created that contains
a unique function identifier for each function in the program.

\item
Open command prompt and change to
\%APPDATA\%$\backslash$Cmajor$\backslash$examples$\backslash$downloads$\backslash$profile$\backslash$llvm
directory:
\begin{verbatim}
cd %APPDATA%\Cmajor\examples\downloads\profile\llvm
\end{verbatim}

\item
Run downloads program:
\begin{verbatim}
downloads
\end{verbatim}

This creates a binary file \textbf{downloads.profdata} that contains the
timing information.

\item
Run \textbf{cmprof downloads.cmprof}:
\begin{verbatim}
cmprof downloads.cmprof
\end{verbatim}

This scans the profile data and creates a  profile report (\textbf{downloads.cmprofreport}).

\end{itemize}

\subsection{Inspecting the profile report}

Run profile report viewer \textbf{cmprofview} by issuing command:
\begin{verbatim}
cmprofview downloads.cmprofreport
\end{verbatim}

Alternatively you can double click the \textbf{downloads.cmprofreport} file in the file explorer
to open it in the profile report viewer.

The columns in the report are as follows:

\begin{itemize}

\item Function:
Full name of the function.

\item Called:
The number of times the function is called.

\item Elapsed Inclusive ms:
Time spent in executing this function and the functions it calls in milliseconds.

\item Elapsed Inclusive \%:
Percentage of time spent is executing this function and the functions it calls
compared to total execution time.

\item Elapsed Exclusive ms:
Time spent in executing this function excluding the time spent in executing
the functions it calls in milliseconds.

\item Elapsed Exclusive \%:
Percentage of time spent is executing this function excluding the time spent in executing
the functions it calls compared to total execution time.

\end{itemize}

You can click the column headers in the report to sort the rows in ascending or descending
order based on the clicked column.

\clearpage

\section{Profiling in Linux}

\subsection{Creating a profile report}

The steps for creating a profile report
for the \emph{Downloads} example application are as follows.

(This example profiles the LLVM backend
but the steps are the same for profiling the C backend.)

\begin{itemize}

\item
If you haven't build the system library using profile configuration
in installation time, now is the time to do it:

\begin{itemize}

\item
Open a terminal window and change to \verb|<major>/system| directory:
\begin{verbatim}
seppo@raid:~$ cd Programming/cmajor-1.2.0/system
\end{verbatim}

\item
Build system library using profile configuration:
\begin{verbatim}
seppo@raid:~/Programming/cmajor-1.2.0/system$
cmc -config=profile system.cms
\end{verbatim}

\end{itemize}

\item
Change to \verb|<major>/examples/downloads| directory:
\begin{verbatim}
seppo@raid:~/Programming/cmajor-1.2.0/system$
cd ../examples/downloads
\end{verbatim}

\item
Build the downloads project using profile configuration:
\begin{verbatim}
seppo@raid:~/Programming/cmajor-1.2.0/examples/downloads$
cmc -config=profile downloads.cmp
\end{verbatim}

\item
Change to the profile/llvm subdirectory:
\begin{verbatim}
seppo@raid:~/Programming/cmajor-1.2.0/examples/downloads$
cd profile/llvm
\end{verbatim}

\item
Run downloads program:
\begin{verbatim}
seppo@raid:~/Programming/cmajor-1.2.0/examples/downloads/profile/llvm$
./downloads
\end{verbatim}

This creates a binary file \textbf{downloads.profdata} that contains the
timing information.

\item
Run \textbf{cmprof downloads.cmprof}:
\begin{verbatim}
seppo@raid:~/Programming/cmajor-1.2.0/examples/downloads/profile/llvm$
cmprof downloads.cmprof
\end{verbatim}

This scans the profile data and creates a  profile report (\textbf{downloads.cmprofreport}) and
three additional text files:
\textbf{downloads.count.txt}, \textbf{downloads.inclusive.txt} and \textbf{downloads.exclusive.txt}.

\end{itemize}

\subsection{Inspecting profile report files}

The \textbf{downloads.cmprofreport} file contains data in random order.
The \textbf{downloads.count.txt} file contains the data sorted by call count.
The \textbf{downloads.inclusive.txt} file contains the data sorted by elapsed inclusive milliseconds.
The \textbf{downloads.exclusive.txt} file contains the data sorted by elapsed exclusive milliseconds.

The columns in the report are as follows:

\begin{itemize}

\item Function identifier.
Unique integer of the function.

\item Function:
Full name of the function.

\item Called:
The number of times the function is called.

\item Elapsed Inclusive ms:
Time spent in executing this function and the functions it calls in milliseconds.

\item Elapsed Inclusive \%:
Percentage of time spent is executing this function and the functions it calls
compared to total execution time.

\item Elapsed Exclusive ms:
Time spent in executing this function excluding the time spent in executing
the functions it calls in milliseconds.

\item Elapsed Exclusive \%:
Percentage of time spent is executing this function excluding the time spent in executing
the functions it calls compared to total execution time.

\end{itemize}

\end{document}
