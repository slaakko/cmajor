\documentclass[oneside, a4paper, 11pt]{article}
\usepackage[utf8]{inputenc}
\usepackage{a4wide}
\usepackage{graphics}
\usepackage{url}
\usepackage[colorlinks=true,linkcolor=blue]{hyperref}
\usepackage{syntax}

\begin{document}

\title{Debugging with Cmajor Debugger}
\author{Seppo Laakko}
\maketitle

\section{Introduction}

Cmajor programs can be debugged using the Cmajor source level debugger (cmdb).
Currently the program to be debugged and libraries it uses must be first compiled with Cmajor compiler (cmc)
using the ``C backend'' option (-backend=c) and the debug configuration.
In Windows programs can be debugged directly in Cmajor Development Environment (IDE) that talks to cmdb behind the scenes.
Cmajor debugger in turn talks to GNU debugger (gdb) behind the scenes.

\section{Debugging Commands}

Table \ref{tab:commands} lists available debugging commands, their abbreviations and IDE equivalents.

\begin{table}[htb]
\caption{Debugging Commands}\label{tab:commands}
\begin{tabular}{lll}
\textbf{Command} & \textbf{Abbreviation} & \textbf{IDE command}\\
\hline
start & &\\
quit & q & \verb/Stop Debugging/\\
help & h & \verb/Help | Debugging/\\
continue & c & \verb/Debug | Continue/\\
next & n & \verb/Debug | Step Over/\\
step & s & \verb/Debug | Step Into/\\
out & o & \verb/Debug | Step Out/\\
break [FILE:]LINE & b [FILE:]LINE & \verb/Debug | Toggle breakpoint/\\
clear N & cl N & \verb/Debug | Toggle breakpoint/\\
callstack & ca\\
frame N & f N\\
list [FILE:]LINE & l [FILE:]LINE &\\
list* & l* &\\
list & l &\\
inspect EXPR & i EXPR& \verb/Debug | Inspect.../\\
show breakpoints\\
set break on throw (on / off)\\
empty line & ENTER\\
\end{tabular}
\end{table}

Descriptions of the commands:

\begin{itemize}

\item start

Starts a debugging session.

\item quit

Ends debugging session and exits.

\item continue

Runs program until it stops to a breakpoint or exits.

\item next

Goes to next source line by stepping over function calls.

\item step

Goes to next source line or steps into a function.

\item out

Goes out of function.

\item break [FILE:]LINE

Sets a breakpoint to a source line.

\item clear N

Clears breakpoint number N.

\item callstack

Shows current call stack.

\item frame N

Sets current call frame for inspecting.

\item list [FILE:]LINE

Lists source code around line LINE.

\item list*

Lists source code around current position.

\item list

Lists next lines.

\item inspect EXPR

Inspects value of expression EXPR.

\item show breakpoints

Shows list of breakpoints.

\item set break on throw on

Sets implicit breakpoint in each throw statement.

\item set break on throw off

Clears implicit breakpoint in each throw statement.

\item empty line

Repeats last command.

\end{itemize}

\section{Inspect Expressions}

Inspect expression is a content expression.

\begin{grammar}
<inspect-expr> ::= <content-expr>
\end{grammar}

\subsection{Content Expression}

Content expression is either an \emph{at} character followed
by a prefix expression, or a sole prefix expression.

\begin{grammar}
<content-expr> ::= '@' <prefix-expr> | <prefix-expr>
\end{grammar}

\emph{at}-prefixed content expression evaluates a content
of a class or pointer to a class.
Specializations of \emph{System.Collections.List},
\emph{System.Collections.Set} and \emph{System.Collections.Map},
receive special treatment from the debugger to print the
contents of them. \emph{at}-prefixed content expression for an ordinary class
prints values of its member variables.
\emph{at}-prefixed content expression for a pointer evaluates the
dynamic type of the class pointed and prints its content.

\subsection{Prefix Expression}

A prefix expression consists of a prefix operator '*' followed by
another prefix expression, or a postfix expression.

\begin{grammar}
<prefix-expr> ::= ("*" <prefix-expr>)
\alt <postfix-expr>
\end{grammar}

The '*' prefix operator derefences a following pointer expression.

\subsection{Postfix Expression}

A postfix expression consists of a primary expression followed by zero or more
postfix operations.

\begin{grammar}
<postfix-expr> ::= <primary-expr> ("." <member-id> | "->" <member-id>)*

<member-id> ::= <identifier> | \lit*{base}

<identifier> ::= (letter | '_') (letter | digit | '_')*
\end{grammar}

A '.' postfix operation selects a member of a class.

A '$->$' postfix operation selects a member of an expression that has a type of a pointer to class.

\subsection{Primary Expression}

A primary expression is a this-pointer, a name of a local variable, a handle expression or a parenthesized perfix expression.

\begin{grammar}
<primary-expr> ::= \lit*{this} | <local-variable-name> | "$" <int> | "(" <prefix-expr> ")"

<local-variable-name> ::= <identifier>
\end{grammar}

A handle expression consists of a '\$' character followed by a integer handle returned by GDB.
Using handle expression one can refer to previously evaluated expressions.

\section{Usage}

\begin{verbatim}
usage: cmdb [options] program [arguments...]
options:
-----------------------------------------
-ide:             use IDE mode
-file=FILE        read commands from FILE
\end{verbatim}

\end{document}
