\documentclass[a4paper,oneside,11.000000pt]{book}

\usepackage{array}
\usepackage{amsmath}
\usepackage{amsfonts}
\usepackage{amssymb}
\usepackage{graphicx}
\usepackage{pifont}
\usepackage{a4wide}
\usepackage{url}
\usepackage{float}
\usepackage{layouts}
\usepackage{supertabular}
\usepackage{titlesec}
\usepackage{listings}
\usepackage{syntax}
\usepackage[colorlinks=true,linkcolor=blue]{hyperref}
\setcounter{secnumdepth}{10}
\setcounter{tocdepth}{4}

\titleformat{\chapter}[hang]{\bf\Huge}{\arabic{chapter}}{1em}{}{}
\titleclass{\subchapter}{straight}[\chapter]
\newcounter{subchapter}
\renewcommand{\thesubchapter}{\thechapter.\arabic{subchapter}}
\titleformat{\subchapter}[hang]{\bf\huge}{\thesubchapter}{0.5em}{}{}
\titlespacing{\subchapter}{0pt}{0pt}{0pt}
\renewcommand\thesection{\thesubchapter.\arabic{section}}
\makeatletter
\newcommand*\l@subchapter{\@dottedtocline{1}{1.5em}{2.3em}}
\renewcommand\l@section{\@dottedtocline{2}{3.8em}{3.2em}}
\renewcommand\l@subsection{\@dottedtocline{3}{7em}{4.1em}}
\renewcommand\l@subsubsection{\@dottedtocline{4}{11.1em}{5em}}
\makeatother

\makeatletter
\renewcommand\subparagraph{\@startsection{subparagraph}{5}{0mm}{-\baselineskip}{0.5\baselineskip}{\normalfont\normalsize\scshape}}
\makeatother

\lstdefinelanguage{Cmajor}{morekeywords={bool, true, false, sbyte, byte, short, ushort, int, uint, long, ulong, float, double, char, void, enum, cast, namespace, using, static, extern, is, explicit, delegate, inline, cdecl, nothrow, public, protected, private, internal, virtual, abstract, override, suppress, default, operator, class, return, if, else, switch, case, default, while, do, for, break, continue, goto, typedef, typename, const, null, this, base, construct, destroy, new, delete, sizeof, try, catch, throw, concept, where, axiom, and, or, not},sensitive=true,morecomment=[l]{//},morecomment=[s]{/*}{*/},morestring=[b]",morestring=[b]',}\lstloadlanguages{Cmajor}
\lstset{language=Cmajor,showstringspaces=false,breaklines=true,basicstyle=\small}
\makeatletter\def\toclevel@chapter{0}\def\toclevel@subchapter{1}\def\toclevel@section{2}\def\toclevel@subsection{3}\def\toclevel@subsubsection{4}\makeatother\begin{document}
\clearpage

\frontmatter
\title{\textsc{System.Net.Sockets Library Reference}
}
\date{\today}
\maketitle
\tableofcontents

\clearpage
\chapter{Description}
\begin{flushleft}
Provides support for TCP sockets.

\end{flushleft}
\chapter{Copyrights}
\begin{verbatim}
========================================================================
Copyright (c) 2012-2016 Seppo Laakko
http://sourceforge.net/projects/cmajor/

Distributed under the GNU General Public License, version 3 (GPLv3).
(See accompanying LICENSE.txt or http://www.gnu.org/licenses/gpl.html)

========================================================================

\end{verbatim}
\clearpage
\chapter{Namespaces}
\begin{flushleft}
\begin{supertabular}[l]{!{\raggedright}p{4.52406cm}!{\raggedright}p{10.6259cm}}
\textbf{Namespace}
& \textbf{Description}
\\
\hline
\hyperlink{global}{Global}
& Global namespace contains C functions for the implementation.

\\
\hyperlink{System.Net.Sockets}{System.Net.Sockets}
& Provides support for TCP sockets.

\\
\end{supertabular}

\end{flushleft}
\clearpage
\mainmatter

\chapter{Usage}

\section{Referencing the System.Net.Sockets library}

Right-click a project node in IDE \verb.|. Project References... \verb.|.
Add System Extension Library Reference... \verb.|.
enable \emph{System.Net.Sockets} check box

\begin{flushleft}
or add following line to your project's .cmp file:
\begin{verbatim}
reference <ext/System.Net.Sockets/System.Net.Sockets.cml>;
\end{verbatim}
\end{flushleft}
\hypertarget{global}{\chapter{Global Namespace}}
\begin{flushleft}
Global namespace contains C functions for the implementation.

\end{flushleft}
\clearpage
\subchapter{Concepts}
\begin{flushleft}
\begin{supertabular}[l]{ll}
\textbf{Concept}
& \textbf{Description}
\\
\hline
\end{supertabular}

\end{flushleft}
\clearpage
\clearpage
\clearpage
\clearpage
\clearpage
\clearpage
\clearpage

\subchapter{Functions}
\begin{flushleft}
\begin{supertabular}[l]{ll}
\textbf{Function}
& \textbf{Description}
\\
\hline
\end{supertabular}

\end{flushleft}
\clearpage
\clearpage
\subchapter{Enumerations}
\begin{flushleft}
\begin{supertabular}[l]{ll}
\textbf{Enumeration}
& \textbf{Description}
\\
\hline
\hyperlink{ShutdownMode}{ShutdownMode}
& Mode for the shut down operation.

\\
\end{supertabular}

\end{flushleft}
\clearpage
\hypertarget{ShutdownMode}{\subsection{ShutdownMode Enumeration}}
\begin{flushleft}
Mode for the shut down operation.

\end{flushleft}
\subsubsection*{Enumeration Constants}
\begin{flushleft}
\begin{supertabular}[l]{lll}
\textbf{Constant}
& \textbf{Value}
& \textbf{Description}
\\
\hline
\hypertarget{ShutdownMode.both}{both}
& 2
& Shuts down both receiving and sending.

\\
\hypertarget{ShutdownMode.receive}{receive}
& 0
& Shuts down receiving from a socket.

\\
\hypertarget{ShutdownMode.send}{send}
& 1
& Shuts down sending to a socket.

\\
\end{supertabular}

\end{flushleft}
\clearpage

\hypertarget{System.Net.Sockets}{\chapter{System.Net.Sockets Namespace}}\begin{flushleft}
Provides support for TCP sockets.

\end{flushleft}
\clearpage

\subchapter{Classes}
\begin{flushleft}
\begin{supertabular}[l]{!{\raggedright}p{5.56131cm}!{\raggedright}p{9.58869cm}}
\textbf{Class}
& \textbf{Description}
\\
\hline
\hyperlink{System.Net.Sockets.NetworkBuffer}{NetworkBuffer}
& A handle to a dynamically allocated memory.

\\
\hyperlink{System.Net.Sockets.SocketByteStream}{SocketByteStream}
& Represent stream of bytes connected to a \hyperlink{System.Net.Sockets.TcpSocket}{TcpSocket}.

\\
\hyperlink{System.Net.Sockets.SocketError}{SocketError}
& An exception class throw when a socket operation fails.

\\
\hyperlink{System.Net.Sockets.SocketLibrary}{SocketLibrary}
& Represents the socket library initializer implemented as a singleton.

\\
\hyperlink{System.Net.Sockets.SocketLibraryException}{SocketLibraryException}
& Exception class thrown when the initialization of the socket library fails.

\\
\hyperlink{System.Net.Sockets.TcpSocket}{TcpSocket}
& Represents a TCP socket.

\\
\end{supertabular}

\end{flushleft}
\clearpage

\hypertarget{System.Net.Sockets.NetworkBuffer}{\section{NetworkBuffer Class}}\begin{flushleft}
A handle to a dynamically allocated memory.

\end{flushleft}

\subsection*{Syntax}\texttt{public class NetworkBuffer;}
\subsection{Member Functions}
\begin{flushleft}
\begin{supertabular}[l]{!{\raggedright}p{7.575cm}!{\raggedright}p{7.575cm}}
\textbf{Member Function}
& \textbf{Description}
\\
\hline
\hyperlink{System.Net.Sockets.NetworkBuffer.constructor.P.System.Net.Sockets.NetworkBuffer.RR.System.Net.Sockets.NetworkBuffer}{NetworkBuffer(System.\-Net.\-Sockets.\-NetworkBuffer\&\-\&\-)}
& Move constructor.

\\
\hyperlink{System.Net.Sockets.NetworkBuffer.operator.assign.P.System.Net.Sockets.NetworkBuffer.RR.System.Net.Sockets.NetworkBuffer}{operator=(System.\-Net.\-Sockets.\-NetworkBuffer\&\-\&\-)}
& Move assignment.

\\
\hyperlink{System.Net.Sockets.NetworkBuffer.Mem.C.P.System.Net.Sockets.NetworkBuffer}{Mem() const}
& Returns a pointer to the allocated memory block.

\\
\hyperlink{System.Net.Sockets.NetworkBuffer.constructor.P.System.Net.Sockets.NetworkBuffer.int}{NetworkBuffer(int)}
& Constructor. Allocates specified number of bytes from the system.

\\
\hyperlink{System.Net.Sockets.NetworkBuffer.Size.C.P.System.Net.Sockets.NetworkBuffer}{Size() const}
& Returns the size of the allocated memory block.

\\
\hyperlink{System.Net.Sockets.NetworkBuffer.destructor.P.System.Net.Sockets.NetworkBuffer}{$\sim$NetworkBuffer()}
& Destructor. Frees the allocated memory back to the system.

\\
\end{supertabular}

\end{flushleft}
\clearpage

\hypertarget{System.Net.Sockets.NetworkBuffer.constructor.P.System.Net.Sockets.NetworkBuffer.RR.System.Net.Sockets.NetworkBuffer}{\subsubsection*{NetworkBuffer(System.Net.Sockets.NetworkBuffer\&\&) Member Function}}\begin{flushleft}
Move constructor.

\end{flushleft}
\subsubsection*{Syntax}
\texttt{public NetworkBuffer(System.Net.Sockets.NetworkBuffer\&\& that);}
\subsubsection*{Parameters}
\begin{flushleft}
\begin{supertabular}[l]{!{\raggedright}p{1.30721cm}!{\raggedright}p{8.55945cm}!{\raggedright}p{2.65538cm}}
\textbf{Name}
& \textbf{Type}
& \textbf{Description}
\\
\hline
that
& \hyperlink{System.Net.Sockets.NetworkBuffer}{System.\-Net.\-Sockets.\-NetworkBuffer\&\-\&\-}
& 
\\
\end{supertabular}

\end{flushleft}
\clearpage

\hypertarget{System.Net.Sockets.NetworkBuffer.operator.assign.P.System.Net.Sockets.NetworkBuffer.RR.System.Net.Sockets.NetworkBuffer}{\subsubsection*{operator=(System.Net.Sockets.NetworkBuffer\&\&) Member Function}}\begin{flushleft}
Move assignment.

\end{flushleft}
\subsubsection*{Syntax}
\texttt{public void operator=(System.Net.Sockets.NetworkBuffer\&\& \_\_parameter0);}
\subsubsection*{Parameters}
\begin{flushleft}
\begin{supertabular}[l]{!{\raggedright}p{2.61335cm}!{\raggedright}p{7.25332cm}!{\raggedright}p{2.65538cm}}
\textbf{Name}
& \textbf{Type}
& \textbf{Description}
\\
\hline
\_\_parameter0
& \hyperlink{System.Net.Sockets.NetworkBuffer}{System.\-Net.\-Sockets.\-NetworkBuffer\&\-\&\-}
& 
\\
\end{supertabular}

\end{flushleft}
\clearpage

\hypertarget{System.Net.Sockets.NetworkBuffer.Mem.C.P.System.Net.Sockets.NetworkBuffer}{\subsubsection*{Mem() const Member Function}}
\begin{flushleft}
Returns a pointer to the allocated memory block.

\end{flushleft}
\subsubsection*{Syntax}\texttt{public void* Mem() const;}

\subsubsection*{Returns}void*
\begin{flushleft}
Returns a pointer to the allocated memory block.

\end{flushleft}
\clearpage

\hypertarget{System.Net.Sockets.NetworkBuffer.constructor.P.System.Net.Sockets.NetworkBuffer.int}{\subsubsection*{NetworkBuffer(int) Member Function}}
\begin{flushleft}
Constructor. Allocates specified number of bytes from the system.

\end{flushleft}
\subsubsection*{Syntax}\texttt{public NetworkBuffer(int size\_);}

\subsubsection*{Parameters}
\begin{flushleft}
\begin{supertabular}[l]{lll}
\textbf{Name}
& \textbf{Type}
& \textbf{Description}
\\
\hline
size\_
& int
& The number of bytes to allocate.

\\
\end{supertabular}

\end{flushleft}
\clearpage

\hypertarget{System.Net.Sockets.NetworkBuffer.Size.C.P.System.Net.Sockets.NetworkBuffer}{\subsubsection*{Size() const Member Function}}
\begin{flushleft}
Returns the size of the allocated memory block.

\end{flushleft}
\subsubsection*{Syntax}\texttt{public int Size() const;}

\subsubsection*{Returns}int
\begin{flushleft}
Returns the size of the allocated memory block.

\end{flushleft}
\clearpage

\hypertarget{System.Net.Sockets.NetworkBuffer.destructor.P.System.Net.Sockets.NetworkBuffer}{\subsubsection*{$\sim$NetworkBuffer() Member Function}}
\begin{flushleft}
Destructor. Frees the allocated memory back to the system.

\end{flushleft}
\subsubsection*{Syntax}\texttt{public $\sim$NetworkBuffer();}
\clearpage

\hypertarget{System.Net.Sockets.SocketByteStream}{\section{SocketByteStream Class}}
\begin{flushleft}
Represent stream of bytes connected to a \hyperlink{System.Net.Sockets.TcpSocket}{TcpSocket}.

\end{flushleft}
\subsection*{Syntax}\texttt{public class SocketByteStream;}

\subsection*{Base Class}System.IO.ByteStream\subsection{Member Functions}
\begin{flushleft}
\begin{supertabular}[l]{!{\raggedright}p{7.575cm}!{\raggedright}p{7.575cm}}
\textbf{Member Function}
& \textbf{Description}
\\
\hline
\hyperlink{System.Net.Sockets.SocketByteStream.constructor.P.System.Net.Sockets.SocketByteStream}{SocketByteStream()}
& Default constructor.

\\
\hyperlink{System.Net.Sockets.SocketByteStream.constructor.P.System.Net.Sockets.SocketByteStream.RR.System.Net.Sockets.SocketByteStream}{SocketByteStream(System.\-Net.\-Sockets.\-SocketByteStream\&\-\&\-)}
& Move constructor.

\\
\hyperlink{System.Net.Sockets.SocketByteStream.operator.assign.P.System.Net.Sockets.SocketByteStream.RR.System.Net.Sockets.SocketByteStream}{operator=(System.\-Net.\-Sockets.\-SocketByteStream\&\-\&\-)}
& Move assignment.

\\
\hyperlink{System.Net.Sockets.SocketByteStream.Read.P.System.Net.Sockets.SocketByteStream.P.byte.int}{Read(byte*, int)}
& Reads at most given number of bytes from the \hyperlink{System.Net.Sockets.TcpSocket}{TcpSocket} to the given buffer.
Returns the number of bytes read.

\\
\hyperlink{System.Net.Sockets.SocketByteStream.ReadByte.P.System.Net.Sockets.SocketByteStream}{ReadByte()}
& Reads one byte of data from the \hyperlink{System.Net.Sockets.TcpSocket}{TcpSocket} and returns it.
Return value of -1 indicates peer has shut down the connection (end of stream).

\\
\hyperlink{System.Net.Sockets.SocketByteStream.constructor.P.System.Net.Sockets.SocketByteStream.RR.System.Net.Sockets.TcpSocket}{SocketByteStream(System.\-Net.\-Sockets.\-TcpSocket\&\-\&\-)}
& Constructor. Initializes the socket byte stream with the given \hyperlink{System.Net.Sockets.TcpSocket}{TcpSocket}.

\\
\hyperlink{System.Net.Sockets.SocketByteStream.constructor.P.System.Net.Sockets.SocketByteStream.C.R.System.String.C.R.System.String}{SocketByteStream(const System.\-String\&\-, const System.\-String\&\-)}
& Constructor. Initializes the socket byte stream by creating a TCP socket and connecting it to the specified node and service.

\\
\hyperlink{System.Net.Sockets.SocketByteStream.Write.P.System.Net.Sockets.SocketByteStream.byte}{Write(byte)}
& Writes given byte to the socket byte stream.

\\
\hyperlink{System.Net.Sockets.SocketByteStream.Write.P.System.Net.Sockets.SocketByteStream.P.byte.int}{Write(byte*, int)}
& Writes given number of bytes from the given buffer to the socket byte stream.

\\
\hyperlink{System.Net.Sockets.SocketByteStream.destructor.P.System.Net.Sockets.SocketByteStream}{$\sim$SocketByteStream()}
& Destructor. Closes the socket.

\\
\end{supertabular}

\end{flushleft}
\clearpage

\hypertarget{System.Net.Sockets.SocketByteStream.constructor.P.System.Net.Sockets.SocketByteStream}{\subsubsection*{SocketByteStream() Member Function}}\begin{flushleft}
Default constructor.

\end{flushleft}
\subsubsection*{Syntax}
\texttt{public SocketByteStream();}
\clearpage

\hypertarget{System.Net.Sockets.SocketByteStream.constructor.P.System.Net.Sockets.SocketByteStream.RR.System.Net.Sockets.SocketByteStream}{\subsubsection*{SocketByteStream(System.Net.Sockets.SocketByteStream\&\&) Member Function}}\begin{flushleft}
Move constructor.

\end{flushleft}
\subsubsection*{Syntax}
\texttt{public SocketByteStream(System.Net.Sockets.SocketByteStream\&\& \_\_parameter0);}
\subsubsection*{Parameters}
\begin{flushleft}
\begin{supertabular}[l]{!{\raggedright}p{2.61335cm}!{\raggedright}p{7.25332cm}!{\raggedright}p{2.65538cm}}
\textbf{Name}
& \textbf{Type}
& \textbf{Description}
\\
\hline
\_\_parameter0
& \hyperlink{System.Net.Sockets.SocketByteStream}{System.\-Net.\-Sockets.\-SocketByteStream\&\-\&\-}
& 
\\
\end{supertabular}

\end{flushleft}
\clearpage

\hypertarget{System.Net.Sockets.SocketByteStream.operator.assign.P.System.Net.Sockets.SocketByteStream.RR.System.Net.Sockets.SocketByteStream}{\subsubsection*{operator=(System.Net.Sockets.SocketByteStream\&\&) Member Function}}\begin{flushleft}
Move assignment.

\end{flushleft}
\subsubsection*{Syntax}
\texttt{public void operator=(System.Net.Sockets.SocketByteStream\&\& \_\_parameter0);}
\subsubsection*{Parameters}
\begin{flushleft}
\begin{supertabular}[l]{!{\raggedright}p{2.61335cm}!{\raggedright}p{7.25332cm}!{\raggedright}p{2.65538cm}}
\textbf{Name}
& \textbf{Type}
& \textbf{Description}
\\
\hline
\_\_parameter0
& \hyperlink{System.Net.Sockets.SocketByteStream}{System.\-Net.\-Sockets.\-SocketByteStream\&\-\&\-}
& 
\\
\end{supertabular}

\end{flushleft}
\clearpage

\hypertarget{System.Net.Sockets.SocketByteStream.Read.P.System.Net.Sockets.SocketByteStream.P.byte.int}{\subsubsection*{Read(byte*, int) Member Function}}
\begin{flushleft}
Reads at most given number of bytes from the \hyperlink{System.Net.Sockets.TcpSocket}{TcpSocket} to the given buffer.
Returns the number of bytes read.

\end{flushleft}
\subsubsection*{Syntax}\texttt{public int Read(byte* buf, int count);}

\subsubsection*{Parameters}
\begin{flushleft}
\begin{supertabular}[l]{lll}
\textbf{Name}
& \textbf{Type}
& \textbf{Description}
\\
\hline
buf
& byte*
& A buffer to read to.

\\
count
& int
& Maximum number of bytes to read.

\\
\end{supertabular}

\end{flushleft}
\subsubsection*{Returns}int
\begin{flushleft}
Returns number of bytes read.
Return value of 0 indicates peer has shut down the connection (end of stream).

\end{flushleft}
\subsubsection*{Remarks}
\begin{flushleft}
Throws \hyperlink{System.Net.Sockets.SocketError}{SocketError} if reading fails.

\end{flushleft}
\clearpage

\hypertarget{System.Net.Sockets.SocketByteStream.ReadByte.P.System.Net.Sockets.SocketByteStream}{\subsubsection*{ReadByte() Member Function}}
\begin{flushleft}
Reads one byte of data from the \hyperlink{System.Net.Sockets.TcpSocket}{TcpSocket} and returns it.
Return value of -1 indicates peer has shut down the connection (end of stream).

\end{flushleft}
\subsubsection*{Syntax}\texttt{public int ReadByte();}
\subsubsection*{Returns}
int
\begin{flushleft}
Returns the byte read, or -1 if end of stream encountered.

\end{flushleft}
\subsubsection*{Remarks}
\begin{flushleft}
Throws \hyperlink{System.Net.Sockets.SocketError}{SocketError} if reading fails.

\end{flushleft}
\clearpage

\hypertarget{System.Net.Sockets.SocketByteStream.constructor.P.System.Net.Sockets.SocketByteStream.RR.System.Net.Sockets.TcpSocket}{\subsubsection*{SocketByteStream(System.Net.Sockets.TcpSocket\&\&) Member Function}}
\begin{flushleft}
Constructor. Initializes the socket byte stream with the given \hyperlink{System.Net.Sockets.TcpSocket}{TcpSocket}.

\end{flushleft}
\subsubsection*{Syntax}
\texttt{public SocketByteStream(System.Net.Sockets.TcpSocket\&\& socket\_);}
\subsubsection*{Parameters}
\begin{flushleft}
\begin{supertabular}[l]{!{\raggedright}p{1.43869cm}!{\raggedright}p{8.42798cm}!{\raggedright}p{3.31976cm}}
\textbf{Name}
& \textbf{Type}
& \textbf{Description}
\\
\hline
socket\_
& \hyperlink{System.Net.Sockets.TcpSocket}{System.\-Net.\-Sockets.\-TcpSocket\&\-\&\-}
& A \hyperlink{System.Net.Sockets.TcpSocket}{TcpSocket}.

\\
\end{supertabular}

\end{flushleft}
\clearpage

\hypertarget{System.Net.Sockets.SocketByteStream.constructor.P.System.Net.Sockets.SocketByteStream.C.R.System.String.C.R.System.String}{\subsubsection*{SocketByteStream(const System.String\&, const System.String\&) Member Function}}
\begin{flushleft}
Constructor. Initializes the socket byte stream by creating a TCP socket and connecting it to the specified node and service.

\end{flushleft}
\subsubsection*{Syntax}
\texttt{public SocketByteStream(const System.String\& node, const System.String\& service);}
\subsubsection*{Parameters}
\begin{flushleft}
\begin{supertabular}[l]{!{\raggedright}p{1.56801cm}!{\raggedright}p{4.63128cm}!{\raggedright}p{5.23538cm}}
\textbf{Name}
& \textbf{Type}
& \textbf{Description}
\\
\hline
node
& const System.\-String\&\-
& A host name or an IP address to connect.

\\
service
& const System.\-String\&\-
& A protocol name or port number to connect.

\\
\end{supertabular}

\end{flushleft}
\subsubsection*{Remarks}
\begin{flushleft}
Throws \hyperlink{System.Net.Sockets.SocketError}{SocketError} if connecting fails.

\end{flushleft}
\clearpage

\hypertarget{System.Net.Sockets.SocketByteStream.Write.P.System.Net.Sockets.SocketByteStream.byte}{\subsubsection*{Write(byte) Member Function}}\begin{flushleft}
Writes given byte to the socket byte stream.

\end{flushleft}

\subsubsection*{Syntax}\texttt{public void Write(byte x);}
\subsubsection*{Parameters}
\begin{flushleft}
\begin{supertabular}[l]{lll}
\textbf{Name}
& \textbf{Type}
& \textbf{Description}
\\
\hline
x
& byte
& A byte to write.

\\
\end{supertabular}

\end{flushleft}
\subsubsection*{Remarks}
\begin{flushleft}
Throws \hyperlink{System.Net.Sockets.SocketError}{SocketError} if writing fails.

\end{flushleft}
\clearpage

\hypertarget{System.Net.Sockets.SocketByteStream.Write.P.System.Net.Sockets.SocketByteStream.P.byte.int}{\subsubsection*{Write(byte*, int) Member Function}}
\begin{flushleft}
Writes given number of bytes from the given buffer to the socket byte stream.

\end{flushleft}
\subsubsection*{Syntax}\texttt{public void Write(byte* buf, int count);}

\subsubsection*{Parameters}
\begin{flushleft}
\begin{supertabular}[l]{lll}
\textbf{Name}
& \textbf{Type}
& \textbf{Description}
\\
\hline
buf
& byte*
& A buffer of data to write.

\\
count
& int
& Number of bytes to write.

\\
\end{supertabular}

\end{flushleft}
\subsubsection*{Remarks}
\begin{flushleft}
Throws \hyperlink{System.Net.Sockets.SocketError}{SocketError} if writing fails.

\end{flushleft}
\clearpage

\hypertarget{System.Net.Sockets.SocketByteStream.destructor.P.System.Net.Sockets.SocketByteStream}{\subsubsection*{$\sim$SocketByteStream() Member Function}}\begin{flushleft}
Destructor. Closes the socket.

\end{flushleft}

\subsubsection*{Syntax}\texttt{public $\sim$SocketByteStream();}
\clearpage

\hypertarget{System.Net.Sockets.SocketError}{\section{SocketError Class}}
\begin{flushleft}
An exception class throw when a socket operation fails.

\end{flushleft}
\subsection*{Syntax}\texttt{public class SocketError;}
\subsection*{Base Class}
System.Exception\subsection{Member Functions}
\begin{flushleft}
\begin{supertabular}[l]{!{\raggedright}p{7.575cm}!{\raggedright}p{7.575cm}}
\textbf{Member Function}
& \textbf{Description}
\\
\hline
\hyperlink{System.Net.Sockets.SocketError.constructor.P.System.Net.Sockets.SocketError}{SocketError()}
& Default constructor.

\\
\hyperlink{System.Net.Sockets.SocketError.constructor.P.System.Net.Sockets.SocketError.C.R.System.Net.Sockets.SocketError}{SocketError(const System.\-Net.\-Sockets.\-SocketError\&\-)}
& Copy constructor.

\\
\hyperlink{System.Net.Sockets.SocketError.operator.assign.P.System.Net.Sockets.SocketError.C.R.System.Net.Sockets.SocketError}{operator=(const System.\-Net.\-Sockets.\-SocketError\&\-)}
& Copy assignment.

\\
\hyperlink{System.Net.Sockets.SocketError.constructor.P.System.Net.Sockets.SocketError.RR.System.Net.Sockets.SocketError}{SocketError(System.\-Net.\-Sockets.\-SocketError\&\-\&\-)}
& Move constructor.

\\
\hyperlink{System.Net.Sockets.SocketError.operator.assign.P.System.Net.Sockets.SocketError.RR.System.Net.Sockets.SocketError}{operator=(System.\-Net.\-Sockets.\-SocketError\&\-\&\-)}
& Move assignment.

\\
\hyperlink{System.Net.Sockets.SocketError.ErrorCode.C.P.System.Net.Sockets.SocketError}{ErrorCode() const}
& Returns the error code.

\\
\hyperlink{System.Net.Sockets.SocketError.constructor.P.System.Net.Sockets.SocketError.C.R.System.String.C.R.System.String.int}{SocketError(const System.\-String\&\-, const System.\-String\&\-, int)}
& Constructor. Initializes the socket error with the specified operation text, error description text and error code.

\\
\hyperlink{System.Net.Sockets.SocketError.constructor.P.System.Net.Sockets.SocketError.C.R.System.String.int}{SocketError(const System.\-String\&\-, int)}
& Constructor. Initializes the socket error with the specified operation text, retrieved error description and the specified error code.

\\
\end{supertabular}

\end{flushleft}
\clearpage

\hypertarget{System.Net.Sockets.SocketError.constructor.P.System.Net.Sockets.SocketError}{\subsubsection*{SocketError() Member Function}}\begin{flushleft}
Default constructor.

\end{flushleft}
\subsubsection*{Syntax}
\texttt{public SocketError();}
\clearpage

\hypertarget{System.Net.Sockets.SocketError.constructor.P.System.Net.Sockets.SocketError.C.R.System.Net.Sockets.SocketError}{\subsubsection*{SocketError(const System.Net.Sockets.SocketError\&) Member Function}}\begin{flushleft}
Copy constructor.

\end{flushleft}
\subsubsection*{Syntax}
\texttt{public SocketError(const System.Net.Sockets.SocketError\& \_\_parameter0);}
\subsubsection*{Parameters}
\begin{flushleft}
\begin{supertabular}[l]{!{\raggedright}p{2.61335cm}!{\raggedright}p{7.25332cm}!{\raggedright}p{2.65538cm}}
\textbf{Name}
& \textbf{Type}
& \textbf{Description}
\\
\hline
\_\_parameter0
& \hyperlink{System.Net.Sockets.SocketError}{const System.\-Net.\-Sockets.\-SocketError\&\-}
& 
\\
\end{supertabular}

\end{flushleft}
\clearpage

\hypertarget{System.Net.Sockets.SocketError.operator.assign.P.System.Net.Sockets.SocketError.C.R.System.Net.Sockets.SocketError}{\subsubsection*{operator=(const System.Net.Sockets.SocketError\&) Member Function}}\begin{flushleft}
Copy assignment.

\end{flushleft}
\subsubsection*{Syntax}
\texttt{public void operator=(const System.Net.Sockets.SocketError\& that);}
\subsubsection*{Parameters}
\begin{flushleft}
\begin{supertabular}[l]{!{\raggedright}p{1.30721cm}!{\raggedright}p{8.55945cm}!{\raggedright}p{4.89046cm}}
\textbf{Name}
& \textbf{Type}
& \textbf{Description}
\\
\hline
that
& \hyperlink{System.Net.Sockets.SocketError}{const System.\-Net.\-Sockets.\-SocketError\&\-}
& Argument to assign.

\\
\end{supertabular}

\end{flushleft}
\clearpage

\hypertarget{System.Net.Sockets.SocketError.constructor.P.System.Net.Sockets.SocketError.RR.System.Net.Sockets.SocketError}{\subsubsection*{SocketError(System.Net.Sockets.SocketError\&\&) Member Function}}\begin{flushleft}
Move constructor.

\end{flushleft}
\subsubsection*{Syntax}
\texttt{public SocketError(System.Net.Sockets.SocketError\&\& that);}
\subsubsection*{Parameters}
\begin{flushleft}
\begin{supertabular}[l]{!{\raggedright}p{1.30721cm}!{\raggedright}p{8.55945cm}!{\raggedright}p{4.93333cm}}
\textbf{Name}
& \textbf{Type}
& \textbf{Description}
\\
\hline
that
& \hyperlink{System.Net.Sockets.SocketError}{System.\-Net.\-Sockets.\-SocketError\&\-\&\-}
& Argument to move from.

\\
\end{supertabular}

\end{flushleft}
\clearpage

\hypertarget{System.Net.Sockets.SocketError.operator.assign.P.System.Net.Sockets.SocketError.RR.System.Net.Sockets.SocketError}{\subsubsection*{operator=(System.Net.Sockets.SocketError\&\&) Member Function}}\begin{flushleft}
Move assignment.

\end{flushleft}
\subsubsection*{Syntax}
\texttt{public void operator=(System.Net.Sockets.SocketError\&\& that);}
\subsubsection*{Parameters}
\begin{flushleft}
\begin{supertabular}[l]{!{\raggedright}p{1.30721cm}!{\raggedright}p{8.55945cm}!{\raggedright}p{4.93333cm}}
\textbf{Name}
& \textbf{Type}
& \textbf{Description}
\\
\hline
that
& \hyperlink{System.Net.Sockets.SocketError}{System.\-Net.\-Sockets.\-SocketError\&\-\&\-}
& Argument to assign from.

\\
\end{supertabular}

\end{flushleft}
\clearpage

\hypertarget{System.Net.Sockets.SocketError.ErrorCode.C.P.System.Net.Sockets.SocketError}{\subsubsection*{ErrorCode() const Member Function}}\begin{flushleft}
Returns the error code.

\end{flushleft}

\subsubsection*{Syntax}\texttt{public int ErrorCode() const;}
\subsubsection*{Returns}int
\begin{flushleft}
Returns the error code.

\end{flushleft}
\clearpage

\hypertarget{System.Net.Sockets.SocketError.constructor.P.System.Net.Sockets.SocketError.C.R.System.String.C.R.System.String.int}{\subsubsection*{SocketError(const System.String\&, const System.String\&, int) Member Function}}
\begin{flushleft}
Constructor. Initializes the socket error with the specified operation text, error description text and error code.

\end{flushleft}
\subsubsection*{Syntax}
\texttt{public SocketError(const System.String\& operation, const System.String\& errorMessage, int errorCode\_);}
\subsubsection*{Parameters}
\begin{flushleft}
\begin{supertabular}[l]{!{\raggedright}p{3.02071cm}!{\raggedright}p{4.63128cm}!{\raggedright}p{5.23538cm}}
\textbf{Name}
& \textbf{Type}
& \textbf{Description}
\\
\hline
operation
& const System.\-String\&\-
& Description of the failed operation.

\\
errorMessage
& const System.\-String\&\-
& Description of the error.

\\
errorCode\_
& int
& Error code.

\\
\end{supertabular}

\end{flushleft}
\clearpage

\hypertarget{System.Net.Sockets.SocketError.constructor.P.System.Net.Sockets.SocketError.C.R.System.String.int}{\subsubsection*{SocketError(const System.String\&, int) Member Function}}
\begin{flushleft}
Constructor. Initializes the socket error with the specified operation text, retrieved error description and the specified error code.

\end{flushleft}
\subsubsection*{Syntax}
\texttt{public SocketError(const System.String\& operation, int errorCode\_);}
\subsubsection*{Parameters}
\begin{flushleft}
\begin{supertabular}[l]{lll}
\textbf{Name}
& \textbf{Type}
& \textbf{Description}
\\
\hline
operation
& const System.\-String\&\-
& Description of failed operation.

\\
errorCode\_
& int
& Error code.

\\
\end{supertabular}

\end{flushleft}
\clearpage

\hypertarget{System.Net.Sockets.SocketLibrary}{\section{SocketLibrary Class}}
\begin{flushleft}
Represents the socket library initializer implemented as a singleton.

\end{flushleft}
\subsection*{Syntax}\texttt{public class SocketLibrary;}

\subsection{Member Functions}
\begin{flushleft}
\begin{supertabular}[l]{!{\raggedright}p{4.19644cm}!{\raggedright}p{10.9536cm}}
\textbf{Member Function}
& \textbf{Description}
\\
\hline
\hyperlink{System.Net.Sockets.SocketLibrary.Init.P.System.Net.Sockets.SocketLibrary}{Init()}
& Initializes the socket library.

\\
\hyperlink{System.Net.Sockets.SocketLibrary.Instance}{Instance()}
& Returns a reference to the socket library singleton instance.

\\
\hyperlink{System.Net.Sockets.SocketLibrary.destructor.P.System.Net.Sockets.SocketLibrary}{$\sim$SocketLibrary()}
& Destructor. Uninitializes the socket library.

\\
\end{supertabular}

\end{flushleft}
\clearpage

\hypertarget{System.Net.Sockets.SocketLibrary.Init.P.System.Net.Sockets.SocketLibrary}{\subsubsection*{Init() Member Function}}\begin{flushleft}
Initializes the socket library.

\end{flushleft}

\subsubsection*{Syntax}\texttt{public void Init();}
\clearpage

\hypertarget{System.Net.Sockets.SocketLibrary.Instance}{\subsubsection*{Instance() Member Function}}
\begin{flushleft}
Returns a reference to the socket library singleton instance.

\end{flushleft}
\subsubsection*{Syntax}
\texttt{public static System.Net.Sockets.SocketLibrary\& Instance();}
\subsubsection*{Returns}
\hyperlink{System.Net.Sockets.SocketLibrary}{System.\-Net.\-Sockets.\-SocketLibrary\&\-}
\begin{flushleft}
Returns a reference to the socket library singleton instance.

\end{flushleft}
\clearpage

\hypertarget{System.Net.Sockets.SocketLibrary.destructor.P.System.Net.Sockets.SocketLibrary}{\subsubsection*{$\sim$SocketLibrary() Member Function}}
\begin{flushleft}
Destructor. Uninitializes the socket library.

\end{flushleft}
\subsubsection*{Syntax}\texttt{public $\sim$SocketLibrary();}
\clearpage

\hypertarget{System.Net.Sockets.SocketLibraryException}{\section{SocketLibraryException Class}}
\begin{flushleft}
Exception class thrown when the initialization of the socket library fails.

\end{flushleft}
\subsection*{Syntax}\texttt{public class SocketLibraryException;}

\subsection*{Base Class}System.Exception\subsection{Member Functions}
\begin{flushleft}
\begin{supertabular}[l]{!{\raggedright}p{7.575cm}!{\raggedright}p{7.575cm}}
\textbf{Member Function}
& \textbf{Description}
\\
\hline
\hyperlink{System.Net.Sockets.SocketLibraryException.constructor.P.System.Net.Sockets.SocketLibraryException}{SocketLibraryException()}
& Default constructor.

\\
\hyperlink{System.Net.Sockets.SocketLibraryException.constructor.P.System.Net.Sockets.SocketLibraryException.C.R.System.Net.Sockets.SocketLibraryException}{SocketLibraryException(const System.\-Net.\-Sockets.\-SocketLibraryException\&\-)}
& Copy constructor.

\\
\hyperlink{System.Net.Sockets.SocketLibraryException.operator.assign.P.System.Net.Sockets.SocketLibraryException.C.R.System.Net.Sockets.SocketLibraryException}{operator=(const System.\-Net.\-Sockets.\-SocketLibraryException\&\-)}
& Copy assignment.

\\
\hyperlink{System.Net.Sockets.SocketLibraryException.constructor.P.System.Net.Sockets.SocketLibraryException.RR.System.Net.Sockets.SocketLibraryException}{SocketLibraryException(System.\-Net.\-Sockets.\-SocketLibraryException\&\-\&\-)}
& Move constructor.

\\
\hyperlink{System.Net.Sockets.SocketLibraryException.operator.assign.P.System.Net.Sockets.SocketLibraryException.RR.System.Net.Sockets.SocketLibraryException}{operator=(System.\-Net.\-Sockets.\-SocketLibraryException\&\-\&\-)}
& Move assignment.

\\
\hyperlink{System.Net.Sockets.SocketLibraryException.constructor.P.System.Net.Sockets.SocketLibraryException.C.R.System.String}{SocketLibraryException(const System.\-String\&\-)}
& Constructor. Initializes the socket library exception with the specified error message.

\\
\end{supertabular}

\end{flushleft}
\clearpage

\hypertarget{System.Net.Sockets.SocketLibraryException.constructor.P.System.Net.Sockets.SocketLibraryException}{\subsubsection*{SocketLibraryException() Member Function}}\begin{flushleft}
Default constructor.

\end{flushleft}
\subsubsection*{Syntax}
\texttt{public SocketLibraryException();}
\clearpage

\hypertarget{System.Net.Sockets.SocketLibraryException.constructor.P.System.Net.Sockets.SocketLibraryException.C.R.System.Net.Sockets.SocketLibraryException}{\subsubsection*{SocketLibraryException(const System.Net.Sockets.SocketLibraryException\&) Member Function}}\begin{flushleft}
Copy constructor.

\end{flushleft}
\subsubsection*{Syntax}
\texttt{public SocketLibraryException(const System.Net.Sockets.SocketLibraryException\& \_\_parameter0);}
\subsubsection*{Parameters}
\begin{flushleft}
\begin{supertabular}[l]{!{\raggedright}p{2.61335cm}!{\raggedright}p{7.25332cm}!{\raggedright}p{2.65538cm}}
\textbf{Name}
& \textbf{Type}
& \textbf{Description}
\\
\hline
\_\_parameter0
& \hyperlink{System.Net.Sockets.SocketLibraryException}{const System.\-Net.\-Sockets.\-SocketLibraryException\&\-}
& 
\\
\end{supertabular}

\end{flushleft}
\clearpage

\hypertarget{System.Net.Sockets.SocketLibraryException.operator.assign.P.System.Net.Sockets.SocketLibraryException.C.R.System.Net.Sockets.SocketLibraryException}{\subsubsection*{operator=(const System.Net.Sockets.SocketLibraryException\&) Member Function}}\begin{flushleft}
Copy assignment.

\end{flushleft}
\subsubsection*{Syntax}
\texttt{public void operator=(const System.Net.Sockets.SocketLibraryException\& that);}
\subsubsection*{Parameters}
\begin{flushleft}
\begin{supertabular}[l]{!{\raggedright}p{1.30721cm}!{\raggedright}p{8.55945cm}!{\raggedright}p{4.89046cm}}
\textbf{Name}
& \textbf{Type}
& \textbf{Description}
\\
\hline
that
& \hyperlink{System.Net.Sockets.SocketLibraryException}{const System.\-Net.\-Sockets.\-SocketLibraryException\&\-}
& Argument to assign.

\\
\end{supertabular}

\end{flushleft}
\clearpage

\hypertarget{System.Net.Sockets.SocketLibraryException.constructor.P.System.Net.Sockets.SocketLibraryException.RR.System.Net.Sockets.SocketLibraryException}{\subsubsection*{SocketLibraryException(System.Net.Sockets.SocketLibraryException\&\&) Member Function}}\begin{flushleft}
Move constructor.

\end{flushleft}
\subsubsection*{Syntax}
\texttt{public SocketLibraryException(System.Net.Sockets.SocketLibraryException\&\& that);}
\subsubsection*{Parameters}
\begin{flushleft}
\begin{supertabular}[l]{!{\raggedright}p{1.30721cm}!{\raggedright}p{8.55945cm}!{\raggedright}p{4.93333cm}}
\textbf{Name}
& \textbf{Type}
& \textbf{Description}
\\
\hline
that
& \hyperlink{System.Net.Sockets.SocketLibraryException}{System.\-Net.\-Sockets.\-SocketLibraryException\&\-\&\-}
& Argument to move from.

\\
\end{supertabular}

\end{flushleft}
\clearpage

\hypertarget{System.Net.Sockets.SocketLibraryException.operator.assign.P.System.Net.Sockets.SocketLibraryException.RR.System.Net.Sockets.SocketLibraryException}{\subsubsection*{operator=(System.Net.Sockets.SocketLibraryException\&\&) Member Function}}\begin{flushleft}
Move assignment.

\end{flushleft}
\subsubsection*{Syntax}
\texttt{public void operator=(System.Net.Sockets.SocketLibraryException\&\& that);}
\subsubsection*{Parameters}
\begin{flushleft}
\begin{supertabular}[l]{!{\raggedright}p{1.30721cm}!{\raggedright}p{8.55945cm}!{\raggedright}p{4.93333cm}}
\textbf{Name}
& \textbf{Type}
& \textbf{Description}
\\
\hline
that
& \hyperlink{System.Net.Sockets.SocketLibraryException}{System.\-Net.\-Sockets.\-SocketLibraryException\&\-\&\-}
& Argument to assign from.

\\
\end{supertabular}

\end{flushleft}
\clearpage

\hypertarget{System.Net.Sockets.SocketLibraryException.constructor.P.System.Net.Sockets.SocketLibraryException.C.R.System.String}{\subsubsection*{SocketLibraryException(const System.String\&) Member Function}}
\begin{flushleft}
Constructor. Initializes the socket library exception with the specified error message.

\end{flushleft}
\subsubsection*{Syntax}
\texttt{public SocketLibraryException(const System.String\& message\_);}
\subsubsection*{Parameters}
\begin{flushleft}
\begin{supertabular}[l]{lll}
\textbf{Name}
& \textbf{Type}
& \textbf{Description}
\\
\hline
message\_
& const System.\-String\&\-
& An error message.

\\
\end{supertabular}

\end{flushleft}
\clearpage
\hypertarget{System.Net.Sockets.TcpSocket}{\section{TcpSocket Class}}
\begin{flushleft}
Represents a TCP socket.

\end{flushleft}
\subsection*{Syntax}\texttt{public class TcpSocket;}

\subsection{Member Functions}
\begin{flushleft}
\begin{supertabular}[l]{!{\raggedright}p{7.575cm}!{\raggedright}p{7.575cm}}
\textbf{Member Function}
& \textbf{Description}
\\
\hline
\hyperlink{System.Net.Sockets.TcpSocket.constructor.P.System.Net.Sockets.TcpSocket}{TcpSocket()}
& Default constructor. Creates an unbound TCP socket.

\\
\hyperlink{System.Net.Sockets.TcpSocket.constructor.P.System.Net.Sockets.TcpSocket.RR.System.Net.Sockets.TcpSocket}{TcpSocket(System.\-Net.\-Sockets.\-TcpSocket\&\-\&\-)}
& Move constructor.

\\
\hyperlink{System.Net.Sockets.TcpSocket.operator.assign.P.System.Net.Sockets.TcpSocket.RR.System.Net.Sockets.TcpSocket}{operator=(System.\-Net.\-Sockets.\-TcpSocket\&\-\&\-)}
& Move assignment.

\\
\hyperlink{System.Net.Sockets.TcpSocket.Accept.P.System.Net.Sockets.TcpSocket}{Accept()}
& Accepts a connection to a bound socket and returns a new connected TCP socket that represents the connection.

\\
\hyperlink{System.Net.Sockets.TcpSocket.Bind.P.System.Net.Sockets.TcpSocket.int}{Bind(int)}
& Binds the socket to a port.

\\
\hyperlink{System.Net.Sockets.TcpSocket.Close.P.System.Net.Sockets.TcpSocket}{Close()}
& Closes the socket.

\\
\hyperlink{System.Net.Sockets.TcpSocket.GetSocketHandle.C.P.System.Net.Sockets.TcpSocket}{GetSocketHandle() const}
& Returns the socket handle.

\\
\hyperlink{System.Net.Sockets.TcpSocket.Listen.P.System.Net.Sockets.TcpSocket.int}{Listen(int)}
& Begins listening connections to a bound TCP socket.

\\
\hyperlink{System.Net.Sockets.TcpSocket.Receive.P.System.Net.Sockets.TcpSocket.P.void.int}{Receive(void*, int)}
& Receives data from a connected socket.

\\
\hyperlink{System.Net.Sockets.TcpSocket.ReceiveAll.P.System.Net.Sockets.TcpSocket}{ReceiveAll()}
& Receives rest of data from a connected socket.
That is: receives data until the peer shuts down its sending side of the connection.

\\
\hyperlink{System.Net.Sockets.TcpSocket.Send.P.System.Net.Sockets.TcpSocket.C.R.System.String}{Send(const System.\-String\&\-)}
& Sends a string of data to a connected socket.

\\
\hyperlink{System.Net.Sockets.TcpSocket.Send.P.System.Net.Sockets.TcpSocket.C.P.void.int}{Send(const void*, int)}
& Sends data to a connected socket.

\\
\hyperlink{System.Net.Sockets.TcpSocket.Shutdown.P.System.Net.Sockets.TcpSocket.ShutdownMode}{Shutdown(ShutdownMode)}
& Shuts down a connected socket.

\\
\hyperlink{System.Net.Sockets.TcpSocket.constructor.P.System.Net.Sockets.TcpSocket.C.R.System.String.C.R.System.String}{TcpSocket(const System.\-String\&\-, const System.\-String\&\-)}
& Constructor. Creates a TCP socket and connects it to the specified node and service.

\\
\hyperlink{System.Net.Sockets.TcpSocket.constructor.P.System.Net.Sockets.TcpSocket.int}{TcpSocket(int)}
& Constructor. Initializes a TCP socket with an existing socket handle.

\\
\hyperlink{System.Net.Sockets.TcpSocket.destructor.P.System.Net.Sockets.TcpSocket}{$\sim$TcpSocket()}
& Destructor. Closes the socket if it is bound or connected.

\\
\end{supertabular}

\end{flushleft}
\clearpage

\hypertarget{System.Net.Sockets.TcpSocket.constructor.P.System.Net.Sockets.TcpSocket}{\subsubsection*{TcpSocket() Member Function}}
\begin{flushleft}
Default constructor. Creates an unbound TCP socket.

\end{flushleft}
\subsubsection*{Syntax}\texttt{public TcpSocket();}
\clearpage

\hypertarget{System.Net.Sockets.TcpSocket.constructor.P.System.Net.Sockets.TcpSocket.RR.System.Net.Sockets.TcpSocket}{\subsubsection*{TcpSocket(System.Net.Sockets.TcpSocket\&\&) Member Function}}\begin{flushleft}
Move constructor.

\end{flushleft}
\subsubsection*{Syntax}
\texttt{public TcpSocket(System.Net.Sockets.TcpSocket\&\& that);}
\subsubsection*{Parameters}
\begin{flushleft}
\begin{supertabular}[l]{!{\raggedright}p{1.30721cm}!{\raggedright}p{8.55945cm}!{\raggedright}p{2.65538cm}}
\textbf{Name}
& \textbf{Type}
& \textbf{Description}
\\
\hline
that
& \hyperlink{System.Net.Sockets.TcpSocket}{System.\-Net.\-Sockets.\-TcpSocket\&\-\&\-}
& 
\\
\end{supertabular}

\end{flushleft}
\clearpage

\hypertarget{System.Net.Sockets.TcpSocket.operator.assign.P.System.Net.Sockets.TcpSocket.RR.System.Net.Sockets.TcpSocket}{\subsubsection*{operator=(System.Net.Sockets.TcpSocket\&\&) Member Function}}\begin{flushleft}
Move assignment.

\end{flushleft}
\subsubsection*{Syntax}
\texttt{public void operator=(System.Net.Sockets.TcpSocket\&\& that);}
\subsubsection*{Parameters}
\begin{flushleft}
\begin{supertabular}[l]{!{\raggedright}p{1.30721cm}!{\raggedright}p{8.55945cm}!{\raggedright}p{2.65538cm}}
\textbf{Name}
& \textbf{Type}
& \textbf{Description}
\\
\hline
that
& \hyperlink{System.Net.Sockets.TcpSocket}{System.\-Net.\-Sockets.\-TcpSocket\&\-\&\-}
& 
\\
\end{supertabular}

\end{flushleft}
\clearpage

\hypertarget{System.Net.Sockets.TcpSocket.Accept.P.System.Net.Sockets.TcpSocket}{\subsubsection*{Accept() Member Function}}
\begin{flushleft}
Accepts a connection to a bound socket and returns a new connected TCP socket that represents the connection.

\end{flushleft}
\subsubsection*{Syntax}\texttt{public System.Net.Sockets.TcpSocket Accept();}

\subsubsection*{Returns}\hyperlink{System.Net.Sockets.TcpSocket}{System.\-Net.\-Sockets.\-TcpSocket}
\begin{flushleft}
Returns a connected TCP socket that represents the connection.

\end{flushleft}
\clearpage

\hypertarget{System.Net.Sockets.TcpSocket.Bind.P.System.Net.Sockets.TcpSocket.int}{\subsubsection*{Bind(int) Member Function}}\begin{flushleft}
Binds the socket to a port.

\end{flushleft}

\subsubsection*{Syntax}\texttt{public void Bind(int port);}
\subsubsection*{Parameters}
\begin{flushleft}
\begin{supertabular}[l]{lll}
\textbf{Name}
& \textbf{Type}
& \textbf{Description}
\\
\hline
port
& int
& A port number to which to bind.

\\
\end{supertabular}

\end{flushleft}
\clearpage

\hypertarget{System.Net.Sockets.TcpSocket.Close.P.System.Net.Sockets.TcpSocket}{\subsubsection*{Close() Member Function}}\begin{flushleft}
Closes the socket.

\end{flushleft}
\subsubsection*{Syntax}
\texttt{public void Close();}
\clearpage

\hypertarget{System.Net.Sockets.TcpSocket.GetSocketHandle.C.P.System.Net.Sockets.TcpSocket}{\subsubsection*{GetSocketHandle() const Member Function}}\begin{flushleft}
Returns the socket handle.

\end{flushleft}

\subsubsection*{Syntax}\texttt{public int GetSocketHandle() const;}
\subsubsection*{Returns}int
\begin{flushleft}
Returns the socket handle.

\end{flushleft}
\clearpage

\hypertarget{System.Net.Sockets.TcpSocket.Listen.P.System.Net.Sockets.TcpSocket.int}{\subsubsection*{Listen(int) Member Function}}
\begin{flushleft}
Begins listening connections to a bound TCP socket.

\end{flushleft}
\subsubsection*{Syntax}\texttt{public void Listen(int backlog);}

\subsubsection*{Parameters}
\begin{flushleft}
\begin{supertabular}[l]{lll}
\textbf{Name}
& \textbf{Type}
& \textbf{Description}
\\
\hline
backlog
& int
& The number of pending connections.

\\
\end{supertabular}

\end{flushleft}
\clearpage

\hypertarget{System.Net.Sockets.TcpSocket.Receive.P.System.Net.Sockets.TcpSocket.P.void.int}{\subsubsection*{Receive(void*, int) Member Function}}\begin{flushleft}
Receives data from a connected socket.

\end{flushleft}

\subsubsection*{Syntax}\texttt{public int Receive(void* buf, int len);}
\subsubsection*{Parameters}
\begin{flushleft}
\begin{supertabular}[l]{lll}
\textbf{Name}
& \textbf{Type}
& \textbf{Description}
\\
\hline
buf
& void*
& A buffer.

\\
len
& int
& Maximum number of bytes to receive.

\\
\end{supertabular}

\end{flushleft}
\subsubsection*{Returns}int
\begin{flushleft}
Returns the number of bytes received.
This might be less than the number of bytes requested.
Return value of 0 indicates that the peer has shut down the connection.

\end{flushleft}
\subsubsection*{Remarks}
\begin{flushleft}
Throws \hyperlink{System.Net.Sockets.SocketError}{SocketError} if reading fails.

\end{flushleft}
\clearpage

\hypertarget{System.Net.Sockets.TcpSocket.ReceiveAll.P.System.Net.Sockets.TcpSocket}{\subsubsection*{ReceiveAll() Member Function}}
\begin{flushleft}
Receives rest of data from a connected socket.
That is: receives data until the peer shuts down its sending side of the connection.

\end{flushleft}
\subsubsection*{Syntax}\texttt{public System.String ReceiveAll();}

\subsubsection*{Returns}System.\-String
\begin{flushleft}
Returns the received data as a string.

\end{flushleft}
\subsubsection*{Remarks}
\begin{flushleft}
Throws \hyperlink{System.Net.Sockets.SocketError}{SocketError} if reading fails.

\end{flushleft}
\clearpage

\hypertarget{System.Net.Sockets.TcpSocket.Send.P.System.Net.Sockets.TcpSocket.C.R.System.String}{\subsubsection*{Send(const System.String\&) Member Function}}
\begin{flushleft}
Sends a string of data to a connected socket.

\end{flushleft}
\subsubsection*{Syntax}\texttt{public void Send(const System.String\& s);}

\subsubsection*{Parameters}
\begin{flushleft}
\begin{supertabular}[l]{lll}
\textbf{Name}
& \textbf{Type}
& \textbf{Description}
\\
\hline
s
& const System.\-String\&\-
& A string to send.

\\
\end{supertabular}

\end{flushleft}
\subsubsection*{Remarks}
\begin{flushleft}
Throws \hyperlink{System.Net.Sockets.SocketError}{SocketError} if sending fails.

\end{flushleft}
\clearpage

\hypertarget{System.Net.Sockets.TcpSocket.Send.P.System.Net.Sockets.TcpSocket.C.P.void.int}{\subsubsection*{Send(const void*, int) Member Function}}\begin{flushleft}
Sends data to a connected socket.

\end{flushleft}

\subsubsection*{Syntax}\texttt{public int Send(const void* buf, int len);}
\subsubsection*{Parameters}
\begin{flushleft}
\begin{supertabular}[l]{lll}
\textbf{Name}
& \textbf{Type}
& \textbf{Description}
\\
\hline
buf
& const void*
& A buffer of data to send.

\\
len
& int
& Maximum number of bytes to send.

\\
\end{supertabular}

\end{flushleft}
\subsubsection*{Returns}int
\begin{flushleft}
Returns the number of bytes sent.
This might be less than the number of bytes requested.

\end{flushleft}
\subsubsection*{Remarks}
\begin{flushleft}
Throws \hyperlink{System.Net.Sockets.SocketError}{SocketError} if sending fails.

\end{flushleft}
\clearpage

\hypertarget{System.Net.Sockets.TcpSocket.Shutdown.P.System.Net.Sockets.TcpSocket.ShutdownMode}{\subsubsection*{Shutdown(ShutdownMode) Member Function}}\begin{flushleft}
Shuts down a connected socket.

\end{flushleft}

\subsubsection*{Syntax}\texttt{public void Shutdown(ShutdownMode mode);}
\subsubsection*{Parameters}
\begin{flushleft}
\begin{supertabular}[l]{lll}
\textbf{Name}
& \textbf{Type}
& \textbf{Description}
\\
\hline
mode
& \hyperlink{ShutdownMode}{ShutdownMode}
& Shut down mode.

\\
\end{supertabular}

\end{flushleft}
\clearpage

\hypertarget{System.Net.Sockets.TcpSocket.constructor.P.System.Net.Sockets.TcpSocket.C.R.System.String.C.R.System.String}{\subsubsection*{TcpSocket(const System.String\&, const System.String\&) Member Function}}
\begin{flushleft}
Constructor. Creates a TCP socket and connects it to the specified node and service.

\end{flushleft}
\subsubsection*{Syntax}
\texttt{public TcpSocket(const System.String\& node, const System.String\& service);}
\subsubsection*{Parameters}
\begin{flushleft}
\begin{supertabular}[l]{!{\raggedright}p{1.56801cm}!{\raggedright}p{4.63128cm}!{\raggedright}p{5.23538cm}}
\textbf{Name}
& \textbf{Type}
& \textbf{Description}
\\
\hline
node
& const System.\-String\&\-
& A host name or an IP address to connect.

\\
service
& const System.\-String\&\-
& A protocol name or port number to connect.

\\
\end{supertabular}

\end{flushleft}
\clearpage

\hypertarget{System.Net.Sockets.TcpSocket.constructor.P.System.Net.Sockets.TcpSocket.int}{\subsubsection*{TcpSocket(int) Member Function}}
\begin{flushleft}
Constructor. Initializes a TCP socket with an existing socket handle.

\end{flushleft}
\subsubsection*{Syntax}\texttt{public TcpSocket(int socket\_);}

\subsubsection*{Parameters}
\begin{flushleft}
\begin{supertabular}[l]{lll}
\textbf{Name}
& \textbf{Type}
& \textbf{Description}
\\
\hline
socket\_
& int
& A handle of an existing TCP socket.

\\
\end{supertabular}

\end{flushleft}
\clearpage

\hypertarget{System.Net.Sockets.TcpSocket.destructor.P.System.Net.Sockets.TcpSocket}{\subsubsection*{$\sim$TcpSocket() Member Function}}
\begin{flushleft}
Destructor. Closes the socket if it is bound or connected.

\end{flushleft}
\subsubsection*{Syntax}\texttt{public $\sim$TcpSocket();}
\clearpage

\subchapter{Constants}
\begin{flushleft}
\begin{supertabular}[l]{llll}
\textbf{Constant}
& \textbf{Type}
& \textbf{Value}
& \textbf{Description}
\\
\hline
\hypertarget{System.Net.Sockets.invalidSocketHandle}{invalidSocketHandle}
& int
& -1
& Represents invalid socket handle.

\\
\end{supertabular}

\end{flushleft}
\clearpage
\end{document}


