\documentclass[a4paper,oneside,11.000000pt]{book}

\usepackage{array}
\usepackage{amsmath}
\usepackage{amsfonts}
\usepackage{amssymb}
\usepackage{graphicx}
\usepackage{pifont}
\usepackage{a4wide}
\usepackage{url}
\usepackage{float}
\usepackage{layouts}
\usepackage{supertabular}
\usepackage{titlesec}
\usepackage{listings}
\usepackage{syntax}
\usepackage[colorlinks=true,linkcolor=blue]{hyperref}
\setcounter{secnumdepth}{10}
\setcounter{tocdepth}{4}

\titleformat{\chapter}[hang]{\bf\Huge}{\arabic{chapter}}{1em}{}{}
\titleclass{\subchapter}{straight}[\chapter]
\newcounter{subchapter}
\renewcommand{\thesubchapter}{\thechapter.\arabic{subchapter}}
\titleformat{\subchapter}[hang]{\bf\huge}{\thesubchapter}{0.5em}{}{}
\titlespacing{\subchapter}{0pt}{0pt}{0pt}
\renewcommand\thesection{\thesubchapter.\arabic{section}}
\makeatletter
\newcommand*\l@subchapter{\@dottedtocline{1}{1.5em}{2.3em}}
\renewcommand\l@section{\@dottedtocline{2}{3.8em}{3.2em}}
\renewcommand\l@subsection{\@dottedtocline{3}{7em}{4.1em}}
\renewcommand\l@subsubsection{\@dottedtocline{4}{11.1em}{5em}}
\makeatother

\makeatletter
\renewcommand\subparagraph{\@startsection{subparagraph}{5}{0mm}{-\baselineskip}{0.5\baselineskip}{\normalfont\normalsize\scshape}}
\makeatother

\lstdefinelanguage{Cmajor}{morekeywords={bool, true, false, sbyte, byte, short, ushort, int, uint, long, ulong, float, double, char, void, enum, cast, namespace, using, static, extern, is, explicit, delegate, inline, cdecl, nothrow, public, protected, private, internal, virtual, abstract, override, suppress, default, operator, class, return, if, else, switch, case, default, while, do, for, break, continue, goto, typedef, typename, const, null, this, base, construct, destroy, new, delete, sizeof, try, catch, throw, concept, where, axiom, and, or, not},sensitive=true,morecomment=[l]{//},morecomment=[s]{/*}{*/},morestring=[b]",morestring=[b]',}\lstloadlanguages{Cmajor}
\lstset{language=Cmajor,showstringspaces=false,breaklines=true,basicstyle=\small}
\makeatletter\def\toclevel@chapter{0}\def\toclevel@subchapter{1}\def\toclevel@section{2}\def\toclevel@subsection{3}\def\toclevel@subsubsection{4}\makeatother\begin{document}
\clearpage

\frontmatter
\title{\textsc{System.IO.Compression Library Reference}
}
\date{\today}
\maketitle
\tableofcontents

\clearpage
\chapter{Description}
\begin{flushleft}
The \hyperlink{System.IO.Compression}{Compression} library contains streams that provide compression and decompression services.
The library is implemented using \textbf{zlib} (\url{http://wwww.zlib.net}) and
\textbf{bzip2} (\url{http://www.bzip.org/}) libraries.

\end{flushleft}
\chapter{Namespaces}
\begin{flushleft}
\begin{supertabular}[l]{!{\raggedright}p{5.49396cm}!{\raggedright}p{9.65604cm}}
\textbf{Namespace}
& \textbf{Description}
\\
\hline
\hyperlink{System.IO.Compression}{System.IO.Compression}
& The \hyperlink{System.IO.Compression}{Compression} namespace contains stream classes that provide data compression and decompression.

\\
\end{supertabular}

\end{flushleft}
\clearpage
\mainmatter

\chapter{Usage}

\section{Referencing the System.IO.Compression library}

Right-click a project node in IDE \verb.|. Project References... \verb.|.
Add System Extension Library Reference... \verb.|.
enable \emph{System.IO.Compression} check box

\begin{flushleft}
or add following line to your project's .cmp file:
\begin{verbatim}
reference <ext/System.IO.Compression/System.IO.Compression.cml>;
\end{verbatim}
\end{flushleft}
\hypertarget{System.IO.Compression}{\chapter{System.IO.Compression Namespace}}
\begin{flushleft}
The \textbf{Compression}
 namespace contains stream classes that provide data compression and decompression.

\end{flushleft}
\clearpage
\subchapter{Classes}
\begin{flushleft}
\begin{supertabular}[l]{!{\raggedright}p{3.97714cm}!{\raggedright}p{11.1729cm}}
\textbf{Class}
& \textbf{Description}
\\
\hline
\hyperlink{System.IO.Compression.BZip2Exception}{BZip2Exception}
& Exception class thrown when \hyperlink{System.IO.Compression.BZip2Stream}{BZip2Stream} cannot compress or decompress data.

\\
\hyperlink{System.IO.Compression.BZip2Stream}{BZip2Stream}
& A stream class that writes data to the underlying byte stream in bzip2 compression format, or reads data compressed in bzip2 format from the underlying byte stream and decompresses it.

\\
\hyperlink{System.IO.Compression.DeflateException}{DeflateException}
& Exception class thrown when \hyperlink{System.IO.Compression.DeflateStream}{DeflateStream} cannot compress or decompress data.

\\
\hyperlink{System.IO.Compression.DeflateStream}{DeflateStream}
& A stream class that writes data to the underlying byte stream in ZLIB compression format, or reads data compressed in ZLIB format from the underlying byte stream and decompresses it.

\\
\end{supertabular}

\end{flushleft}
\clearpage

\hypertarget{System.IO.Compression.BZip2Exception}{\section{BZip2Exception Class}}
\begin{flushleft}
Exception class thrown when \hyperlink{System.IO.Compression.BZip2Stream}{BZip2Stream} cannot compress or decompress data.

\end{flushleft}
\subsection*{Syntax}\texttt{public class BZip2Exception;}

\subsection*{Base Class}System.Exception\subsection{Member Functions}
\begin{flushleft}
\begin{supertabular}[l]{!{\raggedright}p{7.575cm}!{\raggedright}p{7.575cm}}
\textbf{Member Function}
& \textbf{Description}
\\
\hline
\hyperlink{System.IO.Compression.BZip2Exception.constructor.P.System.IO.Compression.BZip2Exception}{BZip2Exception()}
& Default constructor.

\\
\hyperlink{System.IO.Compression.BZip2Exception.constructor.P.System.IO.Compression.BZip2Exception.C.R.System.IO.Compression.BZip2Exception}{BZip2Exception(const System.\-IO.\-Compression.\-BZip2Exception\&\-)}
& Copy constructor.

\\
\hyperlink{System.IO.Compression.BZip2Exception.operator.assign.P.System.IO.Compression.BZip2Exception.C.R.System.IO.Compression.BZip2Exception}{operator=(const System.\-IO.\-Compression.\-BZip2Exception\&\-)}
& Copy assignment.

\\
\hyperlink{System.IO.Compression.BZip2Exception.constructor.P.System.IO.Compression.BZip2Exception.RR.System.IO.Compression.BZip2Exception}{BZip2Exception(System.\-IO.\-Compression.\-BZip2Exception\&\-\&\-)}
& Move constructor.

\\
\hyperlink{System.IO.Compression.BZip2Exception.operator.assign.P.System.IO.Compression.BZip2Exception.RR.System.IO.Compression.BZip2Exception}{operator=(System.\-IO.\-Compression.\-BZip2Exception\&\-\&\-)}
& Move assignment.

\\
\hyperlink{System.IO.Compression.BZip2Exception.constructor.P.System.IO.Compression.BZip2Exception.C.R.System.String.int}{BZip2Exception(const System.\-String\&\-, int)}
& Constructor. Initializes the \textbf{BZip2Exception}
 with the given error message and error code.

\\
\hyperlink{System.IO.Compression.BZip2Exception.ErrorCode.C.P.System.IO.Compression.BZip2Exception}{ErrorCode() const}
& Returns the error code returned by the bzip2 library.

\\
\hyperlink{System.IO.Compression.BZip2Exception.destructor.P.System.IO.Compression.BZip2Exception}{$\sim$BZip2Exception()}
& Destructor.

\\
\end{supertabular}

\end{flushleft}
\clearpage

\hypertarget{System.IO.Compression.BZip2Exception.constructor.P.System.IO.Compression.BZip2Exception}{\subsubsection*{BZip2Exception() Member Function}}\begin{flushleft}
Default constructor.

\end{flushleft}
\subsubsection*{Syntax}
\texttt{public BZip2Exception();}
\clearpage

\hypertarget{System.IO.Compression.BZip2Exception.constructor.P.System.IO.Compression.BZip2Exception.C.R.System.IO.Compression.BZip2Exception}{\subsubsection*{BZip2Exception(const System.IO.Compression.BZip2Exception\&) Member Function}}\begin{flushleft}
Copy constructor.

\end{flushleft}
\subsubsection*{Syntax}
\texttt{public BZip2Exception(const System.IO.Compression.BZip2Exception\& that);}
\subsubsection*{Parameters}
\begin{flushleft}
\begin{supertabular}[l]{!{\raggedright}p{1.30721cm}!{\raggedright}p{8.55945cm}!{\raggedright}p{4.55908cm}}
\textbf{Name}
& \textbf{Type}
& \textbf{Description}
\\
\hline
that
& \hyperlink{System.IO.Compression.BZip2Exception}{const System.\-IO.\-Compression.\-BZip2Exception\&\-}
& Argument to copy.

\\
\end{supertabular}

\end{flushleft}
\clearpage

\hypertarget{System.IO.Compression.BZip2Exception.operator.assign.P.System.IO.Compression.BZip2Exception.C.R.System.IO.Compression.BZip2Exception}{\subsubsection*{operator=(const System.IO.Compression.BZip2Exception\&) Member Function}}\begin{flushleft}
Copy assignment.

\end{flushleft}
\subsubsection*{Syntax}
\texttt{public void operator=(const System.IO.Compression.BZip2Exception\& that);}
\subsubsection*{Parameters}
\begin{flushleft}
\begin{supertabular}[l]{!{\raggedright}p{1.30721cm}!{\raggedright}p{8.55945cm}!{\raggedright}p{4.89046cm}}
\textbf{Name}
& \textbf{Type}
& \textbf{Description}
\\
\hline
that
& \hyperlink{System.IO.Compression.BZip2Exception}{const System.\-IO.\-Compression.\-BZip2Exception\&\-}
& Argument to assign.

\\
\end{supertabular}

\end{flushleft}
\clearpage

\hypertarget{System.IO.Compression.BZip2Exception.constructor.P.System.IO.Compression.BZip2Exception.RR.System.IO.Compression.BZip2Exception}{\subsubsection*{BZip2Exception(System.IO.Compression.BZip2Exception\&\&) Member Function}}\begin{flushleft}
Move constructor.

\end{flushleft}
\subsubsection*{Syntax}
\texttt{public BZip2Exception(System.IO.Compression.BZip2Exception\&\& that);}
\subsubsection*{Parameters}
\begin{flushleft}
\begin{supertabular}[l]{!{\raggedright}p{1.30721cm}!{\raggedright}p{8.55945cm}!{\raggedright}p{4.93333cm}}
\textbf{Name}
& \textbf{Type}
& \textbf{Description}
\\
\hline
that
& \hyperlink{System.IO.Compression.BZip2Exception}{System.\-IO.\-Compression.\-BZip2Exception\&\-\&\-}
& Argument to move from.

\\
\end{supertabular}

\end{flushleft}
\clearpage

\hypertarget{System.IO.Compression.BZip2Exception.operator.assign.P.System.IO.Compression.BZip2Exception.RR.System.IO.Compression.BZip2Exception}{\subsubsection*{operator=(System.IO.Compression.BZip2Exception\&\&) Member Function}}\begin{flushleft}
Move assignment.

\end{flushleft}
\subsubsection*{Syntax}
\texttt{public void operator=(System.IO.Compression.BZip2Exception\&\& that);}
\subsubsection*{Parameters}
\begin{flushleft}
\begin{supertabular}[l]{!{\raggedright}p{1.30721cm}!{\raggedright}p{8.55945cm}!{\raggedright}p{4.93333cm}}
\textbf{Name}
& \textbf{Type}
& \textbf{Description}
\\
\hline
that
& \hyperlink{System.IO.Compression.BZip2Exception}{System.\-IO.\-Compression.\-BZip2Exception\&\-\&\-}
& Argument to assign from.

\\
\end{supertabular}

\end{flushleft}
\clearpage

\hypertarget{System.IO.Compression.BZip2Exception.constructor.P.System.IO.Compression.BZip2Exception.C.R.System.String.int}{\subsubsection*{BZip2Exception(const System.String\&, int) Member Function}}
\begin{flushleft}
Constructor. Initializes the \hyperlink{System.IO.Compression.BZip2Exception}{BZip2Exception} with the given error message and error code.

\end{flushleft}
\subsubsection*{Syntax}
\texttt{public BZip2Exception(const System.String\& message\_, int errorCode\_);}
\subsubsection*{Parameters}
\begin{flushleft}
\begin{supertabular}[l]{lll}
\textbf{Name}
& \textbf{Type}
& \textbf{Description}
\\
\hline
message\_
& const System.\-String\&\-
& Error message.

\\
errorCode\_
& int
& Error code returned by bzip2 library.

\\
\end{supertabular}

\end{flushleft}
\clearpage

\hypertarget{System.IO.Compression.BZip2Exception.ErrorCode.C.P.System.IO.Compression.BZip2Exception}{\subsubsection*{ErrorCode() const Member Function}}
\begin{flushleft}
Returns the error code returned by the bzip2 library.

\end{flushleft}
\subsubsection*{Syntax}\texttt{public int ErrorCode() const;}

\subsubsection*{Returns}int
\begin{flushleft}
Returns the error code returned by the bzip2 library.

\end{flushleft}
\clearpage

\hypertarget{System.IO.Compression.BZip2Exception.destructor.P.System.IO.Compression.BZip2Exception}{\subsubsection*{$\sim$BZip2Exception() Member Function}}\begin{flushleft}
Destructor.

\end{flushleft}
\subsubsection*{Syntax}
\texttt{public $\sim$BZip2Exception();}
\clearpage

\hypertarget{System.IO.Compression.BZip2Stream}{\section{BZip2Stream Class}}
\begin{flushleft}
A stream class that writes data to the underlying byte stream in bzip2 compression format, or reads data compressed in bzip2 format from the underlying byte stream and decompresses it.

\end{flushleft}
\subsection*{Syntax}\texttt{public class BZip2Stream;}
\subsection*{Base Class}
System.IO.ByteStream
\subsection{Example}
\lstset{frameround=fttt}\begin{lstlisting}[frame=trBL]
using System;
using System.IO;
using System.IO.Compression;

int main()
{
    try
    {
        {
            FileByteStream in("bzip2/test.file", FileMode.open);
            FileByteStream out("bzip2/test.bz2", FileMode.create);
            BZip2Stream compressStream(out, CompressionMode.compress);
            in.CopyTo(compressStream);
        }    
        {
            FileByteStream in("bzip2/test.bz2", FileMode.open);
            BZip2Stream decompressStream(in, CompressionMode.decompress);
            FileByteStream out("bzip2/test.decompressed", FileMode.create);
            decompressStream.CopyTo(out);
        }
    }
    catch (const Exception& ex)
    {
        Console.Error() << ex.ToString() << endl();
        return 1;
    }
    return 0;
}
\end{lstlisting}

\subsection{Member Functions}
\begin{flushleft}
\begin{supertabular}[l]{!{\raggedright}p{7.575cm}!{\raggedright}p{7.575cm}}
\textbf{Member Function}
& \textbf{Description}
\\
\hline
\hyperlink{System.IO.Compression.BZip2Stream.constructor.P.System.IO.Compression.BZip2Stream}{BZip2Stream()}
& Default constructor.

\\
\hyperlink{System.IO.Compression.BZip2Stream.constructor.P.System.IO.Compression.BZip2Stream.R.System.IO.ByteStream.System.IO.Compression.CompressionMode}{BZip2Stream(System.\-IO.\-ByteStream\&\-, System.\-IO.\-Compression.\-CompressionMode)}
& Constructor. Initializes the \textbf{BZip2Stream}
 class with the given compression mode and underlying byte stream.

\\
\hyperlink{System.IO.Compression.BZip2Stream.constructor.P.System.IO.Compression.BZip2Stream.R.System.IO.ByteStream.System.IO.Compression.CompressionMode.int}{BZip2Stream(System.\-IO.\-ByteStream\&\-, System.\-IO.\-Compression.\-CompressionMode, int)}
& Constructor. Initializes the \textbf{BZip2Stream}
 class with the given compression mode, buffer size and underlying byte stream.

\\
\hyperlink{System.IO.Compression.BZip2Stream.constructor.P.System.IO.Compression.BZip2Stream.R.System.IO.ByteStream.int}{BZip2Stream(System.\-IO.\-ByteStream\&\-, int)}
& Constructor. Initializes the \textbf{BZip2Stream}
 class using compression mode \hyperlink{System.IO.Compression.CompressionMode.compress}{compress} and given compression level.

\\
\hyperlink{System.IO.Compression.BZip2Stream.constructor.P.System.IO.Compression.BZip2Stream.R.System.IO.ByteStream.int.int}{BZip2Stream(System.\-IO.\-ByteStream\&\-, int, int)}
& Constructor. Initializes the \textbf{BZip2Stream}
 class using compression mode \hyperlink{System.IO.Compression.CompressionMode.compress}{compress},
given compression level and given compression work factor.

\\
\hyperlink{System.IO.Compression.BZip2Stream.constructor.P.System.IO.Compression.BZip2Stream.R.System.IO.ByteStream.int.int.int}{BZip2Stream(System.\-IO.\-ByteStream\&\-, int, int, int)}
& Constructor. Initializes the \textbf{BZip2Stream}
 class using compression mode \hyperlink{System.IO.Compression.CompressionMode.compress}{compress},
given compression level, given compression work factor and given buffer size.

\\
\hyperlink{System.IO.Compression.BZip2Stream.Mode.C.P.System.IO.Compression.BZip2Stream}{Mode() const}
& Returns the compression mode.

\\
\hyperlink{System.IO.Compression.BZip2Stream.Read.P.System.IO.Compression.BZip2Stream.P.byte.int}{Read(byte*, int)}
& Reads compressed data from the underlying byte stream and decompresses it to the given buffer.
Actually the decompression is not done on per call basis but more efficient means, but
the result is as described.

\\
\hyperlink{System.IO.Compression.BZip2Stream.ReadByte.P.System.IO.Compression.BZip2Stream}{ReadByte()}
& Reads compressed data from the underlying byte stream, decompresses it to an internal buffer
and returns one byte of decompressed data.
Actually the decompression is not done on per call basis but more efficient means, but
the result is as described.

\\
\hyperlink{System.IO.Compression.BZip2Stream.Write.P.System.IO.Compression.BZip2Stream.byte}{Write(byte)}
& Writes one byte of data to an internal buffer, compresses it and writes the compressed data
to the underlying byte stream.
Actually the compression is not done on per call basis but more efficient means, but
the result is as described.

\\
\hyperlink{System.IO.Compression.BZip2Stream.Write.P.System.IO.Compression.BZip2Stream.P.byte.int}{Write(byte*, int)}
& Writes given number of bytes from the given buffer to an internal buffer,
compresses it and writes the compressed data to the underlying byte stream.
Actually the compression is not done on per call basis but more efficient means, but
the result is as described.

\\
\hyperlink{System.IO.Compression.BZip2Stream.destructor.P.System.IO.Compression.BZip2Stream}{$\sim$BZip2Stream()}
& Destructor. If the compression mode is \hyperlink{System.IO.Compression.CompressionMode.compress}{compress} compresses the rest of the data
and writes it to the underlying byte stream. Releases occupied memory.

\\
\end{supertabular}

\end{flushleft}
\clearpage

\hypertarget{System.IO.Compression.BZip2Stream.constructor.P.System.IO.Compression.BZip2Stream}{\subsubsection*{BZip2Stream() Member Function}}\begin{flushleft}
Default constructor.

\end{flushleft}
\subsubsection*{Syntax}
\texttt{public BZip2Stream();}
\clearpage

\hypertarget{System.IO.Compression.BZip2Stream.constructor.P.System.IO.Compression.BZip2Stream.R.System.IO.ByteStream.System.IO.Compression.CompressionMode}{\subsubsection*{BZip2Stream(System.IO.ByteStream\&, System.IO.Compression.CompressionMode) Member Function}}
\begin{flushleft}
Constructor. Initializes the \hyperlink{System.IO.Compression.BZip2Stream}{BZip2Stream} class with the given compression mode and underlying byte stream.

\end{flushleft}
\subsubsection*{Syntax}
\texttt{public BZip2Stream(System.IO.ByteStream\& underlyingStream\_, System.IO.Compression.CompressionMode mode\_);}
\subsubsection*{Parameters}
\begin{flushleft}
\begin{supertabular}[l]{!{\raggedright}p{4.10323cm}!{\raggedright}p{5.76344cm}!{\raggedright}p{4.93333cm}}
\textbf{Name}
& \textbf{Type}
& \textbf{Description}
\\
\hline
underlyingStream\_
& System.\-IO.\-ByteStream\&\-
& Underlying byte stream to write to or read from.

\\
mode\_
& \hyperlink{System.IO.Compression.CompressionMode}{System.\-IO.\-Compression.\-CompressionMode}
& Compression mode. Can be \hyperlink{System.IO.Compression.CompressionMode.compress}{compress} or \hyperlink{System.IO.Compression.CompressionMode.decompress}{decompress}.
When the compression mode is \hyperlink{System.IO.Compression.CompressionMode.compress}{compress} the stream supports writing,
when the compression mode is \hyperlink{System.IO.Compression.CompressionMode.decompress}{decompress} the stream supports reading.

\\
\end{supertabular}

\end{flushleft}
\subsubsection*{Remarks}
\begin{flushleft}
When the compression mode is \hyperlink{System.IO.Compression.CompressionMode.compress}{compress}, the stream uses default compression level \hyperlink{System.IO.Compression.defaultBZip2CompressionLevel}{defaultBZip2CompressionLevel},
default work factor \hyperlink{System.IO.Compression.defaultBZip2WorkFactor}{defaultBZip2WorkFactor} and default buffer size 16K for internal input and output buffers.

\end{flushleft}
\clearpage

\hypertarget{System.IO.Compression.BZip2Stream.constructor.P.System.IO.Compression.BZip2Stream.R.System.IO.ByteStream.System.IO.Compression.CompressionMode.int}{\subsubsection*{BZip2Stream(System.IO.ByteStream\&, System.IO.Compression.CompressionMode, int) Member Function}}
\begin{flushleft}
Constructor. Initializes the \hyperlink{System.IO.Compression.BZip2Stream}{BZip2Stream} class with the given compression mode, buffer size and underlying byte stream.

\end{flushleft}
\subsubsection*{Syntax}
\texttt{public BZip2Stream(System.IO.ByteStream\& underlyingStream\_, System.IO.Compression.CompressionMode mode\_, int bufferSize\_);}
\subsubsection*{Parameters}
\begin{flushleft}
\begin{supertabular}[l]{!{\raggedright}p{4.10323cm}!{\raggedright}p{5.76344cm}!{\raggedright}p{4.93333cm}}
\textbf{Name}
& \textbf{Type}
& \textbf{Description}
\\
\hline
underlyingStream\_
& System.\-IO.\-ByteStream\&\-
& Underlying byte stream to write to or read from.

\\
mode\_
& \hyperlink{System.IO.Compression.CompressionMode}{System.\-IO.\-Compression.\-CompressionMode}
& Compression mode. Can be \hyperlink{System.IO.Compression.CompressionMode.compress}{compress} or \hyperlink{System.IO.Compression.CompressionMode.decompress}{decompress}.
When the compression mode is \hyperlink{System.IO.Compression.CompressionMode.compress}{compress} the stream supports writing,
when the compression mode is \hyperlink{System.IO.Compression.CompressionMode.decompress}{decompress} the stream supports reading.

\\
bufferSize\_
& int
& Buffer size for internal input and output buffers.

\\
\end{supertabular}

\end{flushleft}
\subsubsection*{Remarks}
\begin{flushleft}
When the compression mode is \hyperlink{System.IO.Compression.CompressionMode.compress}{compress}, the stream uses default compression level \hyperlink{System.IO.Compression.defaultBZip2CompressionLevel}{defaultBZip2CompressionLevel} and
default work factor \hyperlink{System.IO.Compression.defaultBZip2WorkFactor}{defaultBZip2WorkFactor}.

\end{flushleft}
\clearpage

\hypertarget{System.IO.Compression.BZip2Stream.constructor.P.System.IO.Compression.BZip2Stream.R.System.IO.ByteStream.int}{\subsubsection*{BZip2Stream(System.IO.ByteStream\&, int) Member Function}}
\begin{flushleft}
Constructor. Initializes the \hyperlink{System.IO.Compression.BZip2Stream}{BZip2Stream} class using compression mode \hyperlink{System.IO.Compression.CompressionMode.compress}{compress} and given compression level.

\end{flushleft}
\subsubsection*{Syntax}
\texttt{public BZip2Stream(System.IO.ByteStream\& underlyingStream\_, int compressionLevel\_);}
\subsubsection*{Parameters}
\begin{flushleft}
\begin{supertabular}[l]{!{\raggedright}p{4.10323cm}!{\raggedright}p{5.76344cm}!{\raggedright}p{4.93333cm}}
\textbf{Name}
& \textbf{Type}
& \textbf{Description}
\\
\hline
underlyingStream\_
& System.\-IO.\-ByteStream\&\-
& Underlying byte stream to write the compressed data to.

\\
compressionLevel\_
& int
& Compression level. Can be in range 1..9. Compression level N sets bzip2 block size to N * 100K. Compression level 1 gives least compression and uses minimum memory.
Compression level 9 gives the best compression but takes most memory.

\\
\end{supertabular}

\end{flushleft}
\subsubsection*{Remarks}
\begin{flushleft}
Sets the work factor \hyperlink{System.IO.Compression.defaultBZip2WorkFactor}{defaultBZip2WorkFactor} and uses the default buffer size 16K for internal input and output buffers.

\end{flushleft}
\clearpage

\hypertarget{System.IO.Compression.BZip2Stream.constructor.P.System.IO.Compression.BZip2Stream.R.System.IO.ByteStream.int.int}{\subsubsection*{BZip2Stream(System.IO.ByteStream\&, int, int) Member Function}}
\begin{flushleft}
Constructor. Initializes the \hyperlink{System.IO.Compression.BZip2Stream}{BZip2Stream} class using compression mode \hyperlink{System.IO.Compression.CompressionMode.compress}{compress},
given compression level and given compression work factor.

\end{flushleft}
\subsubsection*{Syntax}
\texttt{public BZip2Stream(System.IO.ByteStream\& underlyingStream\_, int compressionLevel\_, int compressionWorkFactor\_);}
\subsubsection*{Parameters}
\begin{flushleft}
\begin{supertabular}[l]{!{\raggedright}p{4.93333cm}!{\raggedright}p{4.93333cm}!{\raggedright}p{4.93333cm}}
\textbf{Name}
& \textbf{Type}
& \textbf{Description}
\\
\hline
underlyingStream\_
& System.\-IO.\-ByteStream\&\-
& Underlying byte stream to write the compressed data to.

\\
compressionLevel\_
& int
& Compression level. Can be in range 1 to 9 inclusive. Compression level N sets bzip2 block size to N * 100K.
Compression level 1 gives least compression and uses minimal memory.
Compression level 9 gives the best compression but takes most memory.

\\
compressionWorkFactor\_
& int
& Compression work factor. Can be in range 0 to 250 inclusive. Setting the work factor to 0 uses the default work factor 30.
See \url{http://www.bzip.org/1.0.5/bzip2-manual-1.0.5.html} for details.

\\
\end{supertabular}

\end{flushleft}
\subsubsection*{Remarks}
\begin{flushleft}
Uses the default buffer size 16K for internal input and output buffers.

\end{flushleft}
\clearpage

\hypertarget{System.IO.Compression.BZip2Stream.constructor.P.System.IO.Compression.BZip2Stream.R.System.IO.ByteStream.int.int.int}{\subsubsection*{BZip2Stream(System.IO.ByteStream\&, int, int, int) Member Function}}
\begin{flushleft}
Constructor. Initializes the \hyperlink{System.IO.Compression.BZip2Stream}{BZip2Stream} class using compression mode \hyperlink{System.IO.Compression.CompressionMode.compress}{compress},
given compression level, given compression work factor and given buffer size.

\end{flushleft}
\subsubsection*{Syntax}
\texttt{public BZip2Stream(System.IO.ByteStream\& underlyingStream\_, int compressionLevel\_, int compressionWorkFactor\_, int bufferSize\_);}
\subsubsection*{Parameters}
\begin{flushleft}
\begin{supertabular}[l]{!{\raggedright}p{4.93333cm}!{\raggedright}p{4.93333cm}!{\raggedright}p{4.93333cm}}
\textbf{Name}
& \textbf{Type}
& \textbf{Description}
\\
\hline
underlyingStream\_
& System.\-IO.\-ByteStream\&\-
& Underlying byte stream to write the compressed data to.

\\
compressionLevel\_
& int
& Compression level. Can be in range 1 to 9 inclusive. Compression level N sets bzip2 block size to N * 100K.
Compression level 1 gives least compression and uses minimal memory.
Compression level 9 gives the best compression but takes most memory.

\\
compressionWorkFactor\_
& int
& Compression work factor. Can be in range 0 to 250 inclusive. Setting the work factor to 0 uses the default work factor 30.
See \url{http://www.bzip.org/1.0.5/bzip2-manual-1.0.5.html} for details.

\\
bufferSize\_
& int
& Size of internal input and output buffers.

\\
\end{supertabular}

\end{flushleft}
\clearpage

\hypertarget{System.IO.Compression.BZip2Stream.Mode.C.P.System.IO.Compression.BZip2Stream}{\subsubsection*{Mode() const Member Function}}\begin{flushleft}
Returns the compression mode.

\end{flushleft}

\subsubsection*{Syntax}\texttt{public System.IO.Compression.CompressionMode Mode() const;}

\subsubsection*{Returns}
\hyperlink{System.IO.Compression.CompressionMode}{System.\-IO.\-Compression.\-CompressionMode}\begin{flushleft}
Returns the compression mode.

\end{flushleft}
\clearpage

\hypertarget{System.IO.Compression.BZip2Stream.Read.P.System.IO.Compression.BZip2Stream.P.byte.int}{\subsubsection*{Read(byte*, int) Member Function}}
\begin{flushleft}
Reads compressed data from the underlying byte stream and decompresses it to the given buffer.
Actually the decompression is not done on per call basis but more efficient means, but
the result is as described.

\end{flushleft}
\subsubsection*{Syntax}\texttt{public int Read(byte* buf, int count);}

\subsubsection*{Parameters}
\begin{flushleft}
\begin{supertabular}[l]{lll}
\textbf{Name}
& \textbf{Type}
& \textbf{Description}
\\
\hline
buf
& byte*
& A buffer to decompress the data to.

\\
count
& int
& Maximum number of bytes to read.

\\
\end{supertabular}

\end{flushleft}
\subsubsection*{Returns}int
\begin{flushleft}
Returns the number of bytes read. Can be less than the size requested but is always non-negative.
The return value of 0 indicates end of stream.

\end{flushleft}
\subsubsection*{Remarks}
\begin{flushleft}
Throws \hyperlink{System.IO.Compression.BZip2Exception}{BZip2Exception} if an error in decompression process is encountered.
If an error reading from the underlying byte stream is encountered,
can also throw System.\-IO.\-IOException if the underlying byte stream is System.\-IO.\-FileByteStream, or
System.\-Net.\-Sockets.\-SocketError if the underlying byte stream is System.\-Net.\-Sockets.\-SocketByteStream.

\end{flushleft}
\clearpage

\hypertarget{System.IO.Compression.BZip2Stream.ReadByte.P.System.IO.Compression.BZip2Stream}{\subsubsection*{ReadByte() Member Function}}
\begin{flushleft}
Reads compressed data from the underlying byte stream, decompresses it to an internal buffer
and returns one byte of decompressed data.
Actually the decompression is not done on per call basis but more efficient means, but
the result is as described.

\end{flushleft}
\subsubsection*{Syntax}\texttt{public int ReadByte();}
\subsubsection*{Returns}
int
\begin{flushleft}
Returns one byte of decompressed data, or -1 if end of stream is encountered.

\end{flushleft}
\subsubsection*{Remarks}
\begin{flushleft}
Throws \hyperlink{System.IO.Compression.BZip2Exception}{BZip2Exception} if an error in decompression process is encountered.
If an error reading from the underlying byte stream is encountered,
can also throw System.\-IO.\-IOException if the underlying byte stream is System.\-IO.\-FileByteStream, or
System.\-Net.\-Sockets.\-SocketError if the underlying byte stream is System.\-Net.\-Sockets.\-SocketByteStream.

\end{flushleft}
\clearpage

\hypertarget{System.IO.Compression.BZip2Stream.Write.P.System.IO.Compression.BZip2Stream.byte}{\subsubsection*{Write(byte) Member Function}}
\begin{flushleft}
Writes one byte of data to an internal buffer, compresses it and writes the compressed data
to the underlying byte stream.
Actually the compression is not done on per call basis but more efficient means, but
the result is as described.

\end{flushleft}
\subsubsection*{Syntax}\texttt{public void Write(byte x);}

\subsubsection*{Parameters}
\begin{flushleft}
\begin{supertabular}[l]{lll}
\textbf{Name}
& \textbf{Type}
& \textbf{Description}
\\
\hline
x
& byte
& Byte to write.

\\
\end{supertabular}

\end{flushleft}
\subsubsection*{Remarks}
\begin{flushleft}
Throws \hyperlink{System.IO.Compression.BZip2Exception}{BZip2Exception} if an error in compression process is encountered.
If an error writing from the underlying byte stream is encountered,
can also throw System.\-IO.\-IOException if the underlying byte stream is System.\-IO.\-FileByteStream, or
System.\-Net.\-Sockets.\-SocketError if the underlying byte stream is System.\-Net.\-Sockets.\-SocketByteStream.

\end{flushleft}
\clearpage

\hypertarget{System.IO.Compression.BZip2Stream.Write.P.System.IO.Compression.BZip2Stream.P.byte.int}{\subsubsection*{Write(byte*, int) Member Function}}
\begin{flushleft}
Writes given number of bytes from the given buffer to an internal buffer,
compresses it and writes the compressed data to the underlying byte stream.
Actually the compression is not done on per call basis but more efficient means, but
the result is as described.

\end{flushleft}
\subsubsection*{Syntax}\texttt{public void Write(byte* buf, int count);}

\subsubsection*{Parameters}
\begin{flushleft}
\begin{supertabular}[l]{lll}
\textbf{Name}
& \textbf{Type}
& \textbf{Description}
\\
\hline
buf
& byte*
& A buffer of data to write.

\\
count
& int
& Number of bytes to write.

\\
\end{supertabular}

\end{flushleft}
\subsubsection*{Remarks}
\begin{flushleft}
Throws \hyperlink{System.IO.Compression.BZip2Exception}{BZip2Exception} if an error in compression process is encountered.
If an error writing from the underlying byte stream is encountered,
can also throw System.\-IO.\-IOException if the underlying byte stream is System.\-IO.\-FileByteStream, or
System.\-Net.\-Sockets.\-SocketError if the underlying byte stream is System.\-Net.\-Sockets.\-SocketByteStream.

\end{flushleft}
\clearpage

\hypertarget{System.IO.Compression.BZip2Stream.destructor.P.System.IO.Compression.BZip2Stream}{\subsubsection*{$\sim$BZip2Stream() Member Function}}
\begin{flushleft}
Destructor. If the compression mode is \hyperlink{System.IO.Compression.CompressionMode.compress}{compress} compresses the rest of the data
and writes it to the underlying byte stream. Releases occupied memory.

\end{flushleft}
\subsubsection*{Syntax}\texttt{public $\sim$BZip2Stream();}
\clearpage

\hypertarget{System.IO.Compression.DeflateException}{\section{DeflateException Class}}
\begin{flushleft}
Exception class thrown when \hyperlink{System.IO.Compression.DeflateStream}{DeflateStream} cannot compress or decompress data.

\end{flushleft}
\subsection*{Syntax}\texttt{public class DeflateException;}

\subsection*{Base Class}System.Exception\subsection{Member Functions}
\begin{flushleft}
\begin{supertabular}[l]{!{\raggedright}p{7.575cm}!{\raggedright}p{7.575cm}}
\textbf{Member Function}
& \textbf{Description}
\\
\hline
\hyperlink{System.IO.Compression.DeflateException.constructor.P.System.IO.Compression.DeflateException}{DeflateException()}
& Default constructor.

\\
\hyperlink{System.IO.Compression.DeflateException.constructor.P.System.IO.Compression.DeflateException.C.R.System.IO.Compression.DeflateException}{DeflateException(const System.\-IO.\-Compression.\-DeflateException\&\-)}
& Copy constructor.

\\
\hyperlink{System.IO.Compression.DeflateException.operator.assign.P.System.IO.Compression.DeflateException.C.R.System.IO.Compression.DeflateException}{operator=(const System.\-IO.\-Compression.\-DeflateException\&\-)}
& Copy assignment.

\\
\hyperlink{System.IO.Compression.DeflateException.constructor.P.System.IO.Compression.DeflateException.RR.System.IO.Compression.DeflateException}{DeflateException(System.\-IO.\-Compression.\-DeflateException\&\-\&\-)}
& Move constructor.

\\
\hyperlink{System.IO.Compression.DeflateException.operator.assign.P.System.IO.Compression.DeflateException.RR.System.IO.Compression.DeflateException}{operator=(System.\-IO.\-Compression.\-DeflateException\&\-\&\-)}
& Move assignment.

\\
\hyperlink{System.IO.Compression.DeflateException.constructor.P.System.IO.Compression.DeflateException.C.R.System.String.int}{DeflateException(const System.\-String\&\-, int)}
& Constructor. Initializes the \textbf{DeflateException}
 with the given error message and error code.

\\
\hyperlink{System.IO.Compression.DeflateException.ErrorCode.C.P.System.IO.Compression.DeflateException}{ErrorCode() const}
& Returns the error code.

\\
\hyperlink{System.IO.Compression.DeflateException.destructor.P.System.IO.Compression.DeflateException}{$\sim$DeflateException()}
& Destructor.

\\
\end{supertabular}

\end{flushleft}
\clearpage

\hypertarget{System.IO.Compression.DeflateException.constructor.P.System.IO.Compression.DeflateException}{\subsubsection*{DeflateException() Member Function}}\begin{flushleft}
Default constructor.

\end{flushleft}
\subsubsection*{Syntax}
\texttt{public DeflateException();}
\clearpage

\hypertarget{System.IO.Compression.DeflateException.constructor.P.System.IO.Compression.DeflateException.C.R.System.IO.Compression.DeflateException}{\subsubsection*{DeflateException(const System.IO.Compression.DeflateException\&) Member Function}}\begin{flushleft}
Copy constructor.

\end{flushleft}
\subsubsection*{Syntax}
\texttt{public DeflateException(const System.IO.Compression.DeflateException\& that);}
\subsubsection*{Parameters}
\begin{flushleft}
\begin{supertabular}[l]{!{\raggedright}p{1.30721cm}!{\raggedright}p{8.55945cm}!{\raggedright}p{4.55908cm}}
\textbf{Name}
& \textbf{Type}
& \textbf{Description}
\\
\hline
that
& \hyperlink{System.IO.Compression.DeflateException}{const System.\-IO.\-Compression.\-DeflateException\&\-}
& Argument to copy.

\\
\end{supertabular}

\end{flushleft}
\clearpage

\hypertarget{System.IO.Compression.DeflateException.operator.assign.P.System.IO.Compression.DeflateException.C.R.System.IO.Compression.DeflateException}{\subsubsection*{operator=(const System.IO.Compression.DeflateException\&) Member Function}}\begin{flushleft}
Copy assignment.

\end{flushleft}
\subsubsection*{Syntax}
\texttt{public void operator=(const System.IO.Compression.DeflateException\& that);}
\subsubsection*{Parameters}
\begin{flushleft}
\begin{supertabular}[l]{!{\raggedright}p{1.30721cm}!{\raggedright}p{8.55945cm}!{\raggedright}p{4.89046cm}}
\textbf{Name}
& \textbf{Type}
& \textbf{Description}
\\
\hline
that
& \hyperlink{System.IO.Compression.DeflateException}{const System.\-IO.\-Compression.\-DeflateException\&\-}
& Argument to assign.

\\
\end{supertabular}

\end{flushleft}
\clearpage

\hypertarget{System.IO.Compression.DeflateException.constructor.P.System.IO.Compression.DeflateException.RR.System.IO.Compression.DeflateException}{\subsubsection*{DeflateException(System.IO.Compression.DeflateException\&\&) Member Function}}\begin{flushleft}
Move constructor.

\end{flushleft}
\subsubsection*{Syntax}
\texttt{public DeflateException(System.IO.Compression.DeflateException\&\& that);}
\subsubsection*{Parameters}
\begin{flushleft}
\begin{supertabular}[l]{!{\raggedright}p{1.30721cm}!{\raggedright}p{8.55945cm}!{\raggedright}p{4.93333cm}}
\textbf{Name}
& \textbf{Type}
& \textbf{Description}
\\
\hline
that
& \hyperlink{System.IO.Compression.DeflateException}{System.\-IO.\-Compression.\-DeflateException\&\-\&\-}
& Argument to move from.

\\
\end{supertabular}

\end{flushleft}
\clearpage

\hypertarget{System.IO.Compression.DeflateException.operator.assign.P.System.IO.Compression.DeflateException.RR.System.IO.Compression.DeflateException}{\subsubsection*{operator=(System.IO.Compression.DeflateException\&\&) Member Function}}\begin{flushleft}
Move assignment.

\end{flushleft}
\subsubsection*{Syntax}
\texttt{public void operator=(System.IO.Compression.DeflateException\&\& that);}
\subsubsection*{Parameters}
\begin{flushleft}
\begin{supertabular}[l]{!{\raggedright}p{1.30721cm}!{\raggedright}p{8.55945cm}!{\raggedright}p{4.93333cm}}
\textbf{Name}
& \textbf{Type}
& \textbf{Description}
\\
\hline
that
& \hyperlink{System.IO.Compression.DeflateException}{System.\-IO.\-Compression.\-DeflateException\&\-\&\-}
& Argument to assign from.

\\
\end{supertabular}

\end{flushleft}
\clearpage

\hypertarget{System.IO.Compression.DeflateException.constructor.P.System.IO.Compression.DeflateException.C.R.System.String.int}{\subsubsection*{DeflateException(const System.String\&, int) Member Function}}
\begin{flushleft}
Constructor. Initializes the \hyperlink{System.IO.Compression.DeflateException}{DeflateException} with the given error message and error code.

\end{flushleft}
\subsubsection*{Syntax}
\texttt{public DeflateException(const System.String\& message\_, int errorCode\_);}
\subsubsection*{Parameters}
\begin{flushleft}
\begin{supertabular}[l]{lll}
\textbf{Name}
& \textbf{Type}
& \textbf{Description}
\\
\hline
message\_
& const System.\-String\&\-
& Error message.

\\
errorCode\_
& int
& Error code returned by the zlib library.

\\
\end{supertabular}

\end{flushleft}
\clearpage

\hypertarget{System.IO.Compression.DeflateException.ErrorCode.C.P.System.IO.Compression.DeflateException}{\subsubsection*{ErrorCode() const Member Function}}\begin{flushleft}
Returns the error code.

\end{flushleft}

\subsubsection*{Syntax}\texttt{public int ErrorCode() const;}
\subsubsection*{Returns}int
\begin{flushleft}
Returns the error code.

\end{flushleft}
\clearpage

\hypertarget{System.IO.Compression.DeflateException.destructor.P.System.IO.Compression.DeflateException}{\subsubsection*{$\sim$DeflateException() Member Function}}\begin{flushleft}
Destructor.

\end{flushleft}
\subsubsection*{Syntax}
\texttt{public $\sim$DeflateException();}
\clearpage

\hypertarget{System.IO.Compression.DeflateStream}{\section{DeflateStream Class}}
\begin{flushleft}
A stream class that writes data to the underlying byte stream in ZLIB compression format, or reads data compressed in ZLIB format from the underlying byte stream and decompresses it.

\end{flushleft}
\subsection*{Syntax}\texttt{public class DeflateStream;}

\subsection*{Base Class}System.IO.ByteStream
\subsection{Example}
\lstset{frameround=fttt}\begin{lstlisting}[frame=trBL]
using System;
using System.IO;
using System.IO.Compression;

int main()
{
    try
    {
        {
            FileByteStream in("deflate/test.file", FileMode.open);
            FileByteStream out("deflate/test.compressed", FileMode.create);
            DeflateStream compressStream(out, CompressionMode.compress);
            in.CopyTo(compressStream);
        }    
        {
            FileByteStream in("deflate/test.compressed", FileMode.open);
            DeflateStream decompressStream(in, CompressionMode.decompress);
            FileByteStream out("deflate/test.decompressed", FileMode.create);
            decompressStream.CopyTo(out);
        }
    }
    catch (const Exception& ex)
    {
        Console.Error() << ex.ToString() << endl();
        return 1;
    }
    return 0;
}
\end{lstlisting}

\subsection{Member Functions}
\begin{flushleft}
\begin{supertabular}[l]{!{\raggedright}p{7.575cm}!{\raggedright}p{7.575cm}}
\textbf{Member Function}
& \textbf{Description}
\\
\hline
\hyperlink{System.IO.Compression.DeflateStream.constructor.P.System.IO.Compression.DeflateStream}{DeflateStream()}
& Default constructor.

\\
\hyperlink{System.IO.Compression.DeflateStream.constructor.P.System.IO.Compression.DeflateStream.R.System.IO.ByteStream.System.IO.Compression.CompressionMode}{DeflateStream(System.\-IO.\-ByteStream\&\-, System.\-IO.\-Compression.\-CompressionMode)}
& Constructor. Initializes the System.\-IO.\-Compression.\-DeflatStream class with the given compression mode and underlying byte stream.

\\
\hyperlink{System.IO.Compression.DeflateStream.constructor.P.System.IO.Compression.DeflateStream.R.System.IO.ByteStream.System.IO.Compression.CompressionMode.int}{DeflateStream(System.\-IO.\-ByteStream\&\-, System.\-IO.\-Compression.\-CompressionMode, int)}
& Constructor. Initializes the \textbf{DeflateStream}
 class with the given compression mode, buffer size and underlying byte stream.

\\
\hyperlink{System.IO.Compression.DeflateStream.constructor.P.System.IO.Compression.DeflateStream.R.System.IO.ByteStream.int}{DeflateStream(System.\-IO.\-ByteStream\&\-, int)}
& Constructor. Initializes the \textbf{DeflateStream}
 class using compression mode \hyperlink{System.IO.Compression.CompressionMode.compress}{compress} and given compression level.

\\
\hyperlink{System.IO.Compression.DeflateStream.constructor.P.System.IO.Compression.DeflateStream.R.System.IO.ByteStream.int.int}{DeflateStream(System.\-IO.\-ByteStream\&\-, int, int)}
& Constructor. Initializes the \textbf{DeflateStream}
 class using compression mode \hyperlink{System.IO.Compression.CompressionMode.compress}{compress}, given compression level and
given buffer size.

\\
\hyperlink{System.IO.Compression.DeflateStream.Mode.C.P.System.IO.Compression.DeflateStream}{Mode() const}
& Returns the compression mode.

\\
\hyperlink{System.IO.Compression.DeflateStream.Read.P.System.IO.Compression.DeflateStream.P.byte.int}{Read(byte*, int)}
& Reads compressed data from the underlying byte stream and decompresses it to the given buffer.
Actually the decompression is not done on per call basis but more efficient means, but
the result is as described.

\\
\hyperlink{System.IO.Compression.DeflateStream.ReadByte.P.System.IO.Compression.DeflateStream}{ReadByte()}
& Reads compressed data from the underlying byte stream, decompresses it to an internal buffer
and returns one byte of decompressed data.
Actually the decompression is not done on per call basis but more efficient means, but
the result is as described.

\\
\hyperlink{System.IO.Compression.DeflateStream.Write.P.System.IO.Compression.DeflateStream.byte}{Write(byte)}
& Writes one byte of data to an internal buffer, compresses it and writes the compressed data
to the underlying byte stream.
Actually the compression is not done on per call basis but more efficient means, but
the result is as described.

\\
\hyperlink{System.IO.Compression.DeflateStream.Write.P.System.IO.Compression.DeflateStream.P.byte.int}{Write(byte*, int)}
& Writes given number of bytes from the given buffer to an internal buffer,
compresses it and writes the compressed data to the underlying byte stream.
Actually the compression is not done on per call basis but more efficient means, but
the result is as described.

\\
\hyperlink{System.IO.Compression.DeflateStream.destructor.P.System.IO.Compression.DeflateStream}{$\sim$DeflateStream()}
& Destructor. If the compression mode is \hyperlink{System.IO.Compression.CompressionMode.compress}{compress} compresses the rest of the data
and writes it to the underlying byte stream. Releases occupied memory.

\\
\end{supertabular}

\end{flushleft}
\clearpage

\hypertarget{System.IO.Compression.DeflateStream.constructor.P.System.IO.Compression.DeflateStream}{\subsubsection*{DeflateStream() Member Function}}\begin{flushleft}
Default constructor.

\end{flushleft}
\subsubsection*{Syntax}
\texttt{public DeflateStream();}
\clearpage

\hypertarget{System.IO.Compression.DeflateStream.constructor.P.System.IO.Compression.DeflateStream.R.System.IO.ByteStream.System.IO.Compression.CompressionMode}{\subsubsection*{DeflateStream(System.IO.ByteStream\&, System.IO.Compression.CompressionMode) Member Function}}
\begin{flushleft}
Constructor. Initializes the System.\-IO.\-Compression.\-DeflatStream class with the given compression mode and underlying byte stream.

\end{flushleft}
\subsubsection*{Syntax}
\texttt{public DeflateStream(System.IO.ByteStream\& underlyingStream\_, System.IO.Compression.CompressionMode mode\_);}
\subsubsection*{Parameters}
\begin{flushleft}
\begin{supertabular}[l]{!{\raggedright}p{4.10323cm}!{\raggedright}p{5.76344cm}!{\raggedright}p{4.93333cm}}
\textbf{Name}
& \textbf{Type}
& \textbf{Description}
\\
\hline
underlyingStream\_
& System.\-IO.\-ByteStream\&\-
& Underlying byte stream to write to or read from.

\\
mode\_
& \hyperlink{System.IO.Compression.CompressionMode}{System.\-IO.\-Compression.\-CompressionMode}
& Compression mode. Can be \hyperlink{System.IO.Compression.CompressionMode.compress}{compress} or \hyperlink{System.IO.Compression.CompressionMode.decompress}{decompress}.
When the compression mode is \hyperlink{System.IO.Compression.CompressionMode.compress}{compress} the stream supports writing,
when the compression mode is \hyperlink{System.IO.Compression.CompressionMode.decompress}{decompress} the stream supports reading.

\\
\end{supertabular}

\end{flushleft}
\subsubsection*{Remarks}
\begin{flushleft}
When the compression mode is \hyperlink{System.IO.Compression.CompressionMode.compress}{compress}, the stream uses default compression level \hyperlink{System.IO.Compression.defaultDeflateCompressionLevel}{defaultDeflateCompressionLevel}
and default buffer size 16K for internal input and output buffers.

\end{flushleft}
\clearpage

\hypertarget{System.IO.Compression.DeflateStream.constructor.P.System.IO.Compression.DeflateStream.R.System.IO.ByteStream.System.IO.Compression.CompressionMode.int}{\subsubsection*{DeflateStream(System.IO.ByteStream\&, System.IO.Compression.CompressionMode, int) Member Function}}
\begin{flushleft}
Constructor. Initializes the \hyperlink{System.IO.Compression.DeflateStream}{DeflateStream} class with the given compression mode, buffer size and underlying byte stream.

\end{flushleft}
\subsubsection*{Syntax}
\texttt{public DeflateStream(System.IO.ByteStream\& underlyingStream\_, System.IO.Compression.CompressionMode mode\_, int bufferSize\_);}
\subsubsection*{Parameters}
\begin{flushleft}
\begin{supertabular}[l]{!{\raggedright}p{4.10323cm}!{\raggedright}p{5.76344cm}!{\raggedright}p{4.93333cm}}
\textbf{Name}
& \textbf{Type}
& \textbf{Description}
\\
\hline
underlyingStream\_
& System.\-IO.\-ByteStream\&\-
& Underlying byte stream to write to or read from.

\\
mode\_
& \hyperlink{System.IO.Compression.CompressionMode}{System.\-IO.\-Compression.\-CompressionMode}
& Compression mode. Can be \hyperlink{System.IO.Compression.CompressionMode.compress}{compress} or \hyperlink{System.IO.Compression.CompressionMode.decompress}{decompress}.
When the compression mode is \hyperlink{System.IO.Compression.CompressionMode.compress}{compress} the stream supports writing,
when the compression mode is \hyperlink{System.IO.Compression.CompressionMode.decompress}{decompress} the stream supports reading.

\\
bufferSize\_
& int
& Buffer size for internal input and output buffers.

\\
\end{supertabular}

\end{flushleft}
\subsubsection*{Remarks}
\begin{flushleft}
When the compression mode is \hyperlink{System.IO.Compression.CompressionMode.compress}{compress}, the stream uses default compression level \hyperlink{System.IO.Compression.defaultDeflateCompressionLevel}{defaultDeflateCompressionLevel}.

\end{flushleft}
\clearpage

\hypertarget{System.IO.Compression.DeflateStream.constructor.P.System.IO.Compression.DeflateStream.R.System.IO.ByteStream.int}{\subsubsection*{DeflateStream(System.IO.ByteStream\&, int) Member Function}}
\begin{flushleft}
Constructor. Initializes the \hyperlink{System.IO.Compression.DeflateStream}{DeflateStream} class using compression mode \hyperlink{System.IO.Compression.CompressionMode.compress}{compress} and given compression level.

\end{flushleft}
\subsubsection*{Syntax}
\texttt{public DeflateStream(System.IO.ByteStream\& underlyingStream\_, int compressionLevel\_);}
\subsubsection*{Parameters}
\begin{flushleft}
\begin{supertabular}[l]{!{\raggedright}p{4.10323cm}!{\raggedright}p{5.76344cm}!{\raggedright}p{4.93333cm}}
\textbf{Name}
& \textbf{Type}
& \textbf{Description}
\\
\hline
underlyingStream\_
& System.\-IO.\-ByteStream\&\-
& Underlying byte stream to write the compressed data to.

\\
compressionLevel\_
& int
& Compression level. Can be in range -1 to 9.
Compression level \hyperlink{System.IO.Compression.defaultDeflateCompressionLevel}{defaultDeflateCompressionLevel} (-1) equals compression level 6.
Compression level \hyperlink{System.IO.Compression.noDeflateCompression}{noDeflateCompression} (0) uses no compression.
Compression level \hyperlink{System.IO.Compression.fastestDeflateCompression}{fastestDeflateCompression} (1) gives fastest operation but least compression.
Compression level \hyperlink{System.IO.Compression.optimalDeflateCompression}{optimalDeflateCompression} (9) gives slowest operation but best compression.

\\
\end{supertabular}

\end{flushleft}
\subsubsection*{Remarks}
\begin{flushleft}
Uses the default buffer size 16K for internal input and output buffers.

\end{flushleft}
\clearpage

\hypertarget{System.IO.Compression.DeflateStream.constructor.P.System.IO.Compression.DeflateStream.R.System.IO.ByteStream.int.int}{\subsubsection*{DeflateStream(System.IO.ByteStream\&, int, int) Member Function}}
\begin{flushleft}
Constructor. Initializes the \hyperlink{System.IO.Compression.DeflateStream}{DeflateStream} class using compression mode \hyperlink{System.IO.Compression.CompressionMode.compress}{compress}, given compression level and
given buffer size.

\end{flushleft}
\subsubsection*{Syntax}
\texttt{public DeflateStream(System.IO.ByteStream\& underlyingStream\_, int compressionLevel\_, int bufferSize\_);}
\subsubsection*{Parameters}
\begin{flushleft}
\begin{supertabular}[l]{!{\raggedright}p{4.10323cm}!{\raggedright}p{5.76344cm}!{\raggedright}p{4.93333cm}}
\textbf{Name}
& \textbf{Type}
& \textbf{Description}
\\
\hline
underlyingStream\_
& System.\-IO.\-ByteStream\&\-
& Underlying byte stream to write the compressed data to.

\\
compressionLevel\_
& int
& Compression level. Can be in range -1 to 9.
Compression level \hyperlink{System.IO.Compression.defaultDeflateCompressionLevel}{defaultDeflateCompressionLevel} (-1) equals compression level 6.
Compression level \hyperlink{System.IO.Compression.noDeflateCompression}{noDeflateCompression} (0) uses no compression.
Compression level \hyperlink{System.IO.Compression.fastestDeflateCompression}{fastestDeflateCompression} (1) gives fastest operation but least compression.
Compression level \hyperlink{System.IO.Compression.optimalDeflateCompression}{optimalDeflateCompression} (9) gives slowest operation but best compression.

\\
bufferSize\_
& int
& Buffer size for internal input and output buffers.

\\
\end{supertabular}

\end{flushleft}
\clearpage

\hypertarget{System.IO.Compression.DeflateStream.Mode.C.P.System.IO.Compression.DeflateStream}{\subsubsection*{Mode() const Member Function}}\begin{flushleft}
Returns the compression mode.

\end{flushleft}

\subsubsection*{Syntax}\texttt{public System.IO.Compression.CompressionMode Mode() const;}

\subsubsection*{Returns}
\hyperlink{System.IO.Compression.CompressionMode}{System.\-IO.\-Compression.\-CompressionMode}\begin{flushleft}
Returns the compression mode.

\end{flushleft}
\clearpage

\hypertarget{System.IO.Compression.DeflateStream.Read.P.System.IO.Compression.DeflateStream.P.byte.int}{\subsubsection*{Read(byte*, int) Member Function}}
\begin{flushleft}
Reads compressed data from the underlying byte stream and decompresses it to the given buffer.
Actually the decompression is not done on per call basis but more efficient means, but
the result is as described.

\end{flushleft}
\subsubsection*{Syntax}\texttt{public int Read(byte* buf, int count);}

\subsubsection*{Parameters}
\begin{flushleft}
\begin{supertabular}[l]{lll}
\textbf{Name}
& \textbf{Type}
& \textbf{Description}
\\
\hline
buf
& byte*
& A buffer to decompress the data to.

\\
count
& int
& Maximum bumber of bytes to read.

\\
\end{supertabular}

\end{flushleft}
\subsubsection*{Returns}int
\begin{flushleft}
Returns the number of bytes read. Can be less than the size requested but is always non-negative.
The return value of 0 indicates end of stream.

\end{flushleft}
\subsubsection*{Remarks}
\begin{flushleft}
Throws \hyperlink{System.IO.Compression.DeflateException}{DeflateException} if an error in decompression process is encountered.
If an error reading from the underlying byte stream is encountered,
can also throw System.\-IO.\-IOException if the underlying byte stream is System.\-IO.\-FileByteStream, or
System.\-Net.\-Sockets.\-SocketError if the underlying byte stream is System.\-Net.\-Sockets.\-SocketByteStream.

\end{flushleft}
\clearpage

\hypertarget{System.IO.Compression.DeflateStream.ReadByte.P.System.IO.Compression.DeflateStream}{\subsubsection*{ReadByte() Member Function}}
\begin{flushleft}
Reads compressed data from the underlying byte stream, decompresses it to an internal buffer
and returns one byte of decompressed data.
Actually the decompression is not done on per call basis but more efficient means, but
the result is as described.

\end{flushleft}
\subsubsection*{Syntax}\texttt{public int ReadByte();}
\subsubsection*{Returns}
int
\begin{flushleft}
Returns one byte of decompressed data, or -1 if end of stream is encountered.

\end{flushleft}
\subsubsection*{Remarks}
\begin{flushleft}
Throws \hyperlink{System.IO.Compression.DeflateException}{DeflateException} if an error in decompression process is encountered.
If an error reading from the underlying byte stream is encountered,
can also throw System.\-IO.\-IOException if the underlying byte stream is System.\-IO.\-FileByteStream, or
System.\-Net.\-Sockets.\-SocketError if the underlying byte stream is System.\-Net.\-Sockets.\-SocketByteStream.

\end{flushleft}
\clearpage

\hypertarget{System.IO.Compression.DeflateStream.Write.P.System.IO.Compression.DeflateStream.byte}{\subsubsection*{Write(byte) Member Function}}
\begin{flushleft}
Writes one byte of data to an internal buffer, compresses it and writes the compressed data
to the underlying byte stream.
Actually the compression is not done on per call basis but more efficient means, but
the result is as described.

\end{flushleft}
\subsubsection*{Syntax}\texttt{public void Write(byte x);}

\subsubsection*{Parameters}
\begin{flushleft}
\begin{supertabular}[l]{lll}
\textbf{Name}
& \textbf{Type}
& \textbf{Description}
\\
\hline
x
& byte
& Byte to write.

\\
\end{supertabular}

\end{flushleft}
\subsubsection*{Remarks}
\begin{flushleft}
Throws \hyperlink{System.IO.Compression.DeflateException}{DeflateException} if an error in compression process is encountered.
If an error writing from the underlying byte stream is encountered,
can also throw System.\-IO.\-IOException if the underlying byte stream is System.\-IO.\-FileByteStream, or
System.\-Net.\-Sockets.\-SocketError if the underlying byte stream is System.\-Net.\-Sockets.\-SocketByteStream.

\end{flushleft}
\clearpage

\hypertarget{System.IO.Compression.DeflateStream.Write.P.System.IO.Compression.DeflateStream.P.byte.int}{\subsubsection*{Write(byte*, int) Member Function}}
\begin{flushleft}
Writes given number of bytes from the given buffer to an internal buffer,
compresses it and writes the compressed data to the underlying byte stream.
Actually the compression is not done on per call basis but more efficient means, but
the result is as described.

\end{flushleft}
\subsubsection*{Syntax}\texttt{public void Write(byte* buf, int count);}

\subsubsection*{Parameters}
\begin{flushleft}
\begin{supertabular}[l]{lll}
\textbf{Name}
& \textbf{Type}
& \textbf{Description}
\\
\hline
buf
& byte*
& A buffer of data to write.

\\
count
& int
& Number of bytes to write.

\\
\end{supertabular}

\end{flushleft}
\subsubsection*{Remarks}
\begin{flushleft}
Throws \hyperlink{System.IO.Compression.BZip2Exception}{BZip2Exception} if an error in compression process is encountered.
If an error writing from the underlying byte stream is encountered,
can also throw System.\-IO.\-IOException if the underlying byte stream is System.\-IO.\-FileByteStream, or
System.\-Net.\-Sockets.\-SocketError if the underlying byte stream is System.\-Net.\-Sockets.\-SocketByteStream.

\end{flushleft}
\clearpage

\hypertarget{System.IO.Compression.DeflateStream.destructor.P.System.IO.Compression.DeflateStream}{\subsubsection*{$\sim$DeflateStream() Member Function}}
\begin{flushleft}
Destructor. If the compression mode is \hyperlink{System.IO.Compression.CompressionMode.compress}{compress} compresses the rest of the data
and writes it to the underlying byte stream. Releases occupied memory.

\end{flushleft}
\subsubsection*{Syntax}\texttt{public $\sim$DeflateStream();}
\clearpage

\subchapter{Functions}
\begin{flushleft}
\begin{supertabular}[l]{ll}
\textbf{Function}
& \textbf{Description}
\\
\hline
\end{supertabular}

\end{flushleft}
\clearpage
\clearpage
\subchapter{Enumerations}
\begin{flushleft}
\begin{supertabular}[l]{!{\raggedright}p{4.18998cm}!{\raggedright}p{10.96cm}}
\textbf{Enumeration}
& \textbf{Description}
\\
\hline
\hyperlink{System.IO.Compression.CompressionMode}{CompressionMode}
& Compression mode for \hyperlink{System.IO.Compression.BZip2Stream}{BZip2Stream} and \hyperlink{System.IO.Compression.DeflateStream}{DeflateStream} classes.

\\
\end{supertabular}

\end{flushleft}
\clearpage

\hypertarget{System.IO.Compression.CompressionMode}{\subsection{CompressionMode Enumeration}}
\begin{flushleft}
Compression mode for \hyperlink{System.IO.Compression.BZip2Stream}{BZip2Stream} and \hyperlink{System.IO.Compression.DeflateStream}{DeflateStream} classes.

\end{flushleft}
\subsubsection*{Enumeration Constants}
\begin{flushleft}
\begin{supertabular}[l]{!{\raggedright}p{2.61874cm}!{\raggedright}p{1.32877cm}!{\raggedright}p{8.5379cm}}
\textbf{Constant}
& \textbf{Value}
& \textbf{Description}
\\
\hline
\hypertarget{System.IO.Compression.CompressionMode.compress}{compress}
& 0
& Writes compressed data to the underlying byte stream.

\\
\hypertarget{System.IO.Compression.CompressionMode.decompress}{decompress}
& 1
& Reads compressed data from the underlying byte stream and decompresses it.

\\
\end{supertabular}

\end{flushleft}
\clearpage
\subchapter{Constants}
\begin{flushleft}
\begin{supertabular}[l]{!{\raggedright}p{3.6125cm}!{\raggedright}p{1.17735cm}!{\raggedright}p{1.32877cm}!{\raggedright}p{5.89623cm}}
\textbf{Constant}
& \textbf{Type}
& \textbf{Value}
& \textbf{Description}
\\
\hline
\hypertarget{System.IO.Compression.defaultBZip2CompressionLevel}{defaultBZip2CompressionLevel}
& int
& 9
& Default bzip2 compression level.

\\
\hypertarget{System.IO.Compression.defaultBZip2WorkFactor}{defaultBZip2WorkFactor}
& int
& 0
& Default bzip2 work factor.

\\
\hypertarget{System.IO.Compression.defaultDeflateCompressionLevel}{defaultDeflateCompressionLevel}
& int
& -1
& Default deflate compression level.

\\
\hypertarget{System.IO.Compression.fastestDeflateCompression}{fastestDeflateCompression}
& int
& 1
& Fastest deflate compression levvel.

\\
\hypertarget{System.IO.Compression.maximumBZip2WorkFactor}{maximumBZip2WorkFactor}
& int
& 250
& Maximum bzip2 work factor.

\\
\hypertarget{System.IO.Compression.minimumBZip2CompressionLevel}{minimumBZip2CompressionLevel}
& int
& 1
& Minumum bzip2 compression level.

\\
\hypertarget{System.IO.Compression.noDeflateCompression}{noDeflateCompression}
& int
& 0
& No deflate compression.

\\
\hypertarget{System.IO.Compression.optimalBZip2CompressionLevel}{optimalBZip2CompressionLevel}
& int
& 9
& Best bzip2 compression.

\\
\hypertarget{System.IO.Compression.optimalDeflateCompression}{optimalDeflateCompression}
& int
& 9
& Best deflate compression.

\\
\end{supertabular}

\end{flushleft}
\clearpage
\end{document}


